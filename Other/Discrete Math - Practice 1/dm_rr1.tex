\documentclass[14pt,a4paper]{scrartcl}
\usepackage[utf8]{inputenc}
\usepackage[english,russian]{babel}
\usepackage{misccorr,color,ragged2e,amsfonts,amsthm,graphicx,systeme,amsmath,mdframed,lipsum}
\renewcommand\qedsymbol{$\blacksquare$}
\renewcommand*{\proofname}{\text{Доведення}}
\theoremstyle{definition}
\newtheorem{defo}{Означення}[section]
\newtheorem*{teo}{Теорема}
\newtheorem*{example}{Приклад}
\theoremstyle{remark}
\newtheorem*{remark}{Зауваження}
\theoremstyle{definition}
\newtheorem*{consequence}{Наслідок}
\theoremstyle{definition}
\newtheorem{statement}{Утверждение}[section]
\newmdtheoremenv{boxteo}{Теорема}[section]
\setlength\parindent{0pt}
\begin{document}

\def\be{\begin{equation}}
\def\ee{\end{equation}}

\def\bd{\begin{defo}}
\def\ed{\end{defo}}

\def\bbt{\begin{boxteo}}
\def\ebt{\end{boxteo}}

2.b)\\
Скористуємося методом Квайна - Мак-Класкі. Випишемо кон'юнкти ДДНФ.
1. 1110\\
2. 1101\\
3. 1011\\
---------------\\
4. 1100\\
5. 1010\\
6. 1001\\
7. 0110\\
8. 0101\\
Застосуємо закон неповного склеювання:\\
9. 11-0 (1, 4)\\
10. 110- (2, 4)\\
11. 1-10 (1, 5)\\
12. -110 (1, 7)\\
13. 1-01 (2,6)\\
14. -101 (2, 8)\\
15. 101- (3, 5)\\
16. 10-1 (3, 6)\\
Далі скористаємося законом поглинання. Вилучимо кон'юнкти 1-8. Тобто задана функція $f$ має таку скорочену ДНФ:\\
$(x_1 \wedge x_2 \wedge \overline{x_4})\vee (x_1 \wedge x_2 \wedge \overline{x_3}) \vee (x_1 \wedge x_3 \wedge \overline{x_4}) \vee (x_2 \wedge x_3 \wedge \overline{x_4}) \vee (x_1 \wedge \overline{x_3} \wedge x_4) \vee (x_2 \wedge \overline{x_3} \wedge x_4) \vee (x_1 \wedge \overline{x_2} \wedge x_3) \vee (x_1 \wedge \overline{x_2} \wedge x_4)$\\
Далі складемо імлікантну таблицю, визначимо ядро та побудуємо спрощену імплікантну таблицю.\\

Ядровими імплікантами є $-110$ та $-101$.\\ Відповідні кон'юнкти:
$(x_2 \wedge x_3 \wedge \overline{x_4})$
та
$(x_2 \wedge \overline{x_3} \wedge x_4)$.
Викреслимо необхідні строки та стовбці. В результаті отримаємо спрощену імлікантну таблицю.\\
\pagebreak
\\
Випишемо у вигляді формули алгебри висловлень умову \\коректності тупикової ДНФ: $$
(A+B)\&(C+E)\&(D+F)\&(E+F) = (AC + BC + AE + BE)(F + DE) =$$ $$= ACF + BCF + AEF + BEF + ACBE + BCDE + BDE + ADE = $$ $$ = \textbf{скористуємося законом поглинання} =$$ $$ =
ACF + BCF + AEF + BEF + BDE + ADE
$$
Цей запис означає існування 6-ти тупикових форм. Переведемо:\\
$$A = (x_1 \wedge x_2 \wedge \overline{x_4})$$
$$B = (x_1 \wedge x_2 \wedge \overline{x_3})$$
$$C = (x_1 \wedge x_3 \wedge \overline{x_4})$$
$$D = (x_1 \wedge \overline{x_3} \wedge x_4)$$
$$E = (x_1 \wedge \overline{x_2} \wedge x_3)$$
$$F = (x_1 \wedge \overline{x_2} \wedge x_4)$$

Випишемо отримані тупикові ДНФ:\\
1. $(x_1 \wedge x_2 \wedge \overline{x_4}) \vee (x_1 \wedge x_3 \wedge \overline{x_4}) \vee (x_1 \wedge \overline{x_2} \wedge x_4) \vee
(x_2 \wedge x_3 \wedge \overline{x_4}) \vee (x_2 \wedge \overline{x_3} \wedge x_4) $
2. $  (x_1 \wedge x_2 \wedge \overline{x_3}) \vee (x_1 \wedge x_3 \wedge \overline{x_4}) \vee (x_1 \wedge \overline{x_2} \wedge x_4) \vee
(x_2 \wedge x_3 \wedge \overline{x_4}) \vee (x_2 \wedge \overline{x_3} \wedge x_4) $
3. $ (x_1 \wedge x_2 \wedge \overline{x_4}) \vee (x_1 \wedge \overline{x_2} \wedge x_3) \vee (x_1 \wedge \overline{x_2} \wedge x_4) \vee
(x_2 \wedge x_3 \wedge \overline{x_4}) \vee (x_2 \wedge \overline{x_3} \wedge x_4) $
4. $ (x_1 \wedge x_2 \wedge \overline{x_3}) \vee (x_1 \wedge \overline{x_2} \wedge x_3) \vee (x_1 \wedge \overline{x_2} \wedge x_4) \vee
(x_2 \wedge x_3 \wedge \overline{x_4}) \vee (x_2 \wedge \overline{x_3} \wedge x_4)$
5. $(x_1 \wedge x_2 \wedge \overline{x_3}) \vee(x_1 \wedge \overline{x_3} \wedge x_4) \vee (x_1 \wedge \overline{x_2} \wedge x_3) \vee
(x_2 \wedge x_3 \wedge \overline{x_4}) \vee (x_2 \wedge \overline{x_3} \wedge x_4)$
6. $(x_1 \wedge x_2 \wedge \overline{x_4}) \vee(x_1 \wedge \overline{x_3} \wedge x_4) \vee (x_1 \wedge \overline{x_2} \wedge x_3) \vee
(x_2 \wedge x_3 \wedge \overline{x_4}) \vee (x_2 \wedge \overline{x_3} \wedge x_4)$\\
Отримані тупикові ДНФ повністю збігаються з результатом, отриманим за методом карт Картно - 6 тупикових ДНФ по 5 кон'юнктів.
\pagebreak\\
1. Дослідимо на належність основним функціонально-замкненим класам функцію $f_1 (x, y, z) = \left( \overline{x} \rightarrow \overline{y} \right) \vee \overline{z} = (x \vee \overline{y}) \wedge \overline{z}.$
$$
f_1(0, 0, 0) = 1; \quad f_1(1, 1, 1) = 0 \Longrightarrow \begin{gathered}
 f_1 \notin T_1;\\
 f_1 \notin T_0;\\
 f_1 \notin M;\\
\end{gathered}
$$
Обчислимо двоїсту функцію до $f_1:$
$$f_1^*(x,y,z) = (x \wedge \overline{y}) \vee \overline{z} $$
$f^*_1(1, 0, 1) = 1 \neq f_1(0, 0, 0)$, отже $f_1 \notin S$.\\
Обчислимо поліном Жегалкіна. Отримаємо, функція не є лінійною $f_1 \notin L$:\\
$$
f_1(x, y, z) = \overline{(\overline{x} \wedge y) }\wedge \overline{z} =
(1 \oplus(1 \oplus x)y)(1\oplus z) = 1 \oplus y \oplus z \oplus xy \oplus xz \oplus xyz
$$
2. Дослідимо на належність основним функціонально-замкненим класам функцію
$f_2(x, y, z) = (\overline{x} \leftrightarrow \overline{y}) \oplus (\overline{x} \land z) = 1 \oplus x \oplus y \oplus (z \land (1\oplus x)) =
\mathbf{1 \oplus x \oplus y \oplus z \oplus (x \land z)}
$
$$
f_2(0, 0, 0) = 1; \quad f_2(1, 1, 1) = 1; \quad f_2(1,0,1) = 0; \Longrightarrow \begin{gathered}
 f_2\notin T_0;\\
 f_2 \in T_1;\\
 f_2 \notin M;
\end{gathered}
$$
Обчислимо двоїсту функцію до $f_2:$
$$f_2^*(x,y,z) = \overline{1 \oplus \overline{x} \oplus \overline{y} \oplus \overline{z} \oplus (\overline{x} \land \overline{z})} =
1 \oplus x \oplus 1 \oplus 1 \oplus 1 \oplus y \oplus z \oplus x \oplus (x \land z)=
$$
$$ = y \oplus (x \land z) \quad | \quad f^*_2 (0,0,0) = 0 \neq f_2(0,0,0) \Longrightarrow f_2 \notin S $$
Обчислимо поліном Жегалкіна. Отримаємо, функція не є лінійною $f_2 \notin L$:
$$f_2 (x,y,z) = 1 \oplus x\oplus y\oplus z \oplus xz$$
3. Дослідимо на належність основним функціонально-замкненим класам\\ функцію
$f_3(x, y, z) = \overline{x} \land \overline{y} \land z$;\\
$$f_3(0,0,0) = 0; \quad f_3(1,1,1) = 0; \quad f(0,0,1) = 1 \Longrightarrow \begin{gathered}
 f_3 \notin T_1\\
 f_3 \in T_0\\
 f_3 \notin M
\end{gathered}
$$
Обчислимо двоїсту функцію до $f_3:$
$$f_3^*(x,y,z) = \overline{x \land y \land \overline{z}} = \overline{x} \lor \overline{y} \lor z \Longrightarrow f_3^*(0,0,0) = 1 \neq  f_3(0,0,0)\Longrightarrow f_3 \notin S
$$
Обчислимо поліном Жегалкіна. Отримаємо, функція не є лінійною $f_3 \notin L$:
$$f_3 (x,y,z) = (x \oplus 1)(y\oplus 1)z = z \oplus xz \oplus yz \oplus xyz$$
\pagebreak\\
Заповнимо таблицю Поста для заданого набору фунцій:\\
$$\begin{gathered}
 \textbf{Функція} \\
 (x \wedge \overline{y}) \vee \overline{z} \\
 (\overline{x} \leftrightarrow \overline{y}) \oplus (\overline{x} \land z)\\
 \overline{x} \land \overline{y} \land z
\end{gathered}
\left|
\quad
\begin{gathered}
T_0\\
-\\
-\\
+
\end{gathered}
\quad
\begin{gathered}
 T_1\\
 -\\
 +\\
 -
\end{gathered}
\quad
\begin{gathered}
 M\\
 -\\
 -\\
 -
\end{gathered}
\quad
\begin{gathered}
 S\\
 -\\
 -\\
 -
\end{gathered}
\quad
\begin{gathered}
 L\\
 -\\
 -\\
 -
\end{gathered}
\quad
 \right|
$$Отже, для кожного з п'яти основних функціонально-замкнених класів існує принаймі одна функція з набору К, яка цьому класу не належить.\\
\textbf{Таким чином, за теоремою Поста набір К є функціонально повним.}
\end{document}
