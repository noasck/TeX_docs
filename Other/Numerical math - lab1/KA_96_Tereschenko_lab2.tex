\documentclass[14pt,a4paper]{scrartcl}
\usepackage[utf8]{inputenc}
\usepackage{ragged2e}
\usepackage[english,russian]{babel}
\usepackage{misccorr,color,ragged2e,amsfonts,amsthm,graphicx,systeme,amsmath,mdframed,lipsum}
\renewcommand\qedsymbol{$\blacksquare$}
\renewcommand*{\proofname}{\text{Доведення}}
\theoremstyle{definition}
\newtheorem{defo}{Означення}[section]
\newtheorem*{teo}{Теорема}
\newtheorem*{example}{Приклад}
\theoremstyle{remark}
\newtheorem*{remark}{Зауваження}
\theoremstyle{definition}
\newtheorem*{consequence}{Наслідок}
\theoremstyle{definition}
\newtheorem{statement}{Утверждение}[section]
\newmdtheoremenv{boxteo}{Теорема}[section]
\setlength\parindent{0pt}
\DeclareMathOperator*\lowlim{\underline{lim}}
\DeclareMathOperator*\uplim{\overline{lim}}


% Default fixed font does not support bold face
\DeclareFixedFont{\ttb}{T1}{txtt}{bx}{n}{12} % for bold
\DeclareFixedFont{\ttm}{T1}{txtt}{m}{n}{12}  % for normal

% Custom colors
\usepackage{color}
\definecolor{deepblue}{rgb}{0,0,0.5}
\definecolor{deepred}{rgb}{0.6,0,0}
\definecolor{deepgreen}{rgb}{0,0.5,0}

\usepackage{listings}

% Python style for highlighting
\newcommand\pythonstyle{\lstset{
language=Python,
basicstyle=\ttm,
otherkeywords={self},             % Add keywords here
keywordstyle=\ttb\color{deepblue},
emph={MyClass,__init__},          % Custom highlighting
emphstyle=\ttb\color{deepred},    % Custom highlighting style
stringstyle=\color{deepgreen},
frame=tb,                         % Any extra options here
showstringspaces=false            %
}}

\definecolor{javared}{rgb}{0.6,0,0} % for strings
\definecolor{javagreen}{rgb}{0.25,0.5,0.35} % comments
\definecolor{javapurple}{rgb}{0.5,0,0.35} % keywords
\definecolor{javadocblue}{rgb}{0.25,0.35,0.75} % javadoc

\lstset{language=C++,
basicstyle=\ttfamily,
keywordstyle=\color{javapurple}\bfseries,
stringstyle=\color{javared},
commentstyle=\color{javagreen},
morecomment=[s][\color{javadocblue}]{/**}{*/},
numbers=left,
numberstyle=\tiny\color{black},
stepnumber=2,
numbersep=10pt,
tabsize=4,
showspaces=false,
showstringspaces=false}


% Python environment
\lstnewenvironment{python}[1][]
{
\pythonstyle
\lstset{#1}
}
{}

% Python for external files
\newcommand\pythonexternal[2][]{{
\pythonstyle
\lstinputlisting[#1]{#2}}}

% Python for inline
\newcommand\pythoninline[1]{{\pythonstyle\lstinline!#1!}}
%
% \begin{python}
% class MyClass(Yourclass):
%     def __init__(self, my, yours):
%         bla = '5 1 2 3 4'
%         print bla
% \end{python}

\begin{document}



\def\be{\begin{equation}}
\def\ee{\end{equation}}

\def\bd{\begin{defo}}
\def\ed{\end{defo}}

\def\bbt{\begin{boxteo}}
\def\ebt{\end{boxteo}}

\begin{titlepage}
\centering
	\vspace{1cm}
	{ МІНІСТЕРСТВО ОСВІТИ І НАУКИ УКРАЇНИ\\
  НАВЧАЛЬНО-НАУКОВИЙ КОМПЛЕКС\\
  ``ІНСТИТУТ ПРИКЛАДНОГО СИСТЕМНОГО АНАЛІЗУ``\\
  НАЦІОНАЛЬНОГО ТЕХНІЧНОГО УНІВЕРСИТЕТУ УКРАЇНИ\\
  ``КИЇВСЬКИЙ ПОЛІТЕХНІЧНИЙ ІНСТИТУТ ІМЕНІ ІГОРЯ СІКОРСЬКОГО``\\
  КАФЕДРА МАТЕМАТИЧНИХ МЕТОДІВ  СИСТЕМНОГО АНАЛІЗУ\\\par}
	\vspace{5cm}
	{\large Лабораторна робота №2 \\
\textbf{ПРЯМІ МЕТОДИ РОЗ'ЯЗАННЯ СИСТЕМ ЛІНІЙНИХ АЛГЕБРАЇЧНИХ РІВНЯНЬ.\\ Варіант-27} \\ \par}
	\vfill
  \begin{flushright}
  Виконав:\\
   Терещенко Денис, КА-96\\
   Прийняла:\\
   Шубенкова І. А.\\
  \end{flushright}


	\vfill

% Bottom of the page
	{\large КИЇВ - 2020 \par}
\end{titlepage}


\begin{center}

ЗВІТ


\end{center}

\section{Постановка завдання.}

1. Проаналізувати конкретну СЛАР та обгрунтувати обрання методу розв’язання. Реалізувати обраний метод у вигляді окремої процедури або методу відповідного об'єкту "СЛАР".\\
2. Розвязати СЛАР з точністю $\varepsilon = 10^{-5}$.\\
3. За необхідності зробити ітераційне уточнення. \\
4. Обчислити $A^{-1}$ та $\det A$. Визначник матриці повинен бути бічним результатом процедури з першого пункту. Обернену матрицю треба знайти, скориставшись тією ж процедурою, передаючи їй як вектори правої частини стовпці одиничної матриці. Тому краще за все реалізувати будь-який прямий метод у формі, що дозоляє працювати з низкою стовпців правої частини.

\section{Аналітична частина.}
Завдання варіанту:
$$
A = \begin{bmatrix}
  7.03 & 0.94 & 1.13 & 1.135 & -0.81 \\
	1.26 & 3.39 & 1.3 & -1.63 & -1.53  \\
	0.81 & -2.46 & 6.21 & 2.1 & -1.067\\
	1.345 & 0.16 & 2.1 & 5.33 & 16 \\
	1.29 & 0.87 & 1.333 & -8 & 15
\end{bmatrix} \qquad
b = \begin{bmatrix} 2.1 \\ 0.84 \\ -3.44 \\ -0.92 \\ -1.47 \end{bmatrix}
$$

Матриця є несиметричною, тому будемо застосовувати метод Гаусса.\\
\section{Лістинг програми.}

{\small
\begin{lstlisting}
#include "stdafx.h"
#include <iostream>
#include <vector>

using namespace std;

void Decomposition(vector <vector <double>> A, vector <vector <double>> &L,
vector <vector <double>> &U, int n)
{
	for (int j = 0; j < n; j++) {
		L[j][0] = A[j][0];
		U[0][j] = A[0][j] / L[0][0];
	}
	for (int i = 1; i < n; i++) {
		for (int j = i; j < n; j++) {
			for (int k = 0; k < i; k++) {
				U[i][j] += (L[i][k] * U[k][j]);
				L[j][i] += (L[j][k] * U[k][i]);
			}
			L[j][i] = A[j][i] - L[j][i];
			U[i][j] = (1 / L[i][i])*(A[i][j] - U[i][j]);

		}
	}

}
void print_matrix(vector <vector <double>> A, int n)
{
	for (int i = 0; i < n; i++)
	{
		for (int j = 0; j < n; j++)
		{
			cout << "\t" << A[i][j] << "\t";
		}
		cout << endl;
	}
}
void print_matrixv(vector <double> A, int n)
{
	for (int i = 0; i < n; i++)
	{
		cout << "\t" << A[i] << "\t";
	}
	cout << endl;
}
void inverse(vector <vector <double>> A, vector <vector <double>> &B, int n) {
	for (int i = 0; i < n; i++)
		for (int j = 0; j < n; j++) {
			if (i == j) B[i][j] = 1 / A[i][j];
			else {
				if (i < j) B[i][j] = 0;
				else
					for (int k = 0; k < i; k++) {
						B[i][j] += (((-1 / A[i][i])*B[k][j]) * A[i][k]);
					}
			}
		}
}
void inverse2(vector <vector <double>> A, vector <vector <double>> &B, int n) {
	for (int i = 0; i < n; i++)
		for (int j = 0; j < n; j++) {
			if (i == j) B[i][j] = 1 / A[i][j];
			else {
				if (i > j) B[i][j] = 0;
				else
					for (int k = 0; k < j; k++) {
						B[i][j] += (((-1 / A[i][i])*B[i][k]) * A[k][j]);
					}
			}
		}
}
void monm(vector <vector <double>> A, vector <vector <double>> B,
	vector <vector <double>> &R, int n)
{
	for (int i = 0; i < n; i++)
		for (int j = 0; j < n; j++)
			for (int k = 0; k < n; k++)
				R[i][j] += A[i][k] * B[k][j];
}
void calc_monv(vector <vector <double>> A, vector <double> B,
	vector <double> &R, int n)
{
	for (int i = 0; i < n; i++)
		for (int j = 0; j < n; j++)
				R[i] += A[i][j] * B[j];
}
void vonm(vector <double> A, vector <vector <double>> B,
	vector <double> &R, int n)
{
	for (int i = 0; i < n; i++)
		for (int j = 0; j < n; j++)
				R[i] += A[j] * B[i][j];
}
void vmv(vector<double> b, vector <double> x, vector<double>&r,int n) {
	for (int i = 0; i < n; i++) {
		r[i] = b[i] - x[i];
	}
}
int main()
{
	int n;
	cin >> n;
	double n1;
	vector <vector <double>> A(n), L(n), U(n), R(n), Li(n), Ui(n), E(n);
	vector <double> B(n,0), Y(n,0), X(n,0), r(n,0),r1(n,0);
	for (size_t row = 0; row < n; ++row) {
		A[row].resize(n);
		L[row].resize(n);
		U[row].resize(n);
		R[row].resize(n);
		Li[row].resize(n);
		Ui[row].resize(n);
		E[row].resize(n);
	}
	for (int i = 0; i < n; i++)
	{
		for (int j = 0; j < n; j++)
		{
			cin >> n1;
			A[i][j] = n1;
			L[i][j] = 0;
			U[i][j] = 0;
			R[i][j] = 0;
			Ui[i][j] = 0;
			Li[i][j] = 0;
			E[i][j] = 0;
		}
	}
	for (int i = 0; i < n; i++)
	{
		cin >> n1;
		B[i] = n1;
	}
	Decomposition(A, L, U, n);
	cout << "|----- input matrix -----|" << endl;
	print_matrix(A, n);
	cout << "|----- U matrix -----|" << endl;
	print_matrix(U, n);
	cout << "|----- L matrix -----|" << endl;
	print_matrix(L, n);
	monm(L, U, R, n);
	cout << "|----- L*U matrix -----|" << endl;
	print_matrix(R, n);
	inverse(L, Li, n);
	cout << "|----- inverse L matrix -----|" << endl;
	print_matrix(Li, n);
	inverse2(U, Ui, n);
	cout << "|----- inverse U matrix -----|" << endl;
	print_matrix(Ui, n);
	calc_monv(Li,B,Y,n);
	cout << "|----- L^(-1)*b -----|" << endl;
	print_matrixv(Y, n);
	calc_monv(Ui,Y, X, n);
	cout << "|----- solution -----|" << endl;
	print_matrixv(X, n);
	calc_monv(A, X, r, n);
	vmv(B, r, r1, n);
	cout << "|----- residual vector -----|" << endl;
	print_matrixv(r1, n);
	cout << "|----- inverse A matrix -----|" << endl;
	print_matrix(Ai, n);
	calc_monv(Ai,A,Y,n);
	cout << "|----- det A -----|" << endl; calc_det(L, U);
	return 0;
}
\end{lstlisting}}
\section{Результати роботи.}
\begin{verbatim}
|----- input matrix -----|
	7.03        0.94        1.13        1.135       -0.81
	1.26        3.39        1.3     -1.63       -1.53
	0.81        -2.46       6.21        2.1     -1.067
	1.345       0.16        2.1     5.33        16
	1.29        0.87        1.333       -8      15
|----- U matrix -----|
	1       0.133713        0.16074     0.161451        -0.11522
	0       1       0.340668        -0.569119       -0.429866
	0       0       1       0.0729795       -0.298746
	0       0       0       1       3.36677
	0       0       0       0       1
|----- L matrix -----|
	7.03        0       0       0       0
	1.26        3.22152     0       0       0
	0.81        -2.56831        6.95474     0       0
	1.345       -0.0198435      1.89057     4.96358     0
	1.29        0.697511        0.888027        -7.87611        42.2308
|----- L*U matrix -----|
	7.03        0.94        1.13        1.135       -0.81
	1.26        3.39        1.3     -1.63       -1.53
	0.81        -2.46       6.21        2.1     -1.067
	1.345       0.16        2.1     5.33        16
	1.29        0.87        1.333       -8      15
|----- inverse L matrix -----|
	0.142248        0       0       0       0
	-0.0556358      0.310412        0       0       0
	-0.0371129      0.114632        0.143787        0       0
	-0.0246319      -0.0424208      -0.0547666      0.201467        0
	-0.00723972     -0.015449       -0.0132376      0.037574        0.0236794
|----- inverse U matrix -----|
	1       -0.133713       -0.115188       -0.229143       0.794801
	0       1       -0.340668       0.59398     -1.6717
	0       0       1       -0.0729795      0.544451
	0       0       0       1       -3.36677
	0       0       0       0       1
|----- L^(-1)*b -----|
	0.29872     0.143911        -0.476273       -0.0843135      -0.05202
|----- solution -----|
	0.312312        0.343043        -0.498442       0.0908259       -0.05202
|----- residual vector -----|
	4.44089e-16     5.55112e-16     8.88178e-16     5.55112e-16     6.66134e-16
|----- inverse A matrix -----|
	0.144984     -0.0133376     -0.0200064     -0.0100561     0.00850358
	-0.044635     0.298825     -0.0517625     0.0427649     -0.0291451
	-0.0167602     0.100349     0.150524     -0.00370488     0.0206768
	0.00187008     -0.00698634     -0.0104795     0.0740976     -0.0802018
	-0.0075509     -0.00952683     -0.0142903     0.0404361     0.021947
|----- det A -----|
28341
\end{verbatim}
\textbf{Результат}. З точністю $\varepsilon = 10^{-5} $ отримали наступний роз'язок системи лінійних рівнянь:
$$
x_1=0.312312 \quad
x_2=0.343047 \quad
x_3=-0.498441 \quad
x_4=0.0908238 \quad
x_5=-0.0520194 \quad
$$
$$
\det A = 28341.234$$$$ \qquad A^{-1} = \begin{bmatrix}
0.144984   &  -0.013337  &   -0.020006  &   -0.010056  &   0.008508\\
-0.044635   &  0.298825   &  -0.051762  &   0.042764 &    -0.029145\\
-0.016760  &   0.100349   &  0.150524   &  -0.003704 &   0.020676\\
0.001870  &   -0.006986  &   -0.010479  &   0.074097   &  -0.080201\\
-0.007550  &  -0.009526  &  -0.014290   &  0.040436  &   0.021947 \\
\end{bmatrix}
$$
\end{document}
