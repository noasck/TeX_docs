\documentclass[14pt,a4paper]{scrartcl}
\usepackage[utf8]{inputenc}
\usepackage{ragged2e}
\usepackage[english,russian]{babel}
\usepackage{misccorr,color,ragged2e,amsfonts,amsthm,graphicx,systeme,amsmath,mdframed,lipsum}
\renewcommand\qedsymbol{$\blacksquare$}
\renewcommand*{\proofname}{\text{Доведення}}
\theoremstyle{definition}
\newtheorem{defo}{Означення}[section]
\newtheorem*{teo}{Теорема}
\newtheorem*{example}{Приклад}
\theoremstyle{remark}
\newtheorem*{remark}{Зауваження}
\theoremstyle{definition}
\newtheorem*{consequence}{Наслідок}
\theoremstyle{definition}
\newtheorem{statement}{Утверждение}[section]
\newmdtheoremenv{boxteo}{Теорема}[section]
\setlength\parindent{0pt}
\DeclareMathOperator*\lowlim{\underline{lim}}
\DeclareMathOperator*\uplim{\overline{lim}}


% Default fixed font does not support bold face
\DeclareFixedFont{\ttb}{T1}{txtt}{bx}{n}{12} % for bold
\DeclareFixedFont{\ttm}{T1}{txtt}{m}{n}{12}  % for normal

% Custom colors
\usepackage{color}
\definecolor{deepblue}{rgb}{0,0,0.5}
\definecolor{deepred}{rgb}{0.6,0,0}
\definecolor{deepgreen}{rgb}{0,0.5,0}

\usepackage{listings}

% Python style for highlighting
\newcommand\pythonstyle{\lstset{
language=Python,
basicstyle=\ttm,
otherkeywords={self},             % Add keywords here
keywordstyle=\ttb\color{deepblue},
emph={MyClass,__init__},          % Custom highlighting
emphstyle=\ttb\color{deepred},    % Custom highlighting style
stringstyle=\color{deepgreen},
frame=tb,                         % Any extra options here
showstringspaces=false            %
}}

\definecolor{javared}{rgb}{0.6,0,0} % for strings
\definecolor{javagreen}{rgb}{0.25,0.5,0.35} % comments
\definecolor{javapurple}{rgb}{0.5,0,0.35} % keywords
\definecolor{javadocblue}{rgb}{0.25,0.35,0.75} % javadoc

\lstset{language=C++,
basicstyle=\ttfamily,
keywordstyle=\color{javapurple}\bfseries,
stringstyle=\color{javared},
commentstyle=\color{javagreen},
morecomment=[s][\color{javadocblue}]{/**}{*/},
numbers=left,
numberstyle=\tiny\color{black},
stepnumber=2,
numbersep=10pt,
tabsize=4,
showspaces=false,
showstringspaces=false}


% Python environment
\lstnewenvironment{python}[1][]
{
\pythonstyle
\lstset{#1}
}
{}

% Python for external files
\newcommand\pythonexternal[2][]{{
\pythonstyle
\lstinputlisting[#1]{#2}}}

% Python for inline
\newcommand\pythoninline[1]{{\pythonstyle\lstinline!#1!}}
%
% \begin{python}
% class MyClass(Yourclass):
%     def __init__(self, my, yours):
%         bla = '5 1 2 3 4'
%         print bla
% \end{python}

\begin{document}



\def\be{\begin{equation}}
\def\ee{\end{equation}}

\def\bd{\begin{defo}}
\def\ed{\end{defo}}

\def\bbt{\begin{boxteo}}
\def\ebt{\end{boxteo}}

\begin{titlepage}
\centering
	\vspace{1cm}
	{ МІНІСТЕРСТВО ОСВІТИ І НАУКИ УКРАЇНИ\\
  НАВЧАЛЬНО-НАУКОВИЙ КОМПЛЕКС\\
  ``ІНСТИТУТ ПРИКЛАДНОГО СИСТЕМНОГО АНАЛІЗУ``\\
  НАЦІОНАЛЬНОГО ТЕХНІЧНОГО УНІВЕРСИТЕТУ УКРАЇНИ\\
  ``КИЇВСЬКИЙ ПОЛІТЕХНІЧНИЙ ІНСТИТУТ ІМЕНІ ІГОРЯ СІКОРСЬКОГО``\\
  КАФЕДРА МАТЕМАТИЧНИХ МЕТОДІВ  СИСТЕМНОГО АНАЛІЗУ\\\par}
	\vspace{5cm}
	{\large Лабораторна робота №3 \\
\textbf{ІТЕРАЦІЙНІ МЕТОДИ РОЗ'ЯЗАННЯ СИСТЕМ ЛІНІЙНИХ
АЛГЕБРАЇЧНИХ РІВНЯНЬ\\ Варіант-27} \\ \par}
	\vfill
  \begin{flushright}
  Виконав:\\
   Терещенко Денис, КА-96\\
   Прийняла:\\
   Шубенкова І. А.\\
  \end{flushright}


	\vfill

% Bottom of the page
	{\large КИЇВ - 2020 \par}
\end{titlepage}


\begin{center}
ЗВІТ
\end{center}

\textbf{Мета роботи: } засвоїти ітераційні методи, які приходять на допомогу, якщо прямі методи застосувати неможливо.\\
\textbf{Хід роботи: } застосувати студентам для непарних номерів за списком групи – метод простої ітерації, для парних – метод Зейделя.

\section{Постановка завдання.}

1. Запрограмувати один з ітераційних методів, що зводяться до формули:
$$ x_{k+1} = Bx_k + d $$
2. Для методу простої ітерації достатньою умовою збігу ітерацій є наявність
діагональної переваги у системі. Вручну спробуйте перевести задану
систему рівнянь до виду "з діагональною перевагою".


\section{Аналітична частина.}
Завдання варіанту:
$$
A = \begin{bmatrix}
  7.03 & 0.94 & 1.13 & 1.135 & -0.81 \\
	1.26 & 3.39 & 1.3 & -1.63 & -1.53  \\
	0.81 & -2.46 & 6.21 & 2.1 & -1.067\\
	1.345 & 0.16 & 2.1 & 5.33 & 16 \\
	1.29 & 0.87 & 1.333 & -8 & 15
\end{bmatrix} \qquad
b = \begin{bmatrix} 2.1 \\ 0.84 \\ -3.44 \\ -0.92 \\ -1.47 \end{bmatrix}
$$
Реалізуємо розв'язок заданої системи за методом простих ітерацій. Критерієм закінчення ітераційного процесу є норма вектору
нев'язки $\left| \left| \xi_{k+1}  \right|  \right| \leq \varepsilon = 10^{-5} $. Підрахувавши, отримали, що $ a = 38.501 > 1$. Тому м-цю вигідно привести до системи з діагональною перевагою. Тобто:

$$
A^* = \begin{bmatrix}
7.03&0.94&1.13&1.135&-0.81\\
1.26&3.39&1.3&-1.637&-1.53\\
0.81&-2.46&6.21&2.1&-1.067\\
1.345&0.16&2.1&5.33&16\\
2.635&1.03&3.433&-2.67&31\\
\end{bmatrix}
\qquad b^* = \begin{bmatrix}
2.1\\
0.84\\
-3.44\\
-0.92\\
-2.39\\
\end{bmatrix}
$$
\pagebreak
\section{Лістинг програми.}

{\small
\begin{lstlisting}
#include <iostream>
#include <iomanip>
#include <cmath>
#define e 0.00001
using namespace std;
class Matr
{
private:
    int size;
    double **mas;
    double *mas1;
public:
    Matr()
    {
        size = 0;
        mas = NULL;
        mas1 = NULL;
    }
    Matr(int l)
    {
        size = l;
        mas = new double*[l];
        for (int i = 0; i < l; i++)
            mas[i] = new double[l];
        mas1 = new double[l];
    }
    void Add()
    {
        for (int i = 0; i < size; i++)
            for (int j = 0; j < size; j++)
                cin >> mas[i][j];
        for (int i = 0; i < size; i++)
        {
            cin >> mas1[i];
        }

    }
    void Print()
    {
        for (int i = 0; i < size; i++)
        {
            for (int j = 0; j < size; j++)
            {
                cout << setw(4) << mas[i][j] << " ";
            }
            cout << " " << mas1[i] << endl;
        }
    }
    void Preob()
    {
        double temp = 0;
        for (int k = 0; k < size; k++)
        {
            for (int i = 0; i < size; i++)
            {
                temp = mas[i][i]*(-1);
                mas1[i] /= temp;
                for (int j = 0; j <= size; j++)
                {
                    mas[i][j] /= temp;
                }
            }
        }
            for (int i = 0; i < size; i++)
            {
                mas1[i] *= -1;
                for (int j = 0; j < size; j++)
                    mas[i][i] = 0;
            }
    }
    double Norma(double **mas)
    {
        double sum = 0, max = 0;
            for (int i = 0; i < size; i++)
            {
                for (int j = 0; j < size; j++)
                {
                    sum += fabs(mas[i][j]);
                    if (sum > max) max = sum;
                }
                sum = 0;
            }
            return max;
    }
    double Pogr()
    {
        double eps = 0;
        double norm = Norma(mas);
        eps = -((1 - norm) / norm)*e;
        cout<< eps;
        return eps;
    }
    void Itera()
    {
        double *x = new double[size];
        double *x0 = new double[size];
        double *E = new double[size];
        double max = 0, per = 0;
        per = Pogr();
        for (int i = 0; i < size; i++)
            x0[i] = mas1[i];
        int counter = 0;
        do
        {
            for (int i = 0; i < size; i++)
            {
                x[i] = 0;
                for (int j = 0; j < size; j++)
                {
                    x[i] += mas[i][j] * x0[j];
                }
                x[i] += mas1[i];
                E[i] = fabs(x[i] - x0[i]);
            }
            max = 0;
            for (int i = 0; i < size; i++)
            {
                if (max < E[i]) max = E[i];
                x0[i] = x[i];

            }
            cout<< endl <<" Iter:" << counter ;
            cout<< endl<< "x:";
                        for (int i = 0; i < size; i++) cout<< " "<< x[i];
            cout<< endl<< "vec_xi:";
                        for (int i = 0; i < size; i++) cout<< " "<< E[i];

            cout<<endl << "abs_xi = " << max;
            counter++;
        } while (max > per && counter < 50);
        cout << endl << "Count of iterations: " << counter << endl;
        for (int i = 0; i < size; i++)
            cout << "x" << i + 1 << "=" << x[i] << " " << endl;
        delete[] x;
        delete[] x0;
        delete[] E;
    }
};
int main()
{
    setlocale(LC_ALL, "rus");
    int n;
    cout << "Insert number of variables:   ";
    cin >> n;
    Matr a(n);
    cout << "Insert matrix A:" << endl;
    a.Add();
    a.Preob();
    a.Itera();
    cout << endl;
    system("pause");
    return 0;
}




\end{lstlisting}}
\section{Результати роботи.}
\begin{verbatim}
Insert number of variables:   5
Insert matrix A:
7.03 0.94 1.13 1.135 -0.81
1.26 3.39 1.3 -1.63 -1.53
0.81 -2.46 6.21 2.1 -1.067
1.345 0.16 2.1 5.33 16
2.635 1.03 3.433 -2.67 31
2.1
0.84
-3.44
-0.92
-2.39
7.28131e-06
 Iter:0
x: 0.373613 0.231396 -0.449628 0.194261 -0.0642425
vec_xi: 0.0748932 0.0163914 0.104317 0.366869 0.0128543
abs_xi = 0.366869
 Iter:1
x: 0.301287 0.345758 -0.587743 0.0961662 -0.050018
vec_xi: 0.0723263 0.114361 0.138115 0.0980944 0.0142245
abs_xi = 0.138115
 Iter:2
x: 0.325672 0.384858 -0.497391 0.122701 -0.0408237
vec_xi: 0.0243854 0.0391004 0.0903525 0.026535 0.00919435
abs_xi = 0.0903525
 Iter:3
x: 0.302696 0.358054 -0.492476 0.0521751 -0.051916
vec_xi: 0.0229762 0.0268037 0.00491492 0.0705261 0.0110923
abs_xi = 0.0705261
 Iter:4
x: 0.315598 0.325792 -0.478153 0.0901388 -0.055691
vec_xi: 0.0129024 0.032262 0.0143226 0.0379637 0.00377509
abs_xi = 0.0379637
 Iter:5
x: 0.311046 0.332054 -0.506103 0.0935408 -0.0540321
vec_xi: 0.00455261 0.00626214 0.0279497 0.00340191 0.0016589
abs_xi = 0.0279497
 Iter:6
x: 0.314343 0.346849 -0.503894 0.100534 -0.050465
vec_xi: 0.00329719 0.0147947 0.0022091 0.0069931 0.00356711
abs_xi = 0.0147947
 Iter:7
x: 0.311292 0.349749 -0.500215 0.0876793 -0.0508792
vec_xi: 0.00305137 0.00289975 0.00367872 0.0128545 0.000414158
abs_xi = 0.0128545
 Iter:8
x: 0.31234 0.343105 -0.494393 0.0881561 -0.0522307
vec_xi: 0.00104861 0.0066443 0.00582248 0.0004768 0.00135152
abs_xi = 0.0066443
 Iter:9
x: 0.31206 0.340101 -0.497555 0.089854 -0.0527028
vec_xi: 0.000280179 0.00300328 0.00316227 0.0016979 0.000472096
abs_xi = 0.00316227
 Iter:10
x: 0.312641 0.342021 -0.499363 0.092678 -0.0520828
vec_xi: 0.000581357 0.00192013 0.00180844 0.00282395 0.000620036
abs_xi = 0.00282395
 Iter:11
x: 0.312291 0.344137 -0.499527 0.0913249 -0.0517525
vec_xi: 0.000350546 0.00211509 0.000163624 0.00135309 0.000330282
abs_xi = 0.00211509
 Iter:12
x: 0.312291 0.343828 -0.498129 0.0904228 -0.0518914
vec_xi: 4.69963e-10 0.000308499 0.0013979 0.000902031 0.0001389
abs_xi = 0.0013979
 Iter:13
x: 0.312237 0.342796 -0.49797 0.0902983 -0.0521136
vec_xi: 5.38186e-05 0.00103248 0.000158962 0.000124548 0.000222247
abs_xi = 0.00103248
 Iter:14
x: 0.312344 0.342594 -0.498368 0.0909474 -0.0521031
vec_xi: 0.000107005 0.000201147 0.00039805 0.000649104 1.05486e-05
abs_xi = 0.000649104
 Iter:15
x: 0.312331 0.343024 -0.49868 0.0910516 -0.0520055
vec_xi: 1.27045e-05 0.00042974 0.00031133 0.000104201 9.75754e-05
abs_xi = 0.00042974
 Iter:16
x: 0.312318 0.343242 -0.498526 0.0908716 -0.0519753
vec_xi: 1.29992e-05 0.000218252 0.000153421 0.000179941 3.02534e-05
abs_xi = 0.000218252
 Iter:17
x: 0.312297 0.343116 -0.498372 0.0907171 -0.0520139
vec_xi: 2.13064e-05 0.000126868 0.0001542 0.000154536 3.86349e-05
abs_xi = 0.000154536
 Iter:18
x: 0.31231 0.342973 -0.498374 0.0907815 -0.0520383
vec_xi: 1.26761e-05 0.000142955 1.85775e-06 6.44077e-05 2.43601e-05
abs_xi = 0.000142955
 Iter:19
x: 0.312316 0.342989 -0.498458 0.0908565 -0.0520288
vec_xi: 6.20809e-06 1.59754e-05 8.4249e-05 7.49506e-05 9.42544e-06
abs_xi = 8.4249e-05
 Iter:20
x: 0.312316 0.343059 -0.498476 0.0908593 -0.0520141
vec_xi: 3.91208e-07 7.02926e-05 1.82074e-05 2.85364e-06 1.47268e-05
abs_xi = 7.02926e-05
 Iter:21
x: 0.312311 0.343074 -0.498447 0.0908201 -0.0520142
vec_xi: 5.23624e-06 1.48555e-05 2.93597e-05 3.92433e-05 1.06672e-07
abs_xi = 3.92433e-05
 Iter:22
x: 0.312311 0.343045 -0.498427 0.0908097 -0.0520209
vec_xi: 3.82068e-07 2.823e-05 1.98201e-05 1.0372e-05 6.67984e-06
abs_xi = 2.823e-05
 Iter:23
x: 0.312312 0.34303 -0.498436 0.0908229 -0.052023
vec_xi: 1.49374e-06 1.54606e-05 8.77335e-06 1.31868e-05 2.11781e-06
abs_xi = 1.54606e-05
 Iter:24
x: 0.312313 0.343038 -0.498447 0.0908328 -0.0520205
vec_xi: 1.10445e-06 8.19397e-06 1.11425e-05 9.90124e-06 2.49407e-06
abs_xi = 1.11425e-05
 Iter:25
x: 0.312313 0.343048 -0.498447 0.0908292 -0.0520188
vec_xi: 6.1579e-07 9.74885e-06 1.82146e-07 3.62146e-06 1.7206e-06
abs_xi = 9.74885e-06
 Iter:26
x: 0.312312 0.343047 -0.498441 0.0908238 -0.0520194
vec_xi: 5.49887e-07 8.0571e-07 5.46246e-06 5.37404e-06 6.03655e-07
abs_xi = 5.46246e-06
Count of iterations: 27
x1=0.312312
x2=0.343047
x3=-0.498441
x4=0.0908238
x5=-0.0520194
\end{verbatim}
\textbf{Висновок}. Остаточно отримали за 27 ітерацій відповідь, що збігається з точним роз'язком системи, отриманим раніше.
$$
x_1=0.312312 \quad
x_2=0.343047 \quad
x_3=-0.498441 \quad
x_4=0.0908238 \quad
x_5=-0.0520194 \quad
$$
Тож, реалізацію методу простих ітерацій можна вважжати коректною.
\end{document}
