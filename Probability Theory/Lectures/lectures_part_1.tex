\documentclass[14pt,a4paper]{scrartcl}
\usepackage[utf8]{inputenc}
\usepackage{ragged2e}
% \usepackage{esint}
\usepackage[english,russian, ukrainian]{babel}
\usepackage{misccorr,color,ragged2e,amsfonts,amsthm,graphicx,systeme,amsmath,mdframed,lipsum}
\renewcommand\qedsymbol{$\blacksquare$}
\renewcommand*{\proofname}{\text{Доведення}}
\theoremstyle{definition}
\newtheorem*{defo}{Означення}
\newtheorem*{teo}{Теорема}
\newtheorem*{example}{Приклад}
\theoremstyle{remark}
\newtheorem*{remark}{Зауваження}
\theoremstyle{definition}
\newtheorem*{consequence}{Наслідок}
\theoremstyle{definition}
\newtheorem{statement}{Утверждение}[section]
\newmdtheoremenv{boxteo}{Теорема}[section]
\setlength\parindent{0pt}
\DeclareMathOperator*\lowlim{\underline{lim}}
\DeclareMathOperator*\uplim{\overline{lim}}
\newcommand\independent{\protect\mathpalette{\protect\independenT}{\perp}}
\def\independenT#1#2{\mathrel{\rlap{$#1#2$}\mkern2mu{#1#2}}}
%
% \makeatletter
% %% make esint definition in line with amsmath
% \@for\next:={int,iint,iiint,iiiint,dotsint,oint,oiint,sqint,sqiint,
%   ointctrclockwise,ointclockwise,varointclockwise,varointctrclockwise,
%   fint,varoiint,landupint,landdownint}\do{%
%     \expandafter\edef\csname\next\endcsname{%
%       \noexpand\DOTSI
%       \expandafter\noexpand\csname\next op\endcsname
%       \noexpand\ilimits@
%     }%
%   }
% \makeatother
% Default fixed font does not support bold face
\DeclareFixedFont{\ttb}{T1}{txtt}{bx}{n}{12} % for bold
\DeclareFixedFont{\ttm}{T1}{txtt}{m}{n}{12}  % for normal

% Custom colors
\usepackage{color}
\definecolor{deepblue}{rgb}{0,0,0.5}
\definecolor{deepred}{rgb}{0.6,0,0}
\definecolor{deepgreen}{rgb}{0,0.5,0}

\usepackage{listings}

% Python style for highlighting
\newcommand\pythonstyle{\lstset{
language=Python,
basicstyle=\ttm,
otherkeywords={self},             % Add keywords here
keywordstyle=\ttb\color{deepblue},
emph={MyClass,__init__},          % Custom highlighting
emphstyle=\ttb\color{deepred},    % Custom highlighting style
stringstyle=\color{deepgreen},
frame=tb,                         % Any extra options here
showstringspaces=false            %
}}

\definecolor{javared}{rgb}{0.6,0,0} % for strings
\definecolor{javagreen}{rgb}{0.25,0.5,0.35} % comments
\definecolor{javapurple}{rgb}{0.5,0,0.35} % keywords
\definecolor{javadocblue}{rgb}{0.25,0.35,0.75} % javadoc

\lstset{language=C++,
basicstyle=\ttfamily,
keywordstyle=\color{javapurple}\bfseries,
stringstyle=\color{javared},
commentstyle=\color{javagreen},
morecomment=[s][\color{javadocblue}]{/**}{*/},
numbers=left,
numberstyle=\tiny\color{black},
stepnumber=2,
numbersep=10pt,
tabsize=4,
showspaces=false,
showstringspaces=false}


% Python environment
\lstnewenvironment{python}[1][]
{
\pythonstyle
\lstset{#1}
}
{}

% Python for external files
\newcommand\pythonexternal[2][]{{
\pythonstyle
\lstinputlisting[#1]{#2}}}

% Python for inline
\newcommand\pythoninline[1]{{\pythonstyle\lstinline!#1!}}
%
% \begin{python}
% class MyClass(Yourclass):
%     def __init__(self, my, yours):
%         bla = '5 1 2 3 4'
%         print bla
% \end{python}

\begin{document}



\def\be{\begin{equation}}
\def\ee{\end{equation}}
\def\bd{\begin{defo}}
\def\ed{\end{defo}}
\def\bbt{\begin{boxteo}}
\def\ebt{\end{boxteo}}

\begin{center}
	\Huge \textbf{Собственно, теорвер...}
\end{center}

\tableofcontents

\pagebreak




\section{Абсолютно неперервні розподіли.}

\subsection{Рівномірний розподіл.}

Рівномірний розподіл на $[a,b]$. Графік функції щільності розподілу:\\
\begin{center} \includegraphics[scale=0.3]{images/2.png} \end{center}
Позначення: $\xi \sim U(a,b)$. Функція щільності має наступній вигляд:\\
$$
f_\xi (x) = \left\lbrace \begin{gathered}
 c, x \in [a,b] \\
 0, x \notin [a,b]
 \end{gathered} \right. , \text{ де } c = \frac{1}{b-a}
$$
Визначимо функцію розподілу: $$F_\xi (x) = \left\lbrace
\begin{gathered}
0, x \in (-\infty; a]\\
	\int\limits_{a}^{ t}{f_\xi (t) dt} = \frac{1}{b-a} x - \frac{a}{a-b}  = \frac{x-a}{b-a}, x \in (a,b] \\
1, x \in (b; + \infty)
\end{gathered} \right.   $$
Числові характеристики:\\
$ \mathbb{E} \xi  =   \int\limits_{a}^{ b}{ x  f_\xi (x) dx} = \frac{b^2 - a^2}{2b - a}  = \frac{a+b}{2} $ \\
$ \mathbb{E} \xi^2 =  \int\limits_{a}^{ b}{x^2 f_\xi (x) dx} = \frac{1}{b-a} \frac{x^3}{3} \Big|_a^b = \frac{b^3 - a^3}{3(b-a)} = \frac{a^2 + ab + b^2}{3}   $\\
$ \mathbb{D} \xi =  \frac{a^2 + ab + b^2}{3} - \frac{a^2 + 2ab + b^2}{ 4} = \frac{(a-b)^2}{12}  $\\
Величини залежать лише від довжини проміжку.\\ Нехай $[c,b] \subset [a,b]$, тоді знайдемо:
$$
\mathbb{P} \left\lbrace \xi \in [c,d] \right\rbrace = F_\xi(d) - F_\xi (c) = \frac{d-c}{b-a}
$$
\subsection{Експоненціальний розподіл.}
Розглядаємо $ \xi \sim Exp(\lambda), \lambda > 0$. \\
Щільність розподілу:
$$
f_\xi (x) =
 \left\lbrace \begin{gathered} \lambda e^{-\lambda x}, x \geq 0 \\
 0 , x <0 \end{gathered}
 \right.
 \qquad \qquad
 \begin{gathered}
 \includegraphics[scale=0.25]{images/3.png}
 \end{gathered}
$$
Запишемо функію розподілу:
$$
F_\xi (x) = \left\lbrace
\begin{gathered}
  \int\limits_{-\infty}^{ x}{ 0 dt} = 0,\qquad x < 0\\
	F(0)+  \int\limits_{0}^{x}{ \lambda e^ {-\lambda t} dt} = 1 - e^ {- \lambda x}, x \geq 0
\end{gathered} \right. \begin{gathered} \includegraphics[scale=0.25]{images/1.png}
\end{gathered}
$$
Знайдемо: $ \mathbb{P} \left\lbrace \xi \in [x,d] \right\rbrace  = (1 - e^ { - \lambda x})\Big|^d_c = e^{- \lambda c} - e^{- \lambda d}$.\\
Виведемо числові характеристики. Спочатку виведемо формулу $ \forall k \in \mathbb{N}: \mathbb{E} \xi^k:$
$$
 \int\limits_{- \infty}^{ +\infty}{ x^k * f_\xi (x) dx} = \lambda \int\limits_{- \infty}^{ +\infty}{ x^k e^{ - \lambda x} dx} = \frac{1}{ \lambda^k}  * \text{Г}(k+1) = \frac{k!}{\lambda^k}
$$
Користуючись цією формулою, отримаємо числові характеристики розподілу:

$$
\begin{gathered}
 \mathbb{E} \xi = (k=1) = \frac{1}{\lambda} \\
 \mathbb{E} \xi^2 = (k=2) = \frac{2}{\lambda^2} \\
 \mathbb{D} \xi =  \frac{2}{\lambda^2}  - \frac{1}{\lambda^2} = \frac{1}{\lambda^2}  \\
\end{gathered}
\qquad \qquad\qquad\qquad\qquad
\left|
\begin{gathered}
\textbf{Властивості Гамма-функції: }\\
 1. \text{Г}(\alpha) =  \int\limits_{0}^{ \infty}{ x^{\alpha -1 } e^{-x} dx}\\
 2. \text{Г} (n) = (n-1)!\\
 3. \text{Г} (\alpha+1) = \alpha * \text{Г}(\alpha)\\
 4. \text{Г} (1/2) = \sqrt{\pi}
\end{gathered}
\right|
$$

\begin{center}
	Експоненціальна величина описує час безвідмовної роботи приладу до моменту першої відмови. Це твердження не є вірним. Чому?
\end{center}

\textbf{Властивості експоненціального розподілу.}\\
1. Відсутність післядії. \\
$$ \xi \sim Exp(\lambda) \Rightarrow  \begin{gathered}
 \forall t, h > 0 \\
 \mathbb{P} \{ \xi > t+h \big| \xi >t \}  = \mathbb{P} \{ \xi > h \}
\end{gathered}
$$

\begin{proof}
$$
 \mathbb{P} \{ \xi > t+h \big| \xi >t \}  = \frac{ \mathbb{P} \{ \xi > t+h, \xi > t \}}{ \mathbb{P} \{ \xi > t \}} =
  \frac{ \mathbb{P} \{ \xi > t+h \}}{\mathbb{P} \{ \xi > t \} }=
 \frac{ e^{-\lambda (t+h) - e^{-\lambda \infty}} }{ e^{- \lambda t} - e^ {-\lambda \infty} } =
 $$
$$
= e^{-\lambda h} - e^{-\lambda\infty } = \mathbb{P} \left\lbrace \xi  > h \right\rbrace
$$
\end{proof}

2. Стійкість відносно min.\\
$ \xi_1, \xi_2 ... , \xi_n$ - незалежні.$ \left(  \begin{gathered}
\xi_1 \sim Exp(\lambda_1) \\
 \xi_2 \sim Exp(\lambda_2) \\
 ...\\
 \xi_n \sim Exp(\lambda_n) \\
\end{gathered} \right) $  $ \Rightarrow \min \left\lbrace  \xi_1 , \xi_2, .,. , \xi_n \right\rbrace  \sim Exp (  \sum\limits_{i = 1}^{ n}{ \lambda_i})$
\begin{proof}

$$
F_{min(\xi_1, ..., \xi_n)} (x) = \mathbb{P} \left\lbrace \min \left(\xi_1, ... , \xi_n  \right) < x  \right\rbrace  = 1 - \mathbb{P} \left\lbrace \min \left(\xi_1, ... , \xi_n  \right) \geq  x  \right\rbrace =
$$
$$
1 - \mathbb{P} \left\lbrace  \xi_1 \geq x, ... , \xi_n \geq x \right\rbrace = 1 - \mathbb{P} \left\lbrace \xi_1 \geq  x \right\rbrace \cdot ... \cdot \mathbb{P} \left\lbrace   \xi_n \geq 1\right\rbrace = 1 - e ^ { - \lambda_1 x} \cdot e^ { -\lambda_2 x} \cdot ... \cdot e^{- \lambda_n x} =
$$
$$
 = F_{Exp(\lambda_1 + \lambda_2 + ... + \lambda_n)} (x), x\geq 0
$$

\end{proof}
Використання: нехай є прилад, що складається з $n$ блоків. Для коректної роботи приладу необхідно коректна робота всіх блоків. \\
Позначимо: $\xi_i$ - час роботи блоку $i, i = 1, ..., n$.\\
Час роботи всього приладу: $ \xi = \min \left\lbrace \xi_1, \xi_2, ..., \xi_n \right\rbrace $.
\begin{example}
	Прилад - 10 блоків. Кожний з них з ймовірністю 0.99 може пропрацювати 1000 годин. Знайти середній час роботи всього приладу та ймовірність того, що він пропрацює 500 год.\\
	Розглянемо блок і-тий: \\$ \xi_i = Exp(\lambda_i)$.\\
	$\mathbb{P} \left\lbrace \xi_i \geq 1000 \right\rbrace = 0.99 = e ^ {-1000 \lambda} \Longrightarrow \lambda = \frac{ \ln{0.99}}{-1000} \approx 10^{-5}$\\
	$\mathbb{E} \xi_i = \frac{1}{\lambda} = 10^5 $\\
	Для всього приладу:\\
	$ \xi = \min \left\lbrace \xi_1, ... , \xi_n \right\rbrace  \sim Exp( \sum\limits_{i = 1}^{ 10}{\lambda_i}) = Exp(10^{-4})$\\
	$ \mathbb{E}\xi = 10^4 \qquad \mathbb{P} \left\lbrace \xi \geq 500 \right\rbrace = e^{-500 * 10^{-4}} = e^{-0.005} \approx 0.95$
\end{example}
3. Inter-arrival times dependency.
\begin{boxteo}
Розглянемо потік Пуассона з інтенсивністю
 $$\lambda \Rightarrow  N(s,t) \sim Pois(\lambda(t-s)) $$
\begin{center} \includegraphics[scale=0.3]{images/4.png} \end{center}
$\tau_i - $ inter-arrival times.
 $ \Rightarrow\left( \tau_i , i \in \mathbb{Z} \right) - $ незалежні, та $Exp(\lambda).$
\end{boxteo}
\subsection{ Гаусівський (нормальний) розподіл.}
\textbf{Стандартний Гаусівський розподіл.} Позначення:
$$
\xi \sim N (0, 1^2)
$$
Щільність розподілу:
$
f_{\xi} (x) = C \cdot e^{- \frac{x^2}{2} }
$.
З умови нормування та властивостей Гамма-функції:
$$  1 = \int\limits_{- \infty}^{ +\infty}{ f_ \xi (x) dx} = 2C  \int\limits_{0}^{ \infty}{ e^ { -\frac{x^2}{2} } dx} = \left|  \begin{gathered}
 \frac{x^2}{2} = t  \\
 x = \sqrt{2t} \\
 dx = \frac{dt}{\sqrt{2t}}
\end{gathered} \right|  =  2C  \int\limits_{0}^{ +\infty}{e^{-t} \frac{dt}{\sqrt{2} \cdot \sqrt{t}} } = $$
$$
= \sqrt{2} C  \int\limits_{0}^{\infty}{ t^{-1/2} e^{-t}dt} = \sqrt{2} C \text{Г}(1/2) = \sqrt{2\pi}C \Longrightarrow C = \frac{1}{\sqrt{2\pi}}
$$
$$
\begin{gathered}
\text{Остаточно, }\\
\text{щільність розподілу: }\\
f_ \xi (x) = \frac{1}{ \sqrt{2\pi} } e^{ - \frac{x^2}{2} }, \quad x \in \mathbb{R}
\end{gathered}\qquad\qquad \begin{gathered} \includegraphics[scale=0.3]{images/5.png} \end{gathered}
$$
Знайдемо числові характеристики розподілу:\\
$ \mathbb{E} \xi =  \int\limits_{- \infty}^{ +\infty}{ x * \frac{1}{ \sqrt{2\pi} } e^{ - \frac{x^2}{2} } dx} = 0$ (Функція щільності парна)\\
$\mathbb{D} \xi = \mathbb{E} \xi^2 =  \int\limits_{- \infty }^{ +\infty}{ x^2 \frac{1}{ \sqrt{2\pi} } e^{ - \frac{x^2}{2} } dx} = \sqrt{\frac{2}{\pi}}  \int\limits_{0}^{\infty}{x^2
e^ {-x^2/2}dx
}  =  \sqrt{\frac{2}{\pi}}  \int\limits_{0}^{\infty}{2t
e^ {-t} \frac{dt}{ \sqrt{2t}}} = \frac{2}{\sqrt{\pi}}  \int\limits_{0}^{\infty}{ t^ {1/2} e^{-t} dt} = \\ = \frac{2}{\sqrt{\pi}} \text{Г}(3/2) = \frac{2}{\sqrt{\pi}} \cdot \frac{1}{2} \cdot \text{Г}( \frac{1}{2} )   = 1  $\\
Для гаусівського розподілу: $N(a, \sigma^2)$, де $a = \mathbb{E} \xi \quad \sigma = \sqrt{\mathbb{D} \xi}$\\
$$
F_ \xi (x) =  \int\limits_{- \infty}^{ x}{ f_ \xi (t) dt} =  \int\limits_{- \infty}^{ x}{ \frac{1}{\sqrt{2\pi}  }e^{ - \frac{t^2}{2}} dt } =  \int\limits_{- \infty}^{0}{\frac{1}{\sqrt{2\pi}  }e^{ - \frac{t^2}{2}} dt } +  \int\limits_{0}^{ x}{\frac{1}{\sqrt{2\pi}  }e^{ - \frac{t^2}{2}} dt } =
 \frac{1}{2} + \Phi(x)
$$
$\Phi (x) = \frac{1}{\sqrt{2\pi}}  \int\limits_{0}^{ x}{ e^{ - \frac{t^2}{2} } dt} $  - функція Лапласа aka \textbf {CDF }(cumulative distr. function).\\
$\xi \sim N(0,1).$ Знайдемо $ \mathbb{P} \left\lbrace \xi \in [b,c] \right\rbrace,\mathbb{E} \xi^k, k \in \mathbb{N}$:
$$
\mathbb{P} \left\lbrace  \xi \in [b,c] \right\rbrace = F_ \xi (c) - F_ \xi( b) = \Phi (c)- \Phi (b)
$$

$$
k = 2n, n\in \mathbb{N} \qquad \mathbb{E} \xi^k =   \int\limits_{0}^{ +\infty}{ x^k e^ { -x^2/2 }dx} = \frac{2^{k/2}}{\sqrt{\pi}}  \int\limits_{0}^{ +\infty}{ t^ { \frac{k-1}{2}  } e^{-t} dt} = \frac{2^ \frac{k}{2} }{ \sqrt(\pi) } \text{Г} ( \frac{k+1}{2} ) =
$$
$$
= \frac{2^ {k/2}}{\sqrt{\pi}} \cdot \frac{k-1}{2} \cdot \frac{k-3}{2}  \cdot ... \cdot \frac{3}{2} \cdot \frac{1}{2} \cdot \text{Г} ( \frac{1}{2} )= \frac{2^{ \frac{k}{2} }}{\sqrt{\pi}}\cdot \frac{(k-1)!!}{ 2^{ \frac{k}{2} }} \cdot \sqrt{\pi} = (k-1)!!
$$
$$
\mathbb{E} \xi ^k =   \int\limits_{-\infty}^{ +\infty}{ x^k \frac{1}{ \sqrt{2 \pi}} e ^ { - \frac{x^2}{2} } dx} = \left\lbrace \begin{gathered}
 (k-1)!! ,\quad k = 2n, n \in \mathbb{N}\\
 0 ,\qquad k = 2n+ 1, n \in \mathbb{N}
\end{gathered} \right.
$$
Перейдемо до загального гаусівського розподілу.\\
\textbf{Загальний гаусівський розподіл.}
\begin{defo}
Візьмемо, що $\xi_0 \sim N(0, 1)$ - стандартна гаусівська величина. \\
$\xi$ називається гаусівською величиною: $ \xi \sim N(a, \sigma^2)$, якщо $ \xi = a + \sigma \xi_0$ - існує та справедливе перетворення стандартної гаусівської величини.
\end{defo}
Числові характеристики гаусівського розподілу:\\
$\mathbb{E} \xi = \mathbb{E}(a + \sigma \xi_0) = a$\\
$\mathbb{D} \xi = \mathbb{D} (a + \sigma \xi) = \sigma^2 \mathbb{D} \xi = \sigma^2$\\
$ F_ \xi (x) = \mathbb{P} \left\lbrace \xi <x \right\rbrace = \mathbb{P} \left\lbrace a + \sigma\xi  < x \right\rbrace = \mathbb{P} \left\lbrace \xi_0 < \frac{x-a}{ \sigma}  \right\rbrace = F_ { \xi_0}( \frac{x-a}{\sigma} 	) = \frac{1}{2} + \Phi ( \frac{x-a}{\sigma} ) $ \\
$\mathbb{P} \left\lbrace  \xi \in [b,c] \right\rbrace = F_ \xi (c) - F_ \xi (b) = \Phi( \frac{c-a}{\sigma} ) - \Phi( \frac{b-a}{\sigma} )$\\

Знаючи $ F_ \xi (x) $  знайдемо вираз для щільності розподілу:
$$
f_ \xi (x ) = F` \xi (x) =  \frac{1}{\sqrt{\pi}} e^{ - (\frac{x-a}{\sigma})^2/2 } \cdot \frac{1}{\sigma}  =  \frac{1}{\sigma \sqrt{2\pi}} e^ { - \frac{(x-a)^2}{2 \sigma^2} }
$$
$$
E_{N(a, \sigma^2)} =  \frac{ \mathbb{E} \left( \xi - \mathbb{E} \xi \right)^4 }{ \left( \mathbb{D} \xi \right)^2 }  -3
 =\left|
\begin{gathered}
 \xi = a \sigma \xi_0 \\
 \mathbb{E} \xi = a
\end{gathered}
 \right|
 = \frac{ \sigma ^4 \mathbb{R} \xi_0^4 }{ \sigma^4} - 3 = \mathbb{E} \xi_0^4 - 3 = 0
$$
\textbf{Правило } $ \mathbf{``3 \sigma``}$. \quad $ \xi \sim (a, \sigma^2)$ \quad Знайдемо:
$\mathbb{P} \left\lbrace  \left| \xi - a \right | < 3 \sigma  \right\rbrace
 = \\ = \mathbb{P} \left\lbrace  \xi \in (a - 3 \sigma; a + 3\sigma) \right\rbrace
= \Phi ( \frac{a + 3\sigma  -a  }{\sigma}  - \Phi( \frac{a - 3\sigma - a}{\sigma} )) = 2\Phi (3) \approx 0.9974
$\\
Тобто, у багатьох практичних випадках, гаусівська величина відповідає нерівності $\left| \xi - a \right | < 3 \sigma$ з великою вірогідністю.\\


\begin{boxteo}[Центральна гранична теорема]
Розглянемо $\xi_1, \xi_2, ... , \xi_n$ - незалежні, мають однаковий розподіл.
Якщо $\mathbb{E} \xi_i = 0 $ та $\mathbb{D} \xi_1 = 1$ :
$$
\frac{\xi_1 + \xi_2 + ... + \xi_n}{\sqrt{n}} \begin{gathered}
 d\\ \longrightarrow \\
 n \to \infty
\end{gathered}  N (0,1)
$$
\end{boxteo}
Гаусівська випадкова величина добре описує результат дії великої кількості випадкових факторів, дія кожного з яких окремо є досить малою.

\section{Випадкові вектори}
Розглядаємо: $$ \vec{ \xi} = \begin{bmatrix}
 \xi_1 \\
 \xi_2\\
 ...\\
 \xi_n
\end{bmatrix}$$

\begin{defo}
	Випадковий вектор - система випадкових величин $ \xi_1 ... \xi_n$, що задані на спільному ймовірністному просторі $ (\Omega, F, \mathbb{P})$.
\end{defo}

Функція розподілу: $ F_{\vec{\xi}} ( x_1, ..., x_n) = \mathbb{P} \left\lbrace \xi_1 < x_1, ..., \xi_n < x_n \right\rbrace$.
$$
\begin{gathered}
\overline{\xi} = \begin{bmatrix}
\xi_1 \\
\xi_2\\
\end{bmatrix}  \\  \\ F_{\overline{\xi}} (x,y) = \mathbb{P} \left\lbrace \xi_1 < x, \xi_2 < y \right\rbrace
\end{gathered} \qquad\qquad\qquad
\begin{gathered}
 \includegraphics[scale=0.21]{images/9.png}
\end{gathered}
$$
% graphicx
\subsection{Властивості функції  розподілу.}

1. $F_{\overline{\xi}} (x,y) \in [0,1] \quad \forall x,y \in \mathbb{R}$\\
2. $ F_{ \overline{\xi}} $ - неспадна для кожного аргументу. Тобто:
\begin{center}
	$F_{\overline{\xi}} (x_1, y) \leq F_{\overline{\xi}} (x_2, y)$ при $ x_1 \leq x_2$;\\ $ F_{\overline{\xi}} (x, y_1) \leq F_{\overline{\xi}} (x, y_2)$ при $ y_1 \leq y_2$.
\end{center}
$$
\mathbb{P} \left\lbrace \xi_1 < x_1, \xi_2 < y \right\rbrace \leq \mathbb{P} \left\lbrace \xi_1 < x_2, \xi_2 < y \right\rbrace
$$
3, $ F_{\overline{\xi}} $ - неперервна зліва за кожним аргументом. \\
4a.$$
\begin{gathered}
\text{В одновимірному:}\\
  \lim\limits_{x\to  -\infty}{F_{\xi}} = 0 \\
	\lim\limits_{x\to  +\infty}{F_{\xi}} = 1
\end{gathered} \Rightarrow
\begin{gathered}
\text{В багатовимірному:}\\
  \lim\limits_{x\to  -\infty}{ F_ {\overline{\xi}} (x,y)} =  0\\
	  \lim\limits_{y\to  -\infty}{ F_ {\overline{\xi}} (x,y)} =  0
\end{gathered} \qquad \qquad
\begin{gathered}
 \includegraphics[scale=0.24]{images/10.png}
\end{gathered}
$$
\begin{proof}
$  \lim\limits_{n\to  \infty}{ F_{ \overline{\xi}} (x_n, y_n)} = 0 $, якщо $ x_n \to \infty$ або $y_n \to \infty$.
$$ \lim\limits_{n\to  \infty}{ F_{\overline{\xi}} (x_n, y_n) } =   \lim\limits_{n \to  \infty}{ \mathbb{P} \left\lbrace \xi_1 < x_n , \xi_2 < y_n \right\rbrace} = \mathbb{P} \left\lbrace  \bigcap\limits_{n\geq 1} (\xi_1 < x_n, \xi_2 < y_n) \right\rbrace = \mathbb{P} \left\lbrace \emptyset \right\rbrace = 0$$

\end{proof}
4b.
$$
\qquad\qquad
\begin{gathered}
\lim\limits_{\tiny \begin{gathered}
 x\to  +\infty\\ y\to  +\infty
\end{gathered}}{ F_ {\overline{\xi}} (x,y)} =  1
\end{gathered} \qquad \qquad \qquad
\begin{gathered}
 \includegraphics[scale=0.25]{images/11.png}
\end{gathered}
$$
\begin{proof}
$$
 \lim\limits_{n\to  \infty}{ F_{\overline{\xi}} (x_n, y_n) } =   \lim\limits_{n \to  \infty}{ \mathbb{P} \left\lbrace \xi_1 < x_n , \xi_2 < y_n \right\rbrace} = \mathbb{P} \left\lbrace  \bigcup\limits_{n\geq 1} (\xi_1 < x_n, \xi_2 < y_n) \right\rbrace = \mathbb{P} \left\lbrace \Omega \right\rbrace = 1
 $$
\end{proof}
4c.
$$
\lim\limits_{
\begin{gathered}
x\to  +\infty \\ y\in \mathbb{R}
\end{gathered}
}{ F_ {\overline{\xi}} (x,y)} =  \mathbb{P} \left\lbrace \xi_2 < y \right\rbrace = F_{\xi_2} (y)
$$

$$
\lim\limits_{
\begin{gathered}
y\to  +\infty \\ x\in \mathbb{R}
\end{gathered}
}{ F_ {\overline{\xi}} (x,y)} =  \mathbb{P} \left\lbrace \xi_1 < x \right\rbrace = F_{\xi_1} (x)
$$

\begin{defo}
$
\overline{\xi} = \begin{bmatrix}
 \xi_1 \\
 \xi_2
\end{bmatrix} \quad F_{\overline{\xi}} (x,y) - \textbf{сумісна функція розподілу.}
$\\
$\xi_1$ - випадкова величина. $ F_{\xi_1} $ -\textbf{маргінальна функція розподілу}  $ \xi_1$. \\ Щоб отримати маргінальну функцію розподілу, потрібно відправили ''зайві'' аргументи до $ + \infty$.\\
\end{defo}


5.  В одновимірному випадку:
$$
\qquad\quad
\begin{gathered}
\mathbb{P} \left\lbrace  \xi \in [a,b) \right\rbrace  = F_ \xi (b) - F_ \xi (a)
\end{gathered}\qquad \qquad
\begin{gathered}
 \includegraphics[scale=0.26]{images/6.png}\qquad
\end{gathered}$$
У багатовимірному випадку нас цікавить вірогідність $\mathbb{P} \left\lbrace \xi_1 \in [ a_1, b_1), \xi_2 \in [ a_2, b_2) \right\rbrace$ (користуємося правилом знаходження приросту функції 2-ох змінних):
$$\mathbb{P} \left\lbrace \xi_1 \in [ a_1, b_1), \xi_2 \in [ a_2, b_2) \right\rbrace = \mathbb{P} \left\lbrace \overline{\xi}  \in \text{ П}\right\rbrace =$$
$$
= F_{\overline{\xi}} (b_1, b_2) - F_{\overline{\xi}} (b_1, a_2)  - F_{\overline{\xi}} (b_1, a_2) + F_{\overline{\xi}} (a_1, a_2)
$$
\begin{example}
	$$
	\qquad\qquad
	F (x,y) = \left\lbrace
	\begin{gathered}
	 0, x+y \leq 0\\
	 1, x+y > 0
	\end{gathered} \right.
	\qquad \qquad
	\begin{gathered}
	 \includegraphics[scale=0.20]{images/12.png}\qquad
	 	\end{gathered}
	$$
	Задана функція не є функцією розподілу. Розглянемо прямокутник П.
\end{example}
\subsection{Дискретні та неперервні випадкові вектори.}
\subsubsection{Дискретні випадкові вектори.}
\begin{defo}
	$ \overline{\xi} = \begin{bmatrix}
	  \xi_1 \\
	...\\
	\xi_n\\
	\end{bmatrix}$ називають дискретним(неперервним), якщо усі його координати - дискретні(неперервні) випадкові величини.
\end{defo}
$
\overline{\xi} = \begin{bmatrix}
	\xi_1 \\
...\\
\xi_n\\
\end{bmatrix}
$ - дискретний вектор.
$p_{ij} = \mathbb{P} \left\lbrace \xi_1 = x_i , \xi_2= y_j \right\rbrace
\qquad $
$$
\begin{gathered}
 \includegraphics[scale=0.25]{images/13.png}\quad
	\end{gathered}
	\begin{gathered}
	 \includegraphics[scale=0.25]{images/14.png}\qquad
		\end{gathered}\qquad
$$
\subsubsection{Неперервні випадкові вектори.}
	$\xi$ - неперервна $ \Leftrightarrow$ $F_{\xi}$ - неп. функція $\Leftrightarrow$ $\mathbb{P} \left\lbrace \xi=x \right\rbrace  = 0 \quad \forall x \in \mathbb{R}$.
\begin{defo}
$ \overline{\xi}$ - неперервний вектор, якщо $ \mathbb{P} \left\lbrace  \xi = \overline{x} \right\rbrace = 0 \quad \forall  \overline{x}$
\end{defo}
\begin{defo}
	$ \overline{\xi}$ - абсолютно неперервний вектор, якщо $$ \exists f: F_{\xi} =  \int\limits_{-\infty}^{ x}{...  \int\limits_{-\infty}^{ x_n}{ f_{ \overline{\xi}}(x_1, ..., x_n)}dx_1...dx_n} = \mathbb{P} \left\lbrace \xi_1 < x_1 , ... , \xi_n < x_n\right\rbrace$$
	$$
	F_{\xi_1, \xi_2} (x,y) =  \int\limits_{- \infty}^{ x}{  \int\limits_{- \infty }^{ y}{ f_{\xi_1, \xi_2 }(s, t) dsdt}} \qquad \qquad
\begin{gathered}
  \includegraphics[scale=0.2]{images/15.png}
\end{gathered}
	$$

	\subsubsection{Властивості щільності розподілу:}
	1. $f_{\xi_1, \xi_2} (x,y) = \frac{ \partial ^2 F_{\xi_1, \xi_2}(x,y)}{ \partial x \partial y} $ - в точках, де похідна існує.\\
	2. $ \int\limits_{-\infty}^{ +\infty}{   \int\limits_{-\infty}^{ +\infty}{
	f_{\xi_1, \xi_2}(x,y) dxdy = 1
	}}$
	\begin{proof}
$$
 \lim\limits_{			\tiny
 \begin{gathered}
	    x\to  \infty\\
		 y\to \infty
	  \end{gathered}}{  \int\limits_{-\infty}^{ x}{   \int\limits_{-\infty}^{ y}{
		f_{\xi_1, \xi_2}(x,y) dxdy = 1} }} $$
		$$
		F_{\xi_1, \xi_2} (x,y) \begin{gathered}
		   \\
			\longrightarrow \\
			x,y\to \infty
		\end{gathered} 1
		$$
	\end{proof}
\end{defo}
3. $ \mathbb{P} \left\lbrace \overline{\xi} \in B \right\rbrace =  \iint\limits_{B}^{}{
f_{\overline{\xi}}dxdy}$, якщо $B$ - квадрована множина.
\begin{proof} Доведемо спочатку для прямокутників $ B = [a_1, b_1] \times [a_2, b_2]$.
  $$\mathbb{P} \left\lbrace \overline{\xi} \in B \right\rbrace = F(b_1, b_2) - F(b_1, a_2) - F(a_1, b_2) + F(a_2, b_2) =  \iint\limits_{B}^{}{ f_{ \overline{\xi}} dxdy}$$
\end{proof}
\subsection{Рівномірний розподіл на площині.}
$$
\overline{\xi} \sim U(A) \Leftrightarrow  f_{ \overline{\xi}} = \left\lbrace
\begin{gathered}
 c, (x,y) \in A\\
 0, (x,y)\notin A
\end{gathered} \right. \qquad \qquad
\begin{gathered}
 \includegraphics[scale=0.3]{images/16.png}
\end{gathered}
$$
$$
1 = \iint\limits_{\mathbb{R}^2	}^{}{
f_{\xi} (x,y )dxdy = c \cdot S(A) \Rightarrow c = \frac{1}{S(A)}
}$$
\begin{center}
\fbox{
 $
 f_{\overline{ \xi}} = \left\lbrace
 \begin{gathered}
	 \frac{1}{S(A)}, (x,y) \in A\\
	0, (x,y)\notin A
 \end{gathered} \right.
	$
}
\end{center}
$$
\mathbb{P} \left\lbrace \overline{\xi} \in B \right\rbrace = \iint\limits_{B}^{}{ f_{ \overline{ \xi}(x,y)}dxdy} = \iint\limits_{B}^{}{ \frac{1}{S(A)} } dxdy = \frac{S(B)}{S(A)}
$$
\subsection{Маргінальна щільність}
$$
\overline{\xi}  = \begin{bmatrix}
  \xi_1 \\ \xi_2
\end{bmatrix}
\qquad f_{ \overline{ \xi}} \text{ - щільність} \qquad f_{ \overline{ \xi}} (x) =  \int\limits_{- \infty}^{ +\infty}{ f_{ \overline{\xi}}(x,y)dy}
$$

\begin{proof}
$$ \int\limits_{C}^{}{f_{ \xi_1} dx}=
\mathbb{P} \left\lbrace \xi_1 \in C \right\rbrace = \mathbb{P} \left\lbrace ( \xi_1 , \xi_2 ) \in C \times \mathbb{R} \right\rbrace =  \int\limits_{C}^{}{   \int\limits_{-\infty}^{ +\infty}{ f_{ \overline{\xi}}(x,y) dxdy}}$$
\end{proof}

\begin{example}
$$	f_{ \overline{xi}} (x,y) = \left\lbrace  \begin{gathered}
3 , (x,y) \in D\\
0, (x,y) \notin D
	\end{gathered} \right. $$
	За умовою нормування:
	$$
	\begin{gathered}
		S(D) =  \int\limits_{0}^{1}{ ( \sqrt{x} - x^2 ) dx}= ( \frac{2}{3} x^{ \frac{3}{2} } - \frac{x^3}{3}  ) \Big|_0^1 = \frac{1}{3}\\
	f_{\xi} (x) =  \int\limits_{-\infty}^{ +\infty}{ f_{ \overline{\xi}} dy}
	\end{gathered}\qquad
	\begin{gathered}
	 \includegraphics[scale=0.25]{images/17.png}
	\end{gathered}
	$$
	1. $ \begin{gathered}
	 x \in (-\infty; 0)\\
	 x \in (1; + \infty)
	\end{gathered} \quad f_{\xi_1} (x) =   \int\limits_{-\infty}^{ +\infty}{ 0dy} = 0$\\
	2. $x \in [0,1] \quad f_{\xi_1} (x) =   \int\limits_{x^2}^{ \sqrt{x}}{ 3dy} = 3 ( \sqrt{x} - x^2)$
	$$
	f_{\xi_1} (x) = \left\lbrace
\begin{gathered}
 0, x \in (-\infty, 0) \cup (1, +\infty ) \\
  3 ( \sqrt{x} - x^2), x \in [0,1]
\end{gathered}
	 \right.
	$$
	Перевірка: $  \int\limits_{-\infty}^{ +\infty}{ f_{\xi_1 } dx} = 3
	 \int\limits_{0}^{1}{ (\sqrt{x} - x^2)dx} = 1$ .\\
	 Аналогічно для $ f_{\xi_2} (y)$.
\end{example}

\subsection{Числові характеристики випадкових векторів.}
$$
\mathbb{E} \xi_1 =  \int\limits_{-\infty}^{ +\infty}{ x f_{\xi} (x) dx} =  \int\limits_{-\infty}^{ +\infty}{
\left( x
 \int\limits_{-\infty}^{ +\infty}{ f_{ \overline{ \xi}}  (x,y)dy}
 \right)dx =  \int\limits_{-\infty}^{ +\infty}{  \int\limits_{
 -\infty
 }^{ +\infty}{
 x f_{ \overline{ \xi}} (x,y) dxdy
 }}
}
$$
$$
\mathbb{E} \xi_2 =   \int\limits_{-\infty}^{ +\infty}{  \int\limits_{
-\infty
}^{ +\infty}{
y^2 f_{ \overline{ \xi}} (x,y) dxdy
}} =  \int\limits_{-\infty}^{ +\infty}{ dy \int\limits_{
-\infty
}^{ +\infty}{
y^2 f_{ \overline{ \xi}} (x,y) dx
}}  =  \int\limits_{-\infty}^{ +\infty}{ dx \int\limits_{
-\infty
}^{ +\infty}{
y^2 f_{ \overline{ \xi}} (x,y) dy
}}
$$
Таким чином, за подвійним інтегралом рахувати числові характеристики зручніше, адже ми можемо вибрати найпростіший вигляд.

\begin{example}
	Точка розподілена в одиничному крузі, для якого $f_ { \overline{\xi}} (x,y)$ пропорційеа відстані до границі круга. Знайти $ \mathbb{D} \xi_1, \mathbb{D} \xi_2$.
	$$
	f_{ \overline{ \xi}}(x,y) =
	\left\lbrace
	\begin{gathered}
	 (1 - \sqrt{x^2 + y^2})k, (x,y) \in \bigcirc \\
	  0 , (x,y) \notin \bigcirc
	\end{gathered} \right.\qquad
	\begin{gathered}
	 \includegraphics[scale=0.3]{images/18.png}
	\end{gathered}
	$$
	$$
	 1 = k \iint\limits_{\bigcirc}  ( 1 - \sqrt{x^2 + y^2})dxdy = k  \int\limits_{0}^{2\pi}{ d \varphi  \int\limits_{0}^{ 1}{ (1 - \rho)\rho d \rho}} =  \frac{\pi k}{3}  \Longrightarrow k = \frac{3}{\pi}
	$$
	$$
	\mathbb{D}\xi_1 = \mathbb{E} (\xi^2) - (\mathbb{E} \xi)^2 =  \iint\limits_{\bigcirc}^{ } x^2 \frac{3
	}{\pi} (1 - \sqrt{x^2 + y^2})dxdy
	$$
\end{example}

\subsection{Коваріація та її властивості.}
$$
\begin{gathered}
\overline{\xi} = \begin{bmatrix}
 \xi_1 \\
 \xi_2
\end{bmatrix} \qquad \mathbb{E}\xi_1, \mathbb{E}\xi_2. \mathbb{D}\xi_1, \mathbb{D}\xi_2
\end{gathered}
\qquad
\begin{gathered}
 \includegraphics[scale=0.25]{images/19.png}\\
 \text{Маэстро, что с Вами?}
\end{gathered}
$$

\begin{defo}
	Коваріація (кореляційний момент) - $ cov( \xi_1, \xi_2)$.
	$$ cov(\xi_1, \xi_2 ) = \mathbb{E}(\xi_1 - \mathbb{E} \xi_2)(\xi_2 - \mathbb{E}\xi_2) = \mathbb{E}(\xi_1 \xi_2) - \mathbb{E} \xi_1 \xi_2$$
\end{defo}
\textbf{Коваріяція дискретного випадкового вектора.}
$$
cov( \xi_1, \xi_2) = \sum\limits_{i = 1}^{ m}{
 \sum\limits_{j = 1}^{ n}{ x_i \cdot y_j \cdot p_{ij}}
} - \left\lbrace  \sum\limits_{i = 1}^{ m}{
 \sum\limits_{j = 1}^{ n}{ x_i p_{ij}}
} \right\rbrace
\cdot
\left\lbrace \sum\limits_{i = 1}^{ m}{
 \sum\limits_{j = 1}^{ n}{ y_j p_{ij}}
} \right\rbrace
$$
\textbf{Коваріяція неперервного випадкового вектора.}
$$
cov( \xi_1 , \xi_2)=  \iint\limits_{\mathbb{R}^2	}^{ }{ xy  \cdot f_{ \overline{\xi}}  dxdy} - \iint\limits_{\mathbb{R}^2	}^{ }{ x  \cdot f_{ \overline{\xi}}  dxdy} \cdot \iint\limits_{\mathbb{R}^2	}^{ }{ y  \cdot f_{ \overline{\xi}}  dxdy}
$$
\begin{center}
\textbf{Властивості коваріації.}
\end{center}

1. $cov(\xi, \xi)  = \mathbb{D} \xi$.\\
2. Якщо $\xi_1, \xi_2 $ - незалежні , то $ cov( \xi_1 , \xi_2) = \mathbb{E}(\xi_1 - \mathbb{E}\xi_1) (\xi_2 - \mathbb{E}\xi_2) = 0$
\begin{defo}
	$\xi_1$ та $\xi_2$ наз. некорельованими,  якщо $ cov(\xi_1, \xi_2) =0$.
	\end{defo}

3. $ cov(\xi_1, \xi_2 ) = cov(\xi_2, \xi_1)$ (симетричність).\\
4. $cov(\xi, c) = 0$\\
5. $cov( \alpha \xi_1' + \beta \xi_1'', \xi_2) = \mathbb{E}( \alpha \xi_1' + \beta \xi_1 ''- \mathbb{E}(\alpha \xi_1' + \beta \xi_1 ''))
(\xi_2 - \mathbb{E}\xi_2) = \alpha cov(\xi_1', \xi_2) + \beta cov(\xi_1'', \xi_2)$\\
\textbf{Отримали:} Коваріація є білінійним симетричним функціоналом.\\
6. Якщо $\xi_1, \xi_2$ - незалежні, то  $\mathbb{D}(\xi_1 \pm \xi_2) = \mathbb{D}\xi_1 \pm \mathbb{D}\xi_2$.\\
Якщо існує залежність: $ \mathbb{D}( \xi_1 \pm \xi_2) = \mathbb{E}( (\xi_1 \pm \xi_2) - \mathbb{E}(\xi_1 \pm \xi_2))^2 = \mathbb{E}( (\xi_1 - \mathbb{E}\xi_1) \pm (\xi_2 - \mathbb{E}\xi_2)) = \mathbb{E}((\xi_1 - \mathbb{E} \xi_1)^2 + (\xi_2 - \mathbb{E} \xi_2 )^2 \pm 2 (\xi_1 - \mathbb{E} \xi_1)(\xi_2 - \mathbb{E}\xi_2)) =
\mathbb{D}\xi_1 + \mathbb{D}\xi_2  \pm 2cov(\xi_1, \xi_2)$
\\
7. $cov(\mathbb{I}_A, \mathbb{I}_B) = \mathbb{E} (\mathbb{I}_A \mathbb{I}_B) - \left( \mathbb{E} \mathbb{I}_A \right) (\mathbb{E} \mathbb{I}_B) = \mathbb{P} \left\lbrace A \cap B \right\rbrace - \mathbb{P} \left\lbrace A \right\rbrace \mathbb{P} \left\lbrace B  \right\rbrace$\\
8. Нерівність Коші-Буняковського.
$$
\left| cov(\xi_1, \xi_2) \right| \leq \sqrt{	\mathbb{D}\xi_1\cdot \mathbb{D}\xi_2}
$$
\subsection{Коваріаційна матриця вектора та її властивості}

$\xi: $ дисперсія $\mathbb{D} \xi = \mathbb{R}\mathbb{E} (\xi - \mathbb{E}\xi)^2$\\
$ \overline{\xi}: $ коваріаційна матриця $C_{ \overline{\xi}} = \mathbb{E}( \overline{\xi} - \mathbb{E}\overline{\xi})(\overline{\xi} - \mathbb{E} \overline{\xi})^T$
$$
C_{ \overline{\xi}} = \begin{bmatrix}
 \mathbb{D} \xi_1 & cov(\xi_1, \xi_2) & \cdots & cov(\xi_1, \xi_n) \\
 cov(\xi_2, \xi_1) & \mathbb{D}\xi_2 & \cdots & cov(\xi_2, \xi_n)\\
 \vdots & \vdots & \ddots & \vdots\\
 cov(\xi_n, \xi_1)& cov(\xi_n, \xi_1) & \cdots & \mathbb{D}\xi_n
\end{bmatrix}
$$
1. $C_{\overline{\xi}}$ - симетрична матриця.\\
2. $A$ - квадратна матриця $n \times n$, симетрична.\\$A$ - невід'ємно визначена $\Leftrightarrow (A \overline{x}, \overline{x}) \geq \forall  \overline{x} \in \mathbb{R}^n$
$C_{\overline{\xi}}$ - невід'ємно визначена. $ (C_{\overline{\xi}}, \overline{\xi}) = (\mathbb{E}(\overline{\xi} - \mathbb{E}\overline{\xi})(\mathbb{E}(\overline{\xi} - \mathbb{E}\overline{\xi})^T \overline{x}, \overline{x} ) = \mathbb{E} \left| \left| (\overline{\xi} - \mathbb{E} \overline{\xi})^T \overline{x}\right| \right|^2 \geq 0 $.\\
Застосування невід'ємної визначеності.
$$
\overline{\xi} = \begin{bmatrix}
 \xi_1\\
 \xi_2
\end{bmatrix} \qquad \qquad C_{ \overline{\xi}} = \begin{bmatrix}
 \mathbb{D}\xi_1 & cov(\xi_1, \xi_2) \\
 cov(\xi_1, \xi_2) & \mathbb{D} \xi_2
\end{bmatrix} - \text{ невід'ємно визначена}
$$
Застосуємо критерій Сільвестра:
$$
\mathbb{D}\xi_1 \cdot \mathbb{D}\xi_2 - cov^2(\xi_1, \xi(2)) \geq 0 \Longrightarrow  \left| cov(\xi_1, \xi_2) \right| \leq \sqrt{ \mathbb{D} \xi_1 \cdot \mathbb{D}\xi_2}
$$


\textbf{Що означає виродженість коваріаційної матриці?} $ \Leftrightarrow \det C_{ \overline{\xi}} = 0$:
$$
\det C_{ \overline{\xi}} = 0 \Leftrightarrow \exists \overline{x} \in \mathbb{R}^n: \left( C_{\overline{\xi}}, \overline{\xi} \right) = 0
\Leftrightarrow C_{ \overline{\xi}} - \text{не додатня, невід'ємно визначена}
$$
\begin{proof}
$$
\det C_{ \overline{\xi}} = 0 \Rightarrow Ker C_{ \overline{\xi}} \neq \left\lbrace \vec{0} \right\rbrace \Rightarrow \exists \overline{x} \neq 0: C_{ \overline{\xi}} \overline{x} = 0\Rightarrow \left( C_{\overline{\xi}} \overline{x}, \overline{x} \right) = 0
$$
$$
\exists \overline{x} \neq 0: \left( C_{\overline{\xi}} \overline{x}, \overline{x} \right) = 0 \Rightarrow \exists \overline{y} \neq 0: \lambda_1 y_1^2 + ... + \lambda_n y_n^2
$$
$$
\left( C_{\overline{\xi}} \overline{x}, \overline{x} \right) = 0 = \left( \Omega \overline{y}, \overline{y} \right) = \lambda_1 y_1^2 + ... + \lambda_n y_n^2\Leftrightarrow \exists \lambda_i = 0 \Rightarrow \det C_{ \overline{\xi}} = 0
$$
\end{proof}
\begin{boxteo}
	$C_{\overline{\xi}}$ є виродженою т.т.т.к між $ \xi_1, ... , \xi_n$ є афінна залежність. Тобто:
	$$\lambda_1 \xi_1 + \lambda_2 \xi_2 +...+ \lambda_n \xi_n = c$$
\end{boxteo}
\begin{proof}
 $$
\exists \overline{x} \neq 0: \left( C_{\overline{\xi}} \overline{x}, \overline{x} \right) = 0 \Leftrightarrow \exists \overline{ \xi} \neq \vec{0} \in \mathbb{R}^n: \left( \mathbb{E}( \overline{\xi} - \mathbb{E}\overline{\xi})(\overline{\xi} - \mathbb{E} \overline{\xi})^T \cdot \overline{x}, \overline{x} \right) = 0 \Leftrightarrow
 $$
 $$
\Leftrightarrow  \left( \mathbb{E}( \overline{\xi} - \mathbb{E}\overline{\xi}) \cdot \overline{x},(\overline{\xi} - \mathbb{E} \overline{\xi})\cdot \overline{x} \right) = 0 \Leftrightarrow
 $$
 $$
 \Leftrightarrow \mathbb{E} \left| \left|  \left( \overline{\xi} - \mathbb{E}\overline{\xi} \right)^T \overline{x}  \right|  \right|^2 = 0 \Leftrightarrow \left| \left|  \left( \overline{\xi} - \mathbb{E}\overline{\xi} \right)^T \overline{x}  \right|  \right|^2 = 0 \quad \text{м.н.}
\Leftrightarrow $$
$$
\Leftrightarrow\left( \overline{\xi} - \mathbb{E}\overline{\xi} \right)^T \overline{x} = 0 \Leftrightarrow \left( \xi_1 - \mathbb{E} \xi_1 \right) x_1 + ... + (\xi_n - \mathbb{E}\xi_n) x_n = 0 \Leftrightarrow
$$
$ \exists x_1, ..., x_n$ не всі з яких дорівнюють нулю:
$$
\Leftrightarrow
x_1 \xi_1 + x_2 \xi_2 + ... + x_n \xi_n  = x_1 \mathbb{E}\xi_1 + ... + x_n \mathbb{E}\xi_n = c
\Leftrightarrow
$$
Візьмемо $ x_i = \lambda_i$: $ \lambda_1 \xi_1 + ... + \lambda_n \xi_n = c \Leftrightarrow$ афінна залежність.
\end{proof}

Розглянемо $
cov( \xi, \eta) = \mathbb{E}(\xi - \mathbb{E}\xi) (\eta - \mathbb{E}\eta)
$.\\
Застосуємо нерівність Коші-Буняковського:
$ \left| cov(\xi, \eta) \right| \leq \sqrt{ \mathbb{D}\xi \cdot\mathbb{D}\eta} $
\begin{center}
\fbox{
	$r_{\xi, \eta} =\displaystyle\frac{cov(\xi, \eta)}{\sqrt{\mathbb{D}\xi\cdot \mathbb{D}\eta}} $ - коефіцієнт кореляції між $\xi$ та $\eta$.
}\\
\end{center}
$$-1 \leq r_{\xi, \eta} \leq 1 $$

Коефіцієнт показує ''силу'' лінійної залежності між $\xi$ та $\eta$.
$$
r_{\xi, \eta } = 0 \Leftrightarrow cov(\xi, \eta) \Leftrightarrow \xi \text{ та } \eta  \text{- некорельовані.}
$$
$$
r_{\xi, \eta}= \pm 1  \Leftrightarrow \det \begin{bmatrix}
 \mathbb{D} \xi & cov(\xi, \eta)\\
 cov(\xi, \eta) & \mathbb{D} \eta
\end{bmatrix} = 0 \Leftrightarrow \det C_{\xi, \eta} = 0\Leftrightarrow
$$
$$
\Leftrightarrow \mathbb{D}\xi \cdot \mathbb{D}\eta - cov(\xi, \eta) = 0 \Leftrightarrow \left| r_{\xi, \eta}  \right| = 1
$$
\begin{boxteo}
	$r_{\xi, \eta}= \pm 1$ т.т.т.к. $\eta = k \xi + b$, де $k, b \in R$\\
	При цьому $\begin{gathered}
	 r_{\xi, \eta} +1 \Rightarrow k>0\\
	 r_{\xi, \eta} -1 \Rightarrow k<0
	\end{gathered}$
\end{boxteo}
\begin{proof}
 $$r_{\xi, \eta} = \frac{cov(\xi, k \xi +b)}{\mathbb{D}\xi\cdot \mathbb{D}(k \xi +b)} = \frac{k \mathbb{D}\xi}{ \sqrt{k^2 \cdot \mathbb{D}^2 \xi} }  = \frac{k}{ \left| k \right| } = \left\lbrace \begin{gathered}
  1, k>0\\
	-1, k<0\\
 \end{gathered} \right. $$
\end{proof}
\subsection{Незалежність випадкових величин}
\begin{defo}
	Випадкові величини $\xi, \eta$ називають незалежними, якщо події \\$ \left\lbrace \xi\in [a,b] \right\rbrace, \left\lbrace \eta\in [a,b] \right\rbrace $ є незалежними $ \forall a \leq  b , c\leq d$\\
	Зокрема, якщо $\xi, \eta$ - дискретні:
	$$
	\begin{gathered}
	 \xi \in \left\lbrace x_1, ..., x_n \right\rbrace\\
	 \eta \in \left\lbrace y_1, ..., y_n  \right\rbrace
	\end{gathered} \qquad \left\lbrace \xi = x_i \right\rbrace \independent \left\lbrace \eta = y_j \right\rbrace \quad \begin{gathered}
	 \forall i = \overline{1, m}\\
	 \forall j = \overline{1, n}
	\end{gathered}
	$$

\end{defo}

\pagebreak

\begin{boxteo}
	$\xi, \eta$ - незалежні $\Leftrightarrow F_{\xi, \eta} = F_{\xi}(x) \cdot F_{\eta}(y)$
\end{boxteo}
\begin{proof}
Нехай $\xi, \eta$ - незалежні $\Leftrightarrow \forall a\leq b, c\leq d: \mathbb{P} \left\lbrace  \xi \in [a,b], \eta \in [c,d] \right\rbrace = \mathbb{P} \left\lbrace \xi\in[a,b] \right\rbrace \cdot \mathbb{P} \left\lbrace \eta\in[c,d] \right\rbrace \Rightarrow \mathbb{P} \left\lbrace  \xi\in [a,b), \eta \in [c,d) \right\rbrace = \mathbb{P} \left\lbrace \xi \in [a,b) \right\rbrace \cdot \mathbb{P} \left\lbrace \eta \in [c,d) \right\rbrace  $\\
$
\mathbb{P} \left\lbrace \xi<b, \eta <d  \right\rbrace = \mathbb{P} \left\lbrace \xi < b \right\rbrace \cdot \mathbb{P} \left\lbrace \eta <d \right\rbrace \Leftrightarrow F_{\xi, \eta} (b,d) = F_{\xi}(b) \cdot F(\eta)(d)
$\\
Нехай навпаки: $ F_{\xi, \eta } (x,y)  = F_{\xi}(x) - F_{\eta}(y)\quad \forall x,y \in \mathbb{R}$
$$
\mathbb{P} \left\lbrace  \xi\in [a,b), \eta \in [c,d)\right\rbrace = F_{\xi, \eta} (d,b) - F_{\xi,\eta}(b,c) - F_{\xi, \eta }(a,d) + F_{\xi, \eta}(a,c) =
$$
$$
= F_{\xi}(b) F_{\eta} (d) - F_{\xi}(b) F_{\eta}(c) - F_{\xi}(a) F_{\eta}(d) + F_{\xi}(a) F_{\eta}(c)  =$$
$$= \left( F_{\xi}(b) - F_{\xi}(a) \right) \left( F_{\eta}(d) - F_\eta (c) \right) = \mathbb{P} \left\lbrace \xi\in [a,b) \right\rbrace \cdot \mathbb{P} \left\lbrace \eta \in [c,b) \right\rbrace
$$

\end{proof}
\begin{boxteo}Для абсолюно неперервного векторa $\begin{bmatrix}
	 \xi& \eta
	\end{bmatrix}^T$\\
$$\xi \independent \eta \Leftrightarrow f_{\xi, \eta} (x,y)= f_{\xi}(x) f_{\eta}(y) \quad \forall x,y \in \mathbb{R}$$
\end{boxteo}

\begin{proof}
$$
\begin{gathered}
1. \quad  F_{\xi, \eta} (x,y) = F_{\xi}(x) \cdot F_{\eta}(y) \Longrightarrow f_{\xi, \eta} (x,y) = f_{\xi}(x) \cdot f_{\eta}(y)\\
	2. \quad f_{\xi, \eta} (x,y) = f_{\xi}(x) \cdot f_{\eta}(y)\Longrightarrow F_{\xi, \eta} (x,y) = F_{\xi}(x) \cdot F_{\eta}(y)
\end{gathered}\quad \forall x,y \in \mathbb{R}
$$
2.
$$
F_{\xi, \eta} (x,y ) =  \int\limits_{-\infty}^{ x}{  \int\limits_{- \i}^{y}{ f_{\xi, \eta} (s,t) dsdt}} =  \int\limits_{- \i}^{ x}{  \int\limits_{-\i}^{y}{ f_{\xi} (s) \cdot f_{\eta} (t)dsdt}} =
$$
$$
= \int\limits_{-\i}^{ x}{ f_ \xi (s) ds}\cdot  \int\limits_{-\i}^{ +\infty}{f_\eta(t) dt} = F_{\xi}(s)\cdot F_{\eta}(t)
$$
1.
$$
f_{\xi, \eta} (x,y) = \frac{\partial^2 f}{\partial x \partial y} (x,y) = \frac{\d ^2}{\d x \d y} (F_{\xi}(x) \cdot F_{ \eta} (y) ) = \frac{\d}{\d x} (F_{\xi} (x) \cdot f_{\eta} (y) ) = f_{\xi}(x) \cdot f_{\eta}(y)
$$
\end{proof}

\subsection{Умовні розподіли та умовні математичні сподівання}

\subsubsection{Дискретний вектор}
$$
\begin{gathered}
\includegraphics[scale=0.16]{images/14.png}
\end{gathered}\qquad\quad
\begin{gathered}
\text{Розподіли } \xi_2 \text{ за } \xi_1\\
\mathbb{P} \left\lbrace \xi_2 = y_j | \xi_2 = x_i \right\rbrace = \frac{\mathbb{P} \left\lbrace \xi_1 = x_i, \xi_2 = y_j \right\rbrace}{ \mathbb{P} \left\lbrace \xi_1 = x_i \right\rbrace} = \\
= \frac{p_{ij}}{  \sum\limits_{j = 1}^{n}{ p_ij}}
\end{gathered}
$$

\begin{center}
\includegraphics[scale=0.28]{images/20.png} \end{center}
$$
\mathbb{E}(\xi_2 | \xi_1 = x_i)  =  \sum\limits_{j = 1}^{ n}{ y_j \mathbb{P} \left\lbrace \xi_2 = y_j | \xi_1 = x_i \right\rbrace } =  \sum\limits_{j = 1}^{n}{ y_{j} \cdot \frac{p_{ij} }{  \sum\limits_{k = 1}^{ n}{p_{ik}}}}= \frac{  \sum\limits_{j = 1}^{ n}{ y_j p_{ij}}}{  \sum\limits_{j = 1}^{n}{ p_{ij}}}
$$
$$
\begin{matrix}
	\xi_1 & x_1 & \cdots  &x_i & \cdots & x_m \\
	\mathbb{E}_{\xi_2 | \xi_1=x_k}  &  \frac{  \sum\limits_{j = 1}^{ n}{ y_j p_{1j}}}{  \sum\limits_{j = 1}^{n}{ p_{1j}}}  & \cdots  &\frac{  \sum\limits_{j = 1}^{ n}{ y_j p_{ij}}}{  \sum\limits_{j = 1}^{n}{ p_{ij}}}  & \cdots & \frac{  \sum\limits_{j = 1}^{ n}{ y_j p_{mj}}}{  \sum\limits_{j = 1}^{n}{ p_{mj}}}  \\
	\mathbb{P} \left\lbrace \xi_1 = x_k \right\rbrace &   \sum\limits_{j = 1}^{n}{ p_{ij}} & \cdots  &\sum\limits_{j = 1}^{n}{ p_{ij}} & \cdots & \sum\limits_{j = 1}^{n}{ p_{mj}} \\
\end{matrix}
\begin{gathered}
\longrightarrow \text{ряд розподілу}\\
\mathbb{E}(\xi_2| \xi_1)
\end{gathered}
$$
$$
\mathbb{E}[ \mathbb{E} (\xi_2 | \xi_1)] =
\frac{  \sum\limits_{j = 1}^{n}  {y_j \cdot p_{1j}} }   {  \sum\limits_{j = 1}^{ n}{ p_{1j}}} \cdot  \sum\limits_{j = 1}^{ n}{ p_{1j}}+ ... +
\frac{  \sum\limits_{j = 1}^{n}{y_j \cdot p_{mj}} }   {  \sum\limits_{j = 1}^{ n}{ p_{mj}}} \cdot  \sum\limits_{j = 1}^{ n}{ p_{mj}} =  \sum\limits_{i = 1}^{m}{  \sum\limits_{j =1 }^{ n}{ y_j p_{ij}}} = \mathbb{E}\xi_2
$$
$$
\mathbb{E}[ \mathbb{E} (\xi_2 | \xi_1)] = \mathbb{E}  \xi_2  \quad \quad \quad \mathbb{E}[ \mathbb{E} (\xi_1 | \xi_2)] = \mathbb{E}  \xi_1
$$
\subsubsection{Абсолютно неперервний вектор}
$$
\overline{\xi}  \begin{bmatrix}
 \xi_1 \\ \xi_2
\end{bmatrix} \qquad f_{\overline{\xi}} (x,y) - \text{ сумісна щільність розподілу.}
$$
$
f_{\xi_2 | \xi_1} (y|y) = f_{\xi_2| \xi_1 = x} (y)
$ - умовна щільність другої координати за першою.\\

$F_{\xi_2 | \xi_1 = x} (y)$ - умовна функція розподілу $\xi_2$ за умови $\xi_1 = x$.\\
$ F_{\xi_2 | \xi_1 = x} (y) = \mathbb{P} \left\lbrace \xi_2 < y | \xi_1 = x  \right\rbrace = \frac{\mathbb{P} \left\lbrace \xi_1  = x , \xi_2 < y \right\rbrace}{ \mathbb{P} \left\lbrace \xi_1 = x \right\rbrace} = \frac{0}{0} $
\\$$   F_{\xi_2 | \xi_1 = x} (y) = \lim\limits_{\varepsilon \to  \infty}{
 \mathbb{P} \left\lbrace  \xi_2 | \xi_1 \in [x, x+ \varepsilon ) \right\rbrace
 } =  \frac{ \mathbb{P} \left\lbrace \xi_1 \in [x, x+ \varepsilon ) , \xi_2 \in (- \i, y)\right\rbrace}{ \mathbb{P} \left\lbrace \xi_1 \in [x, x + \varepsilon ) \right\rbrace} =
  $$
	$$
	=  \lim\limits_{\varepsilon \to  0} { \frac{\int\limits_{x}^{ x+\varepsilon }{ ds  \int\limits_{- \i}^{ y}{ f_{ \overline{\xi}}}(s,t)dt}}{
  \int\limits_{x}^{ x+\varepsilon }{ f_{\xi_1} (s) ds}}
  } =    \lim\limits_{\varepsilon \to  0} { \frac{\varepsilon \cdot\int\limits_{x}^{ x+\varepsilon }{ ds  \int\limits_{- \i}^{ y}{ f_{ \overline{\xi}}}(s,t)dt}}{ \varepsilon \cdot
  \int\limits_{x}^{ x+\varepsilon }{  f_{\xi_1} (s) ds}}
  }  =  \fbox{$ \displaystyle\frac{  \int\limits_{- \i}^{ y}{ f_{ \overline{\xi}} }(x,t)dt}{ f_{\xi_1} (x)} = F_{\xi_2 | \xi_1 = x} (y)$}
	$$
Знаючи умовну функцію розподілу, можемо знайти умовну щільність:
	$$
	f_{\xi_2 | \xi_1 = x} = F_{\xi_2| \xi_1 = x}'(y) = \frac{f_{\overline{\xi}}(x,y)}{ f_{\xi_1}(x)}
	$$
Знайдемо умовне математичне сподівання $\xi_2$ за $\xi_1$
	$$
	\mathbb{E}(\xi_2 | \xi_1 = x) =  \int\limits_{-\i}^{ +\infty}{ y \cdot f_{\xi_2 | \xi_1 = x} dy} =  \int\limits_{-\i}^{ +\infty}{ y \cdot \frac{f_{\overline{\xi}}(x,y)}{ f_{\xi_1} (x)}  dy} =
	\frac{
	\int\limits_{-\i}^{ +\infty}{ y \cdot f_{\overline{\xi}}(x,y)}dy}{ f_{\xi_1} (x)}
	$$
$ \mathbb{E} (\xi_2| \xi_1)$ - випадкова величина, яка спочатку визначає, куди попала умова (чому дорівнює $x \longleftarrow \xi_1$), а далі визначає $\mathbb{E}(\xi_2| \xi_1 = x)$.\\
$\mathbb{E}(\xi_2| \xi_1)$ - набуває значення $ \mathbb{E}(\xi_2 | \xi_1 = x)$, коли $\xi_1$ набула значення $x$.
$$\xi_1 \longrightarrow x \Rightarrow \mathbb{E}(\xi_2 | \xi_1) \longrightarrow \mathbb{E}(\xi_2 | \xi_1 = x)$$
$ \mathbb{E}(\xi_2 | \xi_1)$ є функцією від $\xi_1$. Якою? $ \mathbb{E}(\xi_2 | \xi_1 = x)$\\
$ \mathbb{E} (\xi_2 | \xi_1 ) = \Psi (\xi_1)$, де $ \Psi(x) = \mathbb{E}( \xi_2 | \xi_1 = x)$

\begin{center}
	Формула повного математичного сподівання (?$\mathbb{E}[\mathbb{E}(\xi_2 | \xi_1)] = \mathbb{E}\xi_2$?)
\end{center}
$$\mathbb{E}[\mathbb{E}(\xi_2 | \xi_1)] = \mathbb{E}\Psi  (\xi_2) =  \int\limits_{-\i}^{ +\infty}{\Psi(x) \cdot f_{\xi_1} (x)dx} =  \int\limits_{-\i}^{ +\infty}{
\frac{
\int\limits_{-\i}^{ +\infty}{ y \cdot f_{\overline{\xi}}(x,y)}dy}{ f_{\xi_1} (x)} \cdot f_{\xi_1} (x) dx} = \mathbb{E} \xi_2$$

\section{Характеристичні функції}

$\xi$ - випадкова величина. Загальна характеристика такої величини - функція розподілу. Існує для кожної величинию. Також є характеристики, такі як ряд розподілу та щільність розподілу - існують не завжди. Введемо ще одну характеристику, яка буде існувати для будь-якої випадкової величини.\\
\bd
\textbf{Характеристична функція випадкової величини.}
$$
\chi_ \xi (t) = \mathbb{E}(\cos{(t \xi)} + i \sin{(t \xi)}   ) =  \mathbb{E}( e^{ it \xi})
$$
\begin{center}
	\fbox{$ X_ \xi : \mathbb{R} \rightarrow \mathbb{C}$}
\end{center}
\ed
Як шукати? Дискретний випадок: $  \sum\limits_{i = 1}^{ n(\infty)}{ e^{itx_i}p_i}$\\
Для абсолютно неперервної величини: $  \int\limits_{-\i}^{ +\infty}{ e^{itx}f_ \xi (x)dx}$

\subsection{Властивості характеристичних функцій}

$\xi$ - випадкова величина. $\chi_ \xi(t) = \mathbb{E} e^{et \xi} =  \int\limits_{-\i}^{ +\infty}{ e^itx f_{\xi}(x)dx}$\\

1. Характеристична функція є унікальною характеристикою ймовірнісного розподілу.\\
2. $ \chi_{\xi} (0) = 1  \qquad \chi_{\xi} (0) =  \mathbb{E} e^0 = 1$.\\
3. $	\left| \chi_ \xi (t)\right| \leq 1 \quad \forall t \in \mathbb{R}$
$$
\left| \mathbb{E} e^{it \xi} \right| = \left| \mathbb{E}\cos{(t \xi)} + i \mathbb{E} \sin{(t \xi)}   \right| = \sqrt{
\mathbb{E} \cos{(t \xi)}^2 + \mathbb{E} \sin{(t \xi)}^2
}  = 1
$$
4. $\chi_ \xi$ - неперервна за $t$ для $\xi$ - абсолютно неперервна випадкова величина.
$$
\chi_ {\xi} (t) =  \int\limits_{-\i}^{ +\infty}{ e^{itx} f_{\xi} (x) dx}
\quad \longleftarrow \quad
\chi_ {\xi} (t+h) =  \int\limits_{-\i}^{ +\infty}{ e^{i(t+h)x} f_{\xi} (x) dx}
$$
$$
(\chi_{\xi} - \text{ непервна в т. } t ) \Longleftrightarrow  \lim\limits_{h\to0}{ \chi_ \xi (t+h) } = \chi_{\xi} (t)
$$
Для $
 \lim\limits_{h\to  0}{  \int\limits_{-\i}^{ +\infty}{ e^{i(t+h)x} f_{\xi}(x)}dx} =  \int\limits_{-\i}^{ +\infty}{ e^{itx} f_ \xi(x) dx}
$, треба щоб $  \int\limits_{ -\i}^{ +\infty}{e^{itx}f_ \xi (x)dx}$ збігався рівномірно на $t \in \mathbb{R}$. $ \left| e^{itx} \cdot f_{\xi}(x) \right| = \left| f_{\xi}(x) \right| = f_{\xi}(x) = M(x) $- мажорантний ряд.\\
$$
 \int\limits_{-\i}^{ +\infty}{ M(x) dx } =  \int\limits_{-\i}^{ +\infty}{f_{\xi}(x)dx } = 1 < \i
$$
$
 \int\limits_{-\i}^{ +\infty}{ e^{itx} f_{\xi} (x) dx }
$ - збігається рівномірно за озн. Вейерштрасса.\\
5. $ \xi_1 \independent \xi_2 \Longrightarrow \chi_{\xi_1 + \xi_2} (t) = \chi_ {\xi_1} (t) \cdot  \chi_ {\xi_2} (t) $\\
6. Якщо $ \exists \mathbb{E} \xi^n$, то $\mathbb{E} \xi^n = \frac{1}{i^n}\cdot \chi^{(n)}_ \xi (0) $.
\begin{proof} В неперервному випадку.
$$
\mathbb{E} \xi^n =  \int\limits_{-\i}^{ +\infty}{x^n \cdot f_ {\xi} (x)dx}
$$
$$
\chi (t) = \mathbb{E} e^{it \xi} =  \int\limits_{-\i}^{ +\infty}{ e^{itx} f_ \xi(x) dx}
$$
$$
\chi^{(n)}(t) =  \int\limits_{-\i}^{ +\infty}{ (ix)^n e^{itx} f_ \xi  (x) dx}
$$
Але потрібна рівномірна збіжність. Скористаємося озн. Вейерштрасса:
$$
\left| (ix)^n e^{itx} f_ \xi  (x)  \right| = \left| x \right|^n \cdot f_ \xi (x) = M(x):
$$
$$
 \int\limits_{-\i}^{ +\infty}{ M(x)dx} =  \int\limits_{-\i}^{ +\infty}{ \left| x \right|^n \cdot f_{\xi}(x)dx < \i }
$$
$$
\chi^{(n)}(0) = i^n  \int\limits_{-\i}^{ +\infty}{ x^n \cdot f_{\xi} (x)dx} = i^n \cdot \mathbb{E}(\xi^n)
$$
\end{proof}


\end{document}
