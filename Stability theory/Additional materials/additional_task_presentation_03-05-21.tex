\documentclass{beamer}
\usepackage[english, russian, ukrainian]{babel}
\usepackage{tikz}
\usepackage{amsfonts, amsmath, mathtools}
\usetheme{Warsaw}
\usecolortheme{default}

\title{Додаткове завдання: Тиждень 13.}
\author{Терещенко Д., Гапон М., Людомирський Ю.}
\institute{KA-96, IASA}
\date{2021}

\def\d{\partial}

\begin{document}

\frame{\titlepage}

\begin{frame}

\frametitle{Задача зі старшими похідними. № 7.16.}
{\small
\textit{Вони не знали навіть того, що вони вже ``знали''. (C) Р. Фейнман}
}
$$
\begin{dcases}
  \int\limits_{0}^{1}{ \ddot{x}^2 + \dot{x}^2 \mathrm{d} t \to \mathrm{extr}} \\
  x(0) = 3 \quad \dot{x} (0) = 0 \quad x(1) = \cosh{1} \quad \dot{x} (1) = \sinh{1}
\end{dcases}
$$
$$
F(t, \dot{x}, \ddot{x}) =  \ddot{x}^2 + \dot{x}^2 \text{ -- з умови.}
$$
\begin{itemize}
  \item Складемо рівняння Ейлера-Пуассона:
  $$
  \frac{\d F}{\d x} - \frac{\mathrm{d}}{ \mathrm{d} t} \frac{\d F}{\d \dot{x}} + \frac{\mathrm{d}^2}{\mathrm{d} t^2} \frac{\d F }{\d \ddot{x}} = 0
  $$
  $$
  \frac{\d F}{\d \dot{x}} = 2 \dot{x} ; \qquad \qquad \frac{\mathrm{d}}{\mathrm{d} t} \frac{\d F}{\d \dot{x}} = 2 x^{(2)};
  $$
  $$
  \frac{\d F}{\d \ddot{x}} = 2 \ddot{x} ; \qquad \qquad \frac{\d^2}{\mathrm{d} t^2} \frac{\d F}{\d \ddot{x}} = 2x^{(4)}  ;
  $$
\end{itemize}
\end{frame}


\begin{frame}
Підставимо та спростимо:
$$
0 -  2 x^{(2)} + 2 x^{(4)} = 0
$$
\fbox{$x^{(4)} - x^{(2)} = 0$}
 \ \textbf{ -- рівняння Ейлера-Пуассона.}

Складемо характеристичне рівняння:
$$
\lambda^4 - \lambda^2 = 0 \quad \Longrightarrow \quad \lambda_1=0 \text{ кр.2 }, \lambda_2 = 1, \lambda_3 = -1
$$
Отже, отримали \alert{сімейство екстремалей:}
$$
x = C_1 + C_2 t + C_3 e^{t} + C_4 e^{-t}
$$
\end{frame}



\begin{frame}
\begin{itemize}
  \item Знайдемо допустимі екстремалі. Для цього підставимо числа з умови:
\end{itemize}
\begin{enumerate}
  \item $x(0) = C_1 + C_3 + C_4= 1$;
  \item $\dot{x} (0) = C_2 + C_3 - C_4 = 0$;
  \item $x(1) = C_1 + C_2 + C_3 e + C_4 e^{-1} = \cosh{1} $
  \item $\dot{x}(1) = C_1 + C_2 + C_3 e - C_4 e^{-1} = \sinh{1} $
\end{enumerate} \par
$$
\begin{array}{r c l}
   (3) - (4) & : \qquad & 2C_4 e^{-1} = \cosh 1 - \sinh{1} = e^{-1} \Longrightarrow C_4 = \frac{1}{2};\\
   (1)  & : \qquad & C_1 + C_3 = \frac{1}{2} ;\\
    (2)& : \qquad & C_2 + C_3 = \frac{1}{2}  .\\
\end{array}
$$
З останніх двох рівностей отримуємо: $C_1 = C_2$.
$$
\begin{array}{r c l}
   (3) + (4) & : \qquad & 2 C_1 + 2C_2 + 2 C_3 e = \cosh 1 + \sinh 1  = e.\\
\end{array}
$$
\end{frame}
\begin{frame}
  Підставляємо: $C_2 = C_1 = \frac{1}{2} - C_3  \quad \Longrightarrow \quad
  2 - 4 C_3 + 2 C_3 e = e;
  $
  $$
  C_3 (2 e - 4) = e -2 \Longrightarrow C_3 = \frac{1}{2} \Longrightarrow C_2 = C_1 = \frac{1}{2} - \frac{1}{2} = 0
  $$
  Отже, \alert{єдина допустима екстремаль має вигляд:}
\begin{center}
 \fbox{$\tilde{x} (t) = \dfrac{e^t + e^{-t}}{2} = \cosh t  $}
\end{center}
\end{frame}
\begin{frame}
\begin{itemize}
  \item \textbf{Перевірка}. $\forall h \in C^2 [a,b] : h(0) = h(1) = h'(0) = h(1) = 0$.
\end{itemize}
$$
J(\tilde{x} + h) - J(\tilde{x}) =  \int\limits_{0}^{1}{ ((\tilde{x} + h)'')^2 + ((\tilde{x} + h)')^2 \mathrm{d} t} -  \int\limits_{0}^{ 1}{
(\tilde{x}'')^2 + (\tilde{x}')^2 \mathrm{d} t =
}
$$
$$
=  \int\limits_{0}^{1}{
2 \underbrace{\tilde{x}'' h''}_{(*)} + ( h'' )^2 + 2 \tilde{x}' h' + (h')^2 \mathrm{d} t
 } =
$$
$$
=
\left| \
\begin{gathered}
\text{Візьмемо ($*$) частинами:}\\
 \int\limits_{0}^{1}{\underbrace{\tilde{x}''}_{u} \underbrace{h''}_{v} \mathrm{d} t}  = \left| \begin{gathered}
  u =  \tilde{x}'' \\
  u' =  \tilde{x}'''
 \end{gathered} \
\begin{gathered}
 v' = h''\\
 v = h'
\end{gathered}\right|
= \underbrace{\tilde{x}'' h' \bigg|_{0}^{1}}_{=0} -  \int\limits_{0}^{1}{ \tilde{x}''' h' \mathrm{d} t} = \\
= \left| \begin{gathered}
 \text{Знайдемо третю похідну:}\\
 \tilde{x}''' = (\cosh t)^{(3)}  = \sinh t
\end{gathered} \right| = -  \int\limits_{0}^{1}{ \sinh t \cdot h' \mathrm{d} t}
\end{gathered}
   \  \right|=
$$
\end{frame}
\begin{frame}
   $$= \int\limits_{0}^{1}{ (h'')^2 + (h')^2 \mathrm{d} t } + 2  \int\limits_{0}^{1}{ \underbrace{\tilde{x}'' h''}_{=-\sinh t \cdot h'} + \underbrace{\tilde{x}'}_{= \sinh t} h' \mathrm{d} t} = $$
   $$
   = \int\limits_{0}^{1}{ (h'')^2 + (h')^2 \mathrm{d} t } + 2  \underbrace{\int\limits_{0}^{1}{ - \sinh t \cdot h' + \sinh t \cdot h' \mathrm{d} t}}_{=0} =
   $$
   $$ = \int\limits_{0}^{1}{ (h'')^2 + (h')^2 \mathrm{d} t }  \mathbf{\geq 0} \qquad
\begin{gathered}
  \forall h \in C^2 [a,b] :\\
  h(0) = h(1) = h'(0) = h(1) = 0
\end{gathered}
   $$
   Отримали \alert{глобальний мінімум} для $\tilde{x} (t) = \cosh t  $.
\end{frame}
\end{document}
