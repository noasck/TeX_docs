\subsection{Стійкість розв'язків лінійних систем}

Лінійна неоднорідна система рівнянь має вигляд (далі ЛНС):

\be
\overline{x}' = A(t) \overline{x} + \overline{f} (t), \text{ де }
\ee

$
A(t) \in \mathbb{R}^{n \times n}, A(t) \in C [ a, + \infty], \overline{f} \in C[a, + \infty]
$
Застосуємо заміну: $ \overline{z} (t) = \overline{x} - \vphi (t)$, де

$ \overline{z}(t)$ - нова невідома вектор-функція, а $\vphi (t)$ - розв'язок, який ми маємо дослідити на стійкість.

Отримали лінійну однорідну систему першого порядку (далі ЛОС):

$$ \overline{z} '   + \vphi = A(t)\overline{z} + A(t )\vphi + \overline{f}(t) $$
\be
\overline{z}' = A(t) \overline{z}
\ee

Заміною ми звели дослідження довільного розв'язку лінійної неоднорідної системи до дослідження нульового розв'язку відвовідної ЛОС. Таким чином, приходимо до висновку, що усі розв'язки є одночасно стійкими або не стійкими, або асимптотично стійкими. А отже, розглядаючи будь-яку лінійну систему, можемо говорити про стійкість не окремого розв'язку, а системи в цілому.\\

Розв'яжемо ЛОС (3) (перейдемо для зручності до змінної $x$): $ \overline{x}'  =  A(t) \overline{x} $ - ЛОС (3).

$X(t) $ - її фундаментальна матриця (далі ФМ). Тоді з.р: $ \overline{x} (t) = X(t) \cdot \overline{C}$, де $\overline{ C} \in \mathbb{R}^n$.
Розв'язок з. К. з початковими умовами $ \overline{x} (t_0) = \overline{x_0}$:
$$
\overline{x}_0 = X(t_0) \cdot C \Rightarrow \overline{C} = X^{-1} (t_0) \cdot \overline{x_0} \Rightarrow \fbox{$x(t) = X(t) X^{-1}(t_0) \overline{x}_0$}
$$


\begin{boxteo}[Про стійкість ЛОС]\quad \\
а) (3) - ст. $\Longleftrightarrow \exists K > 0: \sup\limits_{t\geq  a} ||X(t) || \leq K$.\\
б) (3) - ас. ст. $\Longleftrightarrow  ||X(t)|| \to 0 $, при $ t \to +\infty$.\\
в) (3) - нест. $ \Longleftrightarrow \exists \left\lbrace t_n \right\rbrace_{n=1}^{\infty} : ||X(t_n)|| \to +\infty $, при $n \to \infty$
\end{boxteo}



\begin{proof}
 а) \fbox{$\Leftarrow$} Нехай $ \exists K > 0 : \sup\limits_{t\geq a} ||X(t)|| \leq K$.

 Доведемо стійкість розв'язку $\overline{x} (t) = \vec{0}. $.

За означенням, візьмемо розв'язок довільної задачі Коші з початковими умовами $\overline{x } (t_0) = \overline{x}_0$.\\
Нехай $|| \overline{x}_0 || < \delta$ і розглянемо $ || \overline{x} (t)|| = || X(t) \cdot X^{-1} (t_0) \cdot \overline{x}_0 || \leq ||X(t)|| \cdot || X^{-1} (t_0)|| \cdot || \overline{x}_0|| \leq K \cdot || X^{-1} (t_0)|| \cdot || \overline{x}_0|| < K || X^{-1} (t_0)||\delta < \varepsilon $ при $ \delta = \dfrac{\varepsilon }{K || X^{-1} (t_0)|| + 1} $.\\
Отже, $\forall t_0 \geq  a \quad \forall \varepsilon >0 \quad \exists \delta > 0 \quad \left(  \delta = \dfrac{\varepsilon }{K || X^{-1} (t_0)|| + 1}  \right)$  для довільного розв'язку з $ || \overline{x}_0|| < \delta$ справедливо $ ||\overline{x} (t)|| < \varepsilon  \Longrightarrow $ стійкість розв'язку (системи).\\
\fbox{$\Rightarrow$} Нехай (3) - стійка. Припустимо від супротивного, що $\exists  \left\lbrace t_n \right\rbrace_{n\geq 1}^{\infty} : ||X(t_n)|| \to {+\infty} $ при $ n \to \infty$.\\
Тоді $\exists j = \overline{1, n} : || \overline{x}^{j} (t_n)|| \to \infty$, де $\overline{x}^j$  - це $j$-тий стовпчик ФМ. \\ Покладемо $\forall \delta > 0:$
$$
\overline{x}^{\delta}_0 = \frac{\delta X(t_0) \overline{e}_j}{2 ||X(t_0)||} \text{ , де }\overline{e}_j = \begin{bmatrix}
 0\\
 \vdots\\
 1\\
 \vdots\\
 0
\end{bmatrix} - j
$$

Тоді $|| \overline{x}_0^{\delta}|| = \dfrac{1}{ 2 ||X(t_0)||}  \cdot \delta ||X(t_0) \cdot \overline{e}_j|| < \delta $.\\
Розглядаємо розв'язок з.К. з початковими умовами $ \overline{x} (t_0) = \overline{x} _0 ^ \delta$. Маємо:
$$
\overline{x} (t) = X(t) \cdot X^{-1}(t_0) \cdot \overline{x}_0 ^\delta = X(t) X^{-1} (t_0) \cdot \dfrac{ \delta X(t_0) \overline{e}_j}{ 2 ||X(t_0)||} = \frac{\delta}{2} \cdot \frac{X(t) \overline{e}_j}{ ||X(t_0)||} =   \frac{\delta}{2 ||X(t_0)|| } \cdot \overline{x}^j (t)
$$
$$
\Longrightarrow \forall \varepsilon >0 \quad \exists n_0 \in \mathbb{N} \quad : \forall n \geq n_0
$$
$$
||\overline{x} (t_n)|| = \frac{\delta}{ 2 ||X(t_0)|| } \cdot ||\overline{x}^j (t_n)|| \to \infty > \varepsilon  -
$$
З попереднього випливає нестійкість $ \Rightarrow  $ суперечність початковій побудові $ \Rightarrow$ a).\\
Пункт б) доводиться аналогічно а).\\
Пункт в) випливає із пукнта а).
\end{proof}


\subsection{Стійкість ЛОС зі сталою матрицею.}

\begin{equation}
\overline{x} ' (t) = A \overline{x} (t) \text{, де $A$ - стала матриця $n \times n$}
\end{equation}
\begin{boxteo} \quad \\
a) (4) - стійка $ \Longleftrightarrow  \forall \lambda $ - власне число матриці $A$: \\
$\Re \lambda \leq 0$, причому якщо $ \Re \lambda = 0$, то йому відповідають лише одновимірні клітини Жордана. \\
б) (4) - асимптотично стійка $ \Longleftrightarrow \forall \lambda $ - власні числа матриці $A: \Re \lambda < 0 $.\\
в) (4) - нестійка $ \Longleftrightarrow $ не є стійкою.
\end{boxteo}

\begin{proof}
  Нехай $ \lambda = \alpha + i\beta$ - власне число матриці $A \Rightarrow $ у ФМ цьому власному числу відповідає розв'язок: \\
  - якщо $ \lambda$ відповідають лише одновимірні клітини Жордана:
  $$
  \overline{x} (t) = e^{\alpha t} ( \overline{Q} _0 \cos{(\beta t)} + \overline{R}_0 \sin{(\beta t)} )
 $$
 - якщо клітина Жордана розміру $l$:
 $$
 \overline{x} (t) = e^{\alpha t} ( \overline{Q} _{l-1} \cos{(\beta t)} + \overline{R}_{l-1} \sin{(\beta t)} )
 $$
 Тоді:\\
якщо $ \Re \lambda = \alpha < 0  \Rightarrow || \overline{x} (t)|| \to 0 $ за $t \to \infty$.\\
якщо $ \Re \lambda = \alpha >0 \Rightarrow || \overline{x} (t)|| \to + \infty $ за $t \to \infty$.\\
якщо $ \Re \lambda = 0 $, то:\\
\hspace*{1cm} - якщо лише одновимірні клітини Жордана: $||\overline{x}(t) ||$ - обмежена.\\
\hspace*{1cm} - якщо клітини Жордана розмірності $l \geq 2$ : $ || \overline{x} (t)|| \to + \infty $ за $t \to \infty$.
\end{proof}
\begin{example}
    $$\begin{cases}
        \dot{x} = 3x + y \\
        \dot{y} = y-x
    \end{cases} \qquad \qquad A = \begin{bmatrix}
     3 & 2 \\ -1 & 1
    \end{bmatrix}
    $$
    $$
    \det \left( A - \lambda I  \right) = \begin{vmatrix}
      3 - \lambda & 1 \\
      -1 & 1- \lambda
    \end{vmatrix}  = (3 - \lambda) (1- \lambda) + 1 = \lambda^2 - 4 \lambda + 3 = (\lambda-2 )^2
    $$
    Отримали дійсне $\lambda=2$, кратності 2. $ \Re \lambda = 2 > 0 \Rightarrow $ Система нестійка.
\end{example}
\textbf{Зауваження.} Перевірку умов теореми в частині, що стосується стійкості, можно здійснювати нне знаходячи власних чисел матриці $A$.

\begin{boxteo}[Критерій Рауса-Гурвіца]
$$
\det \left( A - \lambda I \right) = a_0 \lambda^n + a_1 \lambda^{n-1} + ... + a_n ; \quad a_1 \in \mathbb{R}, a_0 > 0;
$$
$
\Re \lambda < 0 \quad \forall \lambda \Leftrightarrow
$ всі головні мінори матриці Гурвіца $H$ додатні, де $ H =  \left( h_{ij} \right)^n_{ij=1} $
$$
h_{ij} = \begin{cases}
    a_{2i-j}, & 0 \leq 2i - j \leq  n;\\
    0 , & \text{інкше.}
\end{cases}
$$
\end{boxteo}
