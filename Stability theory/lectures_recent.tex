\section{Лекція 8}
\subsection{Задача зі старшими похідними.}
Нехай $C^k [a,b]$ --- простір $k$-раз неперервно-диференційованих на $[a,b]$ функцій.\par

$C^k[a,b]$ --- нормований простір з нормою:
$$
||\varphi(x)||_{C^k[a,b]} =
\max\limits_{x\in[a,b]}{ |\varphi(x)|} +
 \max\limits_{x\in[a,b]}{|\varphi'(x)|} + \dots
  \max\limits_{x\in[a,b]}{|\varphi^{(k)}(x)|}
$$
Нехай $a,b \in \mathbb{R}, a<b$ -- задані числа.
$$
F(x,y,p_1, \dots, p_k) \in C^{k+1}\left( [a,b] \times \mathbb{R}^{k+1} \right) \text{ -- задана функція.}
$$
Розглядаємо інтеграл:
$$
J(y) =  \int\limits_{a}^{ b}{
F(x, y, y' , y'', \dots , y^{(k)}) \mathrm{d} x
}\  \text{ на множині функцій:}
$$
$$
M = \left\lbrace
y\in C^{k}[a,b] \ \bigg| \  y^{(j)} (a) = A_j , y^{(j)} (b) = B_j , j=\overline{0, k-1}
 \right\rbrace \qquad A_j ,  B_j \in \mathbb{R}
$$
Функції $y \in M$ називаються \textit{допустимими}.
\begin{defo}
Означення слабкого локального(глобального) мінімуму (максимуму) інтегралу $J(y)$ на $M$ аналогічні попереднім задачам.
\end{defo}
\begin{defo}
 Задача пошуку слабкого  локального(глобального) мінімуму (максимуму) інтегралу $J(y)$ називається \textbf{задачею зі старшими похідними.}
\end{defo}
Для довільної функції $y\in M$, для довільної функції:
$$
h \in C^{k}[a,b] : h^{(j)}(a) = h^{(j)}(b) = 0 \ \forall \alpha \in \mathbb{R}
$$
розглянемо сімейство функцій:
$$
y(x, \alpha) = y(x) + \alpha h(x)
$$
Тоді $\forall \alpha \in \mathbb{R} \  y(x, \alpha) \in M$. Розглядаємо:
$$
J(y + \alpha h) =  \int\limits_{a }^{b}{F(x, y + \alpha h(x), y' + \alpha h' (x), \dots , y^{(k)} + \alpha h^{(k)}) \mathrm{d} x}
$$
Тоді функція: $\Phi (\alpha) = J(y + \alpha h)$ є неперервно-диференційованою по $\alpha$ в деякому околі т.$\alpha = 0$:
$$
\Phi'(0) = \frac{\mathrm{d}}{\mathrm{d} \alpha} J(y + \alpha h)\bigg|_{\alpha = 0}
$$
\begin{defo}
 \textbf{Допустимим приростом} функції $y\in M$ називають довільну функцію $h(x) \in C^k[a,b]:$
 $$
 h^{(j)} (a) = h^{(j)} (b) = 0 \qquad \forall j = \overline{0, k-1}
 $$
 Вираз $ \dfrac{\mathrm{d}}{\mathrm{d} \alpha} J(y + \alpha h) \bigg|_{\alpha = 0},$ де $h(x)$ -- допустимий приріст, називається \textit{першою варіацією} функціоналу $J(y)$ на функції $y(x)$ в напрямку $h(x)$.
 Позначення:
 $$
 \delta J(y, h) = \frac{\mathrm{d}}{\mathrm{d} \alpha} J (y + \alpha h) \bigg|_{\alpha = 0}
 $$

\end{defo}
Нехай $\ty(x)$ -- розв`язок задачі зі старшими похідними. Тоді т.$\alpha = 0 $ є точкою локальнного екстремуму функції $\Phi (\alpha)$ . Отже $\Phi' (0) = 0 \Longrightarrow$ \fbox{$\delta J(\ty, h)=0$} для кожного допустимого приросту $h$. Таким чином, отримали необхідну умову розв'язку в термінах першої варіації функціоналу.
\begin{lema}[Узагальнена лема Лагранжа]
Якщо $f \in C[a,b] $ і $ \int\limits_{a}^{b}{f(x)h(x) \mathrm{d} x}  = 0$:
$$
\forall h(x) \in C^{k}[a,b] : h^{(j)} (a) = h^{(j)} (b) = 0 , j = \overline{0, k-1}
$$
то $f(x) = 0 \quad \forall x\in [a,b]$.
\end{lema}
\begin{proof}
 Припустимо від супротивного: $\exists x_0 \in [a,b] : f(x_0) \neq 0$. Нехай $f(x_0)>0$.\par
 Оскільки $f\in C[a,b]$, то знайдеться окіл т. $x_0 : \Theta_{\varepsilon} (x_0) = (x_0 - \varepsilon, x_0 + \varepsilon )$ такий, що:
 $$
 f(x) >0 \qquad \forall x \in \Theta_{\varepsilon}(x_0)
 $$
 Візьмемо:
 $$
 h(x) = \begin{cases}
  (x - (x_0 - \varepsilon))^{2k} (x - (x_0 + \varepsilon))^{2k}, & x \in \Theta_{\varepsilon};\\
  0 , & x \notin \Theta_{\varepsilon};\\
 \end{cases}
 $$
 Тоді $h \in C^{k}[a,b] ; h^{(j)} (a) = h^{(j)}(b) = 0, j=\overline{0, k-1}$. При цьому:
 $$
 0 =  \int\limits_{a}^{b}{ f(x) h(x) \mathrm{d} x} =  \int\limits_{x_0 - \varepsilon}^{x_0 + \varepsilon}{f(x) h(x)} >0,
 $$
 оскільки $f(x)h(x) >0$ на $\Theta_{\varepsilon}$. Отримали \textbf{протиріччя}.
\end{proof}
\newpage

\begin{boxteo}
 Нехай допустима функція $\ty (x) \in C^{2k} [a,b]$ є розв'язком задачі зі старшими похідними. Тоді $\ty(x)$ задовольняє рівняння \textit{Ейлера-Пуассона} на $[a,b]$:
 $$
 \frac{\d F}{\d y} - \frac{\mathrm{d}}{\mathrm{d} x} \frac{\d F}{\d y'} + \frac{\d^2}{\mathrm{d} x^2} \frac{\d F}{\d y''} - \dots + (-1)^k \frac{\mathrm{d}^k}{\mathrm{d} x^k} \frac{\d F}{\d y^{(k)}} = 0
 $$
\end{boxteo}

\begin{proof}
   Нехай допустима функція $\ty (x) \in C^{2k} [a,b]$ є розв'язком задачі зі старшими похідними. Тоді:
   $$
\begin{gathered}
\forall h(x) \in C^{k}[a,b] : h^{(j)} (a)  = h^{(j)}(b) = 0 \\
0 = \delta J(\ty, h) = \frac{\mathrm{d}}{\mathrm{d} \alpha} J(\ty + \alpha h)\bigg|_{\alpha = 0} =
\end{gathered}
   $$
   $$
   = \frac{\mathrm{d}}{\mathrm{d} \alpha }\!  \left(   \int\limits_{a}^{b}{
   F(x,\ty + \alpha h(x) , \ty' + \alpha h'(x),\dots,
\ty^{(k)} + \alpha h^{(k)} (x)
    ) \mathrm{d} x
   } \right)\!\!\bigg|_{\alpha = 0} =
   $$
   $$
    = \int\limits_{a}^{b}{
    \frac{\d F(x,\ty, \ty', \dots, \ty^{(k)})}{\d y} \cdot h(x) +  \frac{\d F(x,\ty, \ty', \dots, \ty^{(k)})}{\d y'} \cdot h'(x) +
    }
   $$
   $$
   + \frac{\d F(x,\ty, \ty', \dots, \ty^{(k)})}{\d y''} \cdot h''(x) + \dots + \frac{\d F(x,\ty, \ty', \dots, \ty^{(k)})}{\d y^{(k)}} \cdot h^{(k)}(x) \  \mathrm{d} x
   $$
   Проінтегрувавши частинами кожен доданок, починаючи з дрогого, стільки разів, скільки потрібно, зоб позбутися похідних $h(x)$ (2-гий доданок -- 1 раз, 3-тій додаок -- 2 рази і т.д.) та враховуючи, що $h^{(j)} = a = h^{(j)} (b) = 0, j = \overline{0, k-1}$. \par Далі з узагальненої леми Лагранжа отримаємо рівняння Ейлера-Пуассона.
\end{proof}
Будь-який розв'язок рівняння Ейлера-Пуассона називається \textbf{екстремаллю}. Будь-яка екстремаль, що є допустимою функцією, називається \textit{допустимою екстремаллю} функціоналу $J(y)$.

\subsection*{Приклад.}
$$
\begin{dcases}
 J(y) =  \int\limits_{0}^{1}{ (y'')^2 + (y')^2 - 2 y  \mathrm{d} x } \to \mathrm{extr};\\
 y(0) = y' (0) = 0 \quad y(1) = - \frac{1}{2} \quad y'(1) = -1 .
\end{dcases}
$$
\begin{enumerate}
  \item Складаємо і розв'язуємо рівняння Ейлера-Пуассона:
  $$
  \frac{\d F}{\d y} - \frac{\mathrm{d}}{\mathrm{d} x} \frac{\d F}{\d y'} + \frac{\mathrm{d}^2}{ \mathrm{d} x ^2} \frac{\d F}{\d y''} = 0
  $$
  За умовою: $ F(x, y, y', y'') = (y'')^2 + (y' )^2 - 2y$.
  $$
  \frac{
  \d F
  }{
\d y
  } = -2 ; \  \frac{\d F}{\d y'} = 2 y' ; \  \frac{\mathrm{d}}{\mathrm{d} x} \left( \frac{\d F}{
  \d y'
  }  \right)  = 2 y'' ; \  \frac{\d F}{\d y''
  }  = 2y''; \  \frac{\mathrm{d}^2}{ \mathrm{d} x^2} \left( \frac{\d F}{\d y''}  \right)  = 2 y^{(4)}.
  $$
Маємо:
$$
-2 - 2 y'' + 2 y^{(4)} = 0
$$
$$
y^{(4)} - y'' = 1 \text{ -- рівняння Ейлера-Пуассона.}
$$
Розв'яжемо відповідне лінійне однорідне рівняння:
$$
y^{(4)} - y'' = 0
\quad \Longrightarrow \quad \lambda^4 - \lambda^2 = 0
$$
$$
\lambda^2 (\lambda-1) (\lambda+1 ) = 0 \quad \Longrightarrow \quad \lambda_1 = 0 \text{ кр. 2} ,
\lambda_2 = 1 , \lambda_3 = -1
$$
$$
y_o = C_1 + C_2x + C_3 e^x + C_4 e^{-x}
$$
Розглянемо праву частину початкового ЛНР:
$$
1 = e^{\lambda} = e^0  \Longrightarrow \lambda = 0 \text{ -- корінь кратності 2.}
$$
$$
y_{\text{ч}} (x) = Ax^2
$$
$$
y_{\text{ч}}' = 2Ax ; \qquad  y_{\text{ч}}'' = 2 A ; \qquad  y_{\text{ч}}^{(3)} = y_{\text{ч}}^{(4)} = 0
$$
Підставимо: $-2 A = 1 \Rightarrow A= - \dfrac{1}{2} \Rightarrow y_{\text{ч}}(x) = -  \dfrac{1}{2} x^2$. Загальний розв'язок:
$$
y(x) = C_1 + С_2 x + C_3 e^x + C_4 e^{-x} - \frac{1}{2} x^2 \textit{ -- сімейство екстремалей.}
$$
\item Знайдемо допустимі екстремалі:
$$
y(0) = 0 \qquad y'(0) =0 \qquad y(1) = -  \frac{1}{2}  \qquad y' (1) = -1
$$
$$
y' (x) = C_2 + C_3 e^{x} - C_4 e^{-x} - x
$$
$$
\begin{array}{r l}
y(0) = 0: & C_1 + C_3 + C_4 = 0\\
y'(0) = 0 : &C_2 + C_3 - C_4 = 0 \\
y(1) = - \frac{1}{2} : &  C_1 + C_2 + C_3 e + C_4 e^{-1 } - \frac{1}{2} = - \frac{1}{2}\\
y'(1) = -1 : & C_2 + C_3 e - C_4 e^{-1}  -1 = -1
\end{array}
$$
$$
\begin{cases}
 C_1 + C_3 + C_4 = 0\\
C_2 + C_3 - C_4 = 0 \\
C_1 + C_2 + C_3 e + C_4 e^{-1} = 0\\
C_2 + C_3 - C_4 e^{-1} = 0
\end{cases} \qquad
\begin{cases}
 C_1 = - C_3 - C_4 \\
 C_2 = C_4 - C_3 \\
 - 2 C_3 + C_3 e + C_4 e^{-1} = 0\\
 C_4 - C_3 + C_3 e - C_4 e^{-1} = 0
\end{cases}
$$
$$
 C_4 = 2 C_3 e - C_3 e^2 \qquad
 2C_3 e - C_3 e^2 - C_3 + C_3 e - 2 C_3 + C_3 e = 0 \Longrightarrow
$$
$$
C_3 = 0 \Longrightarrow C_4 = 0 \Longrightarrow C_1 = 0, C_2 = 0
$$
Отримали: \fbox{$\ty (x) = - \frac{1}{2} x^2 $} --- \textit{допустима екстремаль.}
\item Перевірка.
 $
\forall h \in C^2[a,b] : h(0) = h(1) = h'(0) = h'(1) = 0:
$
$$
J(\ty + h) - J(\ty) =  \int\limits_{0}^{1}{ (\ty'' + h'')^2 + (\ty' + h')^2 - 2 (\ty + h) \ \mathrm{d} x} -
$$
$$
-  \int\limits_{0}^{ 1}{(\ty'')^2 + (\ty')^2 - 2 \ty \ \mathrm{d} x} =  \int\limits_{0}^{1}{
2 \ty'' h'' + (h'')^2 + 2 \ty'h' + (h')^2 - 2 h
\ \mathrm{d} x
} \oeq
$$
$$
 \int\limits_{0}^{1}{2 \ty'' h'' \mathrm{d} x} = \left|\ \begin{gathered}
  \text{Візьмемо частинами:}\\
  v' = h'' \qquad u = 2\ty''\\
  v = h'  \qquad u' = 2 \ty'''
 \end{gathered} \ \right| = \underbrace{2 \ty'' h'\bigg|_0^1}_{= 0} -  \int\limits_{0}^{1}{\ty''' h' \ \mathrm{d} x} =
$$
$$
=\! \left|\ \begin{gathered}
 \text{Візьмемо частинами:}\\
 v' = h' \qquad u = 2\ty^{(3)}\\
 v = h  \qquad u' = 2 \ty^{(4)}
\end{gathered} \ \right| \!=\!   \underbrace{-2\ty''' h\bigg|_0^1}_{=0} +  \int\limits_{0}^{1}{2 \ty^{(4)}} h \ \mathrm{d} x \!=\!\! \begin{bmatrix}
 \ty = - \frac{1}{2} x^2 \\
 \ty' = -x \\ \ty'' = -1\\
 \ty''' = \ty^{(4)} = 0
\end{bmatrix}\! \!=\! 0
$$
$$
 \int\limits_{0}^{1}{2 \ty' h' \ \mathrm{d} x} = \left|\ \begin{gathered}
  \text{Візьмемо частинами:}\\
  v' = h' \qquad u = 2\ty'\\
  v = h  \qquad u' = 2 \ty''
 \end{gathered} \ \right| =  \underbrace{2 \ty' h'\bigg|_0^1}_{= 0} -  \!\int\limits_{0}^{ 1}{
 2 \ty'' h \ \mathrm{d} x =\!  \int\limits_{0}^{1}{2 h \ \mathrm{d} x}
 }
$$
$$
\oeq \! \int\limits_{0}^{1}{(h'')^2 + 2 h + (h')^2 - 2 h \mathrm{d} x} =\!  \int\limits_{0}^{1}{(h'')^2 + (h')^2 \mathrm{d} x} \geq 0
$$
 Отримали \textbf{глобальний мінімум.}
\end{enumerate}
\newpage

\subsection{Класична ізопериметрична задача.}
Нехай $a, b \in \mathbb{R}, \  a<b$ -- задані числа.
$$
F(x,y,p),\mathcal{C} (x,y , p) \in C^2  ([a,b] \times \mathbb{R}^2) \text{ -- задані.}
$$
Розглядаємо інтеграл:
$ \displaystyle
J(y) =  \int\limits_{a}^{ b}{ F(x, y(x), y'(x))} \mathrm{d} x
$ на множині:
$$
M = \left\lbrace
y \in C^1 [a,b]  \ \bigg| \  y(a) = A , y(b) = B, K(y) =  \int\limits_{a}^{b}{\mathcal{C} (x,y, y') \mathrm{d} x  = l }
 \right\rbrace,
$$
де $A, B, l \in \mathbb{R}$ -- задані числа.\par
\begin{itemize}
  \item Функції $y \in M$ називаються \textit{допустимими.}
  \item Умова $K(y) = l$ називається \textit{умовою зв'язку.}
  \item Означення слабкого локального/глобального min/max аналогічні.
\end{itemize}
\begin{defo}
 Задача пошуку слабкого локального/глобального екстремуму функціоналу $J(y)$ на $M$ називається \textbf{класичною ізопериметричною задачею.}
\end{defo}
Окрім $\delta J(y, h)$ розглянемо також:
$$
\begin{gathered}
\delta K(y,h ) = \frac{\mathrm{d}}{\mathrm{d} \alpha} K(y + \alpha h)\bigg|_{\alpha = 0} =  \int\limits_{a}^{b}{
\frac{\d \mathcal{C}}{\d y} h(x) +
\frac{\d \mathcal{C}}{\d y'} h'(x)  \mathrm{d} x
} \\
\forall h \in C^1[a,b] : h(a) = h(b) = 0
\end{gathered}
$$
\begin{defo} \! \textbf{Функцією Лагранжа} називають \!($\lambda_0, \lambda$ \!--\! множники Лагранжа):
  $$
  L(x, y(x) , y'(x), \lambda_0, \lambda) = \lambda_0 F(x, y, y') + \lambda\mathcal{C}(x, y, y')
  $$
\end{defo}
\begin{boxteo}
 Нехай $\ty (x) \in C^2[a,b]$ -- розв'язок класичної ізопериметричною задачі. Нехай при цьому $\delta K(\ty, h) \neq 0$. Тоді знайдеться такий множник Лагранжа $\lambda$, що $\forall \lambda_0 >0 \  \ty(x) $ задовольняє рівняння Ейлера на $[a,b]$:
 $$
 \frac{\d L }{\d y} - \frac{\mathrm{d}}{\mathrm{d} x} \frac{\d L}{\d y'} = 0
 $$
\end{boxteo}
\begin{proof}
 Нехай $\ty(x)$ -- слабкий локальний мінімум задачі. Зауважимо, що з умови теореми:
 $$
 \exists h_0 \in C^1 [a,b] : h_0(a) = h_0(b) = 0 \Rightarrow \delta K(\ty, h_0) \neq 0
 $$
 Для довільного $h \in C^1 [a,b] : h(a) = h(b) = 0$ розглянемо функції:
 $$
\begin{gathered}
\varphi(\alpha, \beta) = J(\ty + \alpha h + \beta h_0) \\
\psi (\alpha , \beta) = K(\ty + \alpha h + \beta h_0)
\end{gathered} \quad \forall \alpha, \beta \in \mathbb{R}
 $$
 Тоді $\varphi(0, 0) = J(\ty) ; \ \psi(0, 0) = K(\ty).$
 $$
 \frac{\d \varphi}{\d \alpha} \bigg|_{\alpha = \beta = 0} = \delta J(\ty, h); \qquad   \frac{\d \varphi}{\d \beta} \bigg|_{\alpha = \beta = 0} = \delta J(\ty, h_0);
 $$
 $$
  \frac{\d \psi}{\d \alpha} \bigg|_{\alpha = \beta = 0} = \delta K(\ty, h); \qquad   \frac{\d \psi}{\d \beta} \bigg|_{\alpha = \beta = 0} = \delta K(\ty, h_0);
 $$
 Припустимо, що Якобіан:
 $$
 \frac{\d (\varphi, \psi)}{ \d (\alpha , \beta)}  \bigg|_{\alpha = \beta = 0}  \equiv 0 \quad \forall h \in C^1[a,b] : h(a) = h(b) = 0
 $$
 Якщо це так, то:
$$
0 = \frac{\d (\varphi, \psi)}{ \d (\alpha , \beta)}  \bigg|_{\alpha = \beta = 0} = \begin{vmatrix}
 \frac{\d \varphi}{\mathrm{d} \alpha} & \frac{\d \varphi}{ \d \beta}\\
 \frac{\d \psi}{\mathrm{d} \alpha} & \frac{\d \psi}{ \d \beta}
\end{vmatrix} = \begin{vmatrix}
 \delta J(\ty, h) & \delta J(\ty, h_0)\\
 \delta K(\ty, h) & \delta K(\ty, h_0)
\end{vmatrix} =
$$
$$
= \delta J(\ty, h) \cdot \delta K(\ty, h_0) - \delta J(\ty, h_0) \cdot \delta K(\ty, h) = 0
$$
Оскільки $\delta K (\ty, h_0) \neq  0$, то поділимо на $\delta K (\ty, h_0)$:
$$
\delta J(\ty, h) - \frac{\delta J(\ty, h_0)}{\delta K(\ty, h_0)} \delta K(\ty, h) = 0
$$
Покладемо $\lambda_0 = 1, \lambda = - \dfrac{\delta J(\ty, h_0)}{\delta K(\ty, h_0)}$. Якщо $\lambda_0 \neq 1, \lambda_0 > 0$, то $\lambda = - \lambda_0 \dfrac{\delta J(\ty, h_0)}{\delta K(\ty, h_0)}$:
$$
\lambda_0 \delta J(\ty, h) + \lambda \delta K(\ty, h) = 0 \qquad \forall h \in C^1[a,b]: h(a) = h(b) = 0
$$
$$
0 = \lambda_0  \int\limits_{a}^{b}{
\frac{\d F}{\d y} \cdot h + \underbrace{\frac{\d F}{\d y'} \cdot h'}_{\text{частинами}} \mathrm{d} x
}
 + \lambda  \int\limits_{a}^{b}{ \frac{\d \mathcal{C}}{\d y}\cdot h + \underbrace{\frac{\d \mathcal{C}}{\d y' }\cdot h'}_{\text{частинами}} \mathrm{d} x } =
$$
$$
= \lambda_0 \int\limits_{a}^{b}{
\left(
\frac{\d F}{\d y} - \frac{\mathrm{d}}{\mathrm{d} x} \frac{\d F}{\d y'} \right) h \mathrm{d} x
} + \lambda \int\limits_{a}^{b}{
\left(
\frac{\d \mathcal{C}}{\d y} - \frac{\mathrm{d}}{\mathrm{d} x} \frac{\d \mathcal{C}}{\d y'} \right) h \  \mathrm{d} x
} =
$$
$$
=  \int\limits_{a}^{b}{
\left[
\left( \lambda_0 \frac{\d F}{\d y}  + \lambda \frac{\d \mathcal{C}}{\d y} \right)
- \frac{\mathrm{d}}{\mathrm{d} x}
\left(
 \lambda_0 \frac{\d F}{\d y'} + \lambda \frac{\d \mathcal{C}}{\d y'}
 \right)
\right] \cdot h \  \mathrm{d} x
} =
$$
$$
=  \int\limits_{a}^{b}{
\left( \frac{\d L}{\d y} - \frac{\mathrm{d}}{\mathrm{d} x} \frac{\d L}{\d y'}\right)\cdot h \  \mathrm{d} x
}  = 0 \xRightarrow{\text{Лема Лагранжа}} \text{ рівняння Ейлера для } L.
$$
\end{proof}
