\section{Лекція 5}
\subsection{Перші інтеграли систем диференційних рівнянь.}
\begin{example}
 Розглянемо рівняння
 $y' = y \circledast \ \Longrightarrow \  y =  C \cdot e^x $ - заг. розв.\\
 Легко побачити, що функція:
 $$
 \psi (x,y)  = y \cdot e^{-x}
 $$
 Має таку властивість:
 $$
 \forall y(x) \text{ - розв. } \ \frac{\d}{\d x} \psi(x, y(x)) = (y(x)\cdot e^{-x})' = y' \cdot e^{-x}  = y \cdot e^{-x} = 0
 $$
 Таким чином, $\forall y(x) $ - розв'язок $\psi (x, y(x)) = const$. Інакше кажучи, $\psi(x,y) $ є сталою вздовд довільного розв'язку рівняння $\circledast$. \\Такі функції називають \textbf{першими інтегралами}.
\end{example}
\begin{defo}
Розглянемо систему:
  \be\label{spd3}
  \overrightarrow{x}' = \overrightarrow{f} (t, \overrightarrow{x}),
  \ee
  де $f: D \to \mathbb{R}^n ; D \subset \mathbb{R}^{n+1}; f \in C(D)$.\\
  Функція $U(t, \overrightarrow{x}) \in C^1 (D)$ називається \textbf{першим інтегралом} системи \eqref{spd3}, якщо для кожного розв'язку системи \eqref{spd3} $\overrightarrow{x} = \overrightarrow{x}(t), t \in I$ виконується:
$$
\frac{d}{d t} U ( t, \overrightarrow{x}(t)) = 0 \quad \forall t \in I
$$
(тобто $U$ є сталою вздовж довільного розв'язку системи \eqref{spd3})
\end{defo}
\begin{example}
  $$
  \begin{dcases}
   \dot{x} = y^2 + z^2\\
   \dot{y} = z \\
   \dot{z} = -y
  \end{dcases}
  $$
  Розглядаємо функцію $U(t,x,y,z) = y^2 + z^2$.\\
  Тоді для кожного розв'язку системи $\begin{bmatrix}
   x(t)&
   y(t)&
   z(t)
  \end{bmatrix}^T$ маємо:
  $$
  \frac{d}{dt} U(t, x(t), y(t), z(t)) = \frac{\d U}{\d t} + \frac{\d U}{\d x}\cdot \frac{\d x}{\d t}  + \frac{\d U}{\d y}\cdot \frac{\d y}{\d t} +
  \frac{\d U}{\d z}\cdot \frac{\d z}{\d t} = 2yz - 2zy = 0
  $$
  Таким чином, $U(t,x,y,z)= y^2 + z^2$ - І інтеграл системи.
\end{example}
\begin{remark}
    Похідною функції $U\in C^1(D) $ є функція:
    $$
    \dot{U}_f (t, \overrightarrow{x}) = \frac{\d U}{\d t} +  \sum\limits_{i = 1}^{n}{ \frac{\d U}{\d x_i} \cdot f_i(t, \overrightarrow{x})  }
    $$
\end{remark}
\begin{boxteo}[Аналітичний критерій першого інтегралу]
  $U\in C^1(D)$ є першим інтегралом системи \eqref{spd3} т.т.т.к.:
  $$
  \forall (t,x) \in D \ \ \dot{U}_f (t,x) = 0
  $$
\end{boxteo}
\begin{proof}
 \fbox{$\Longleftarrow$} $\dot{U}_f (t,x) \Rightarrow$ Для довільного розв'язку $\overrightarrow{x}(t), t\in I$.
 $$
 0 = \frac{\d U}{\d t} +  \sum\limits_{i = 1}^{ n}{ \frac{\d U}{\d x_i}  } \cdot f_i (t, \overrightarrow{x}) = \frac{\d U}{\d t} +  \sum\limits_{i = 1}^{ n}{ \frac{\d U}{\d x_i}  } \cdot \frac{d x_i}{dt} = \frac{d}{dt} U(t, \overrightarrow{x}(t)) \  \Rightarrow
 $$
 $$
\Rightarrow \ \frac{d}{dt} U(t, \overrightarrow{x}(t)) = 0 \ \Rightarrow \ U \text{ - перший інтеграл.}
 $$
\fbox{$\Longrightarrow$} Нехай $U$ - І інтеграл. Тоді для кожного розв'язку виконується:
$$
0 = \frac{\d U}{\d t} +  \sum\limits_{i = 1}^{ n}{ \frac{\d U}{\d x_i} } \cdot \frac{d x_i}{dt} =  \frac{\d U}{\d t} +  \sum\limits_{i = 1}^{ n}{ \frac{\d U}{\d x_i} } \cdot f_i = \dot{U}_f (t, \overrightarrow{x}(t))
$$
Отже, $\forall \overrightarrow{x}(t)$ - розв'язку: $\dot{U}_f(t, \overrightarrow{x}(t)) = 0$.\\
І оскільки $\overrightarrow{f} \in C(D)$, то за т. Пеано $\forall (t,\overrightarrow{x}) \in D$ знайдеться розв'язок, що проходить через цю точку, а отже: $\forall (t, \overrightarrow{x}) \in D \ \  \dot{U}_f(t, \overrightarrow{x})=0$.
\end{proof}

\begin{remark}
    З прикладу бачимо, що для диф. рівняння першого порядку пошук першого інтегралу, фактично, завершує розв'язання цього рівняння.
\end{remark}
\textbf{Питання.} скільки потрібно знайти перших інтегралів системи  \eqref{spd3}, щоб вважати її розв'язаною?
\begin{remark}
    Якщо $U$ - перший інтеграл системи, то $\forall \varphi \in C(D)$:
    $$
    \varphi(U) \ \text{ - також перший інтеграл.}
    $$
\end{remark}
\def\rank{\text{rank}}
\begin{defo}
  Перші інтегралии назвемо функціонально незалежними, якщо ранг матриці Якобі для цих перших інтегралів $ \left\lbrace U_i \right\rbrace_{i=1}^{k}$ дорівнює їх кількості.
  $$
  \rank{\begin{bmatrix}
   \frac{\d U_1}{\d x_1} & \cdots & \frac{\d U_1}{\d x_n} \\
   \frac{\d U_2}{\d x_1} & \cdots & \frac{\d U_2}{\d x_n} \\
   \vdots & \ddots & \vdots\\
   \frac{\d U_k}{\d x_1} & \cdots & \frac{\d U_k}{\d x_n}
  \end{bmatrix}} = k
  $$
\end{defo}
\begin{defo}
 Набір з n функціонально незалежних перших інтегралів називається повним набором перших інтегралів.
\end{defo}
\begin{boxteo}[про існування повного набору перших інтегралів] \ \\
Нехай $\overrightarrow{f} \in C^1(D)$. Тоді $\forall (t, \overrightarrow{t}) \in D$ знайдеться окіл цієї точки, в якому існує повний набір.
\end{boxteo}
\begin{boxteo}[про загальний розв'язок системи \eqref{spd3}]

  Нехай $\overrightarrow{f} \in C^1(D)$;  \ \\ $\left\lbrace U_i \right\rbrace_{i=1}^n$ - повний набір перших інтегралів системи \eqref{spd3}. Тоді співвідношення:
  $$
\left\lbrace U_i (x,y)  = c_i \right\rbrace^n_{i=1}
  $$
  задають загальний розв'язок системи \eqref{spd3}.
\end{boxteo}
\subsection{Методи розв'язання систем n-го порядку.}
\subsubsection{Метод виключення.} (Метод зведення до рівняння n-го порядку відносно однієї зі змінних)
\begin{example}
$$ \begin{dcases}
 \frac{dy}{dx} = \frac{y^2}{z-x}\\
 \frac{dz}{dx} = y+1
 \end{dcases}$$
 Продиференціюємо перше рівняння:
 $$
 y'' = \frac{2yy'(z-x) - (z' - 1)y^2}{(z-x)^2}
 $$
\end{example}
 З першого рівняння: $z-x = \dfrac{y^2}{y'} $.\\
З другого рівняння: $z' - 1 = y$.\\
Підставимо:
$$
y'' = \frac{2yy' \cdot \frac{y^2}{y'} -y^3}{ \frac{y^4}{(y')^{2}} }
$$
$$
y'' = \frac{y^3}{y^4} \cdot (y')^{2} \ \Longrightarrow \  y'' = \frac{(y')^{2}}{y}
$$
Зауважимо, що ми втратили розв'язок $y=0$, поділивши на $y$. Отримали:\\ $y'' = \frac{(y')^2}{y} $ \text{- рівняння , що не залежить від} $x$.\\
Заміна: $y$ - нова незалежна змінна:
$$
y' = p, p = p(y) \text{ - нова невід'ємна функція.}
$$
$$
y'' = \frac{dy'}{dx} = \frac{dy'}{dy} \cdot \frac{dy}{dx} = p' \cdot p
$$
$$
p'p = \frac{p}{y} \quad p' = \frac{p}{y} \quad \frac{dp}{dy} = \frac{p}{y} \quad \frac{dp}{p} = \frac{dy}{y}
$$
$$
\ln \left| p \right| = \ln |y| + \ln |c_1| \ \Longrightarrow \  p = c_1 \cdot y
$$
$$
y' = c_1 y \ \Longrightarrow \  \frac{dy}{y} = c_1 \ \Longrightarrow \ \ln|y| = c_1 x + \ln|c_2| \ \Longrightarrow \  \fbox{$y = c_2 \cdot e^{c_1 x}$}
$$
Знайдемо z:
$$
z - x = \frac{y^2}{y'} = \frac{c_2^2 e^{2c_1 x}}{c_1 c_2 e^{c_1 x}} = \frac{c_2}{c_1} e^{c_1 x}
$$
$$
z  = \frac{c_2}{c_1} e^{c_1 x}   +x
$$
Остаточна відповідь:
$$
\left[
\begin{array}{l}
\begin{dcases}
 y = c_2 e^{c_1 x}\\
 z = \frac{c_2}{c_1} e^{c_1 x} + x
\end{dcases}
\\
\begin{dcases}
 y = 0\\
 z = x + c
\end{dcases}
\end{array}
 \right.
$$

\subsubsection{Метод інтегрованих комбінацій.}
