
Розглянемо \( \delta^2 J(y, h): \)
\[
 \begin{split}
 \delta^2 J(y, h) &= \frac{\mathrm{d}^2 J(y + \alpha h)}{\mathrm{d} \alpha^2} \bigg|_{ \alpha = 0} = \frac{\mathrm{d}^2}{\mathrm{d} \alpha^2}   \int\limits_{a}^{b}{
 F(x,y + \alpha h, y' + \alpha h') \mathrm{d} x \bigg|_{\alpha} = 0 =
 } \\
 &= \! \frac{\mathrm{d}}{\mathrm{d} \alpha}\!
 \int\limits_{a}^{b}{\!\!
 \left(
\frac{\d F}{\d y} (x, y + \alpha h, y' + \alpha h') h + \frac{\d F}{\d y'} (x, y + \alpha h, y' + \alpha h')  h'
  \right) \mathrm{d} x \bigg|_{\alpha = 0}\!\!\!\!
 }= \\
 &=  \int\limits_{a}^{b}{
 \left(
\frac{\d^2 F}{\d y ^2 } (x, y, \ty) \cdot h^2 + 2 \frac{\d F}{\d y \d \ty} h h' + \frac{\d^2 F}{\d (y')^2} \cdot (h')^2
  \right) \mathrm{d} x
 } \oeq
 \end{split}
\]
Проінтегруємо другий доданок частинами:
\[
\begin{split}
\int\limits_{a}^{b}{
\frac{\d^2 F}{\d y \d y'} 2 h h' \mathrm{d} x
}  = [2hh' = (h^2)'] =  \int\limits_{a}^{b}{ \frac{\d^2 F}{\d y\d y'} \cdot (h^2)' \mathrm{d} x} = \\
= \left|
\begin{gathered}
u = \frac{\d^2 F}{\d y \d y'} \\
u' = \frac{\mathrm{d}}{\mathrm{d} x} \left(  \frac{\d^2 F}{\d y \d y'} \right)
\end{gathered} \
\begin{gathered}

\end{gathered}
 \right| = \underbrace{\frac{\d^2 F}{\d y \d y'} \cdot h^2 \bigg|_{a}^{b}}_{=0} -  \int\limits_{a}^{b}{
 \frac{\mathrm{d}}{\mathrm{d} x} \left( \frac{\d^2 F}{\d y \d y'} \right) \cdot h^2
 \mathrm{d} x}
\end{split}
\]

\[
  \Rightarrow \delta^2 J(y, h) =  \int\limits_{a}^{b}{
  \left[
\frac{\d^2 F}{\d (y')^2} \cdot (h')^2 + \left(
\frac{\d^2 F}{\d y^2} - \frac{\mathrm{d}}{\mathrm{d} x} \frac{\d^2 F}{\d y \d y'}
\right)\cdot  h^2
   \right] \mathrm{d} x
  }
\]


Маємо: \( \delta^2 J(y, h) =  \int\limits_{a}^{b}{
\underbrace{P(x) \cdot (h')^2 + Q(x)\cdot h^2}_{= \mathcal{C}}
  \mathrm{d} x
} \). \par
Запишемо рівняння Ейлера для \(  \delta^2 J(y, h) \) відносно \( h \):
\[
 \frac{\d \mathcal{C}}{h} - \frac{\mathrm{d}}{\mathrm{d} x} \frac{\d \mathcal{C}}{\d h'} = 0
\]
\[
2 h Q(x) - \frac{\mathrm{d}}{\mathrm{d} x } \left( 2 P'(x) \cdot h' \right) = 0
\]
\begin{center}
 \fbox{\( \displaystyle \frac{\mathrm{d}}{\mathrm{d} x} \left( P(x) \cdot h' \right) =
 h \cdot Q(x)
  \)}  \textbf{ --- рівняння Якобі.}
\end{center}
\[
 \frac{\mathrm{d}}{\mathrm{d} x}  \left(  \frac{\d^2 F}{\d (y')^2} \cdot h' \right) = h \cdot
 \frac{\d^2 F}{\d y^2}  - h \cdot \frac{\mathrm{d}}{\mathrm{d} x} \frac{\d^2 F}{\d y \d y'}
\]
\[
\frac{\mathrm{d}}{\mathrm{d} x}  \left(  \frac{\d^2 F}{\d (y')^2} \cdot h' \right) = h \cdot
\frac{\d^2 F}{\d y^2}  - h \cdot \frac{\mathrm{d}}{\mathrm{d} x} \frac{\d^2 F}{\d y \d y'} + h' \cdot \frac{\d^2 F}{\d y \d y'} -  h' \cdot \frac{\d^2 F}{\d y \d y'}
\]
\[
 \Rightarrow \frac{\mathrm{d}}{\mathrm{d} x}   \left(  \frac{\d^2 F}{\d (y')^2} \cdot h' \right) =   h \cdot
 \frac{\d^2 F}{\d y^2}  -
 \frac{\mathrm{d}}{\mathrm{d} x} \left( \frac{\d^2 F}{\d y \d y'} \cdot h \right) +
 h' \cdot \frac{\d^2 F}{\d y \d y'}
\]
\begin{center}
\fbox{\( \displaystyle
 \frac{\mathrm{d} }{\mathrm{d} x} \left( \frac{\d^2 F}{ \d (y')^2} \cdot h' + \frac{\d^2 F}{\d y \d y'} \cdot h \right) = h \frac{\d^2}{\d y^2} + h' \frac{\d^2 F}{\d y \d y'} \)}
\end{center}
\begin{defo}
 Назвемо т. \( \mathcal{C} \in (a,b] \) \textit{спряженою} з точкою \( a \), якщо
 існує розв'язок рівняння Якобі: \( h(x) \neq 0 \) такий, що \( h(a) = h(\mathcal{C}) = 0 \).
\end{defo}

\begin{boxteo}
 Якщо \( \ty (x) \in M \) забезпечує слабкий локальний екстремум функціоналу \( J(y) \) на
 \(  M \), то виконується \textbf{умова Якобі}: на \( (a,b) \) немає  точок, спряжених з точкою \( a \). \par
 Без доведення. \( \blacksquare \)
\end{boxteo}
\newpage
\begin{boxteo}[про достатні умови]
  Нехай для \( \ty(x) \in M \) виконується:
  \begin{enumerate}
    \item Рівняння Ейлера на \( [a,b] \).
    \item Посилена умова Лежандра:
    \[
     \frac{\d^2 F}{\d (y')^2} > 0 \ \ (< 0)
    \]
    \item Посилена умова Якобі: на \( (a,b] \) немає точок спряжених з точкою \( a \).\par
    Тоді \( \ty (x) \) забезпечує слабкий локальний мінімум (максимум) функціоналу \( J(y) \) на \( M \).\par
    Без доведення. \( \blacksquare \)
  \end{enumerate}
\end{boxteo}
\subsection{Сильний локальний екстремум.}
Нехай \( a,b \in \mathbb{R}, a< b  \) -- задані числа.
\[
 F(x,y,p) \in C^2 \left( [a,b] \times \mathbb{R}^2 \right)
\]
Розглянемо простір \( KC^1[a,b] \) -- простір кусково-гладких функцій на \( [a,b] \) (неперервних функцій на \( [a,b] \), похідна яких має не більш, ніж скінчену кількість особливих точок І-го роду).\par
\( KC^1[a,b] \) -- нормований простір з нормою:
\[
 ||y(x)||_{KC^1[a,b]} =  \max\limits_{x\in[a,b]}{|y(x)|}
\]
Розглянемо інтеграл:
\[
 J(y) =  \int\limits_{a}^{b}{
 F(x, y(x), y'(x)) \mathrm{d} x
 }
\]
на множині \( M = \left\lbrace
y(x) \in KC^1 [a,b] \ \big| \  y(a) = A, y(b) = B
 \right\rbrace \)
\begin{defo}
 Функція \( \ty (x) \in M \) забезпечує \textit{сильний локальний екстремум} функціоналу \( J(y) \) на \( M \), якщо:
 \[
  \exists \delta>0 : \forall y \in M \ \  ||y - \ty||_{KC^1[a,b]} < \delta \text{ справдовується, що:}
 \]
 \[
  J(y) \geq J(\ty) \text{ (для мінімуму)}
 \]
 \[
  J(y) \leq  J(\ty) \text{ (для максимуму)}
 \]
 --- \textit{сильний глобальний} екстремум, якщо нерівність виконується \( \forall y \in M \).
\end{defo}
\begin{lema}[про зглажування кутів]
\[
\inf_{
\substack{y\in KC^1[a,b] \\
y(a) = A \\
y(b) = B}
}{J(y)} =
\inf_{
\substack{y\in C^1[a,b] \\
y(a) = A \\
y(b) = B}
}{J(y)}
\]
\[\sup_{
\substack{y\in KC^1[a,b] \\
y(a) = A \\
y(b) = B}
}{J(y)} =
\sup_{
\substack{y\in C^1[a,b] \\
y(a) = A \\
y(b) = B}
}{J(y)}
\]
\end{lema}

\begin{consequence}
  Якщо слабкий глобальний мінімум (максимум) досягається, то він дорівнює сильному глобальному мінімуму (максимуму) і функція, яка забезпечує слабкий глобальний екстремум, забезпечує водночас і сильний глобальний екстремум.
\end{consequence}
\begin{boxteo}[Вейєрштрасса]
  Нехай допустима функція:
  \[
   \ty (x) \in C^1[a,b] \ \text{ задовольняє:}
  \]
  \begin{enumerate}
    \item Рівняння Ейлера.
    \item Посилену умову Лежандра:
    \[
     \frac{\d^2 F}{\d (y')^2} > 0 \text{ -- для мінімума.} \qquad
     \frac{\d^2 F}{\d (y')^2} < 0 \text{ -- для максимума.}
    \]
    \item Посилену умову Якобі.
    \item Умову Вейєрштрасса:
    \( \
     \forall (x,y,u) , (x,y,v)  \text{ функція Вейєрштрасса:}
    \)
    \[
     E(x,y,u,v) = F(x,y,v) - F(x,y,u) - (v-u) \frac{\d F}{\d u}  \underset{\text{для min}}{ (x,y,u) \geq  0}
    \]
    \[
      E(x,y,u,v) \leq 0\text{ --- для максимума.}
    \]
    Тоді \( \ty(x) \) забезпечує водночас і слабкий, і сильний локальний мінімум (максимум) функціоналу \( J(y) \) на \( M \).
  \end{enumerate}
\end{boxteo}
\begin{example}
\[
 \begin{dcases}
    \int\limits_{0}^{1}{ (y'(x))^2 + xy(x) \mathrm{d} x} \to \mathrm{extr};\\
    y(0) = 0 \qquad y(1) = 0.\\
 \end{dcases}
\]
\begin{enumerate}
  \item Складемо і розв'яжемо рівняння Ейлера:
  \[
   \frac{\d F}{\d y} - \frac{\mathrm{d}}{\mathrm{d} x} \frac{\d F}{\d y'} = 0
  \]
  \[
   x - 2y'' = 0 \quad \Longrightarrow \quad y'' = \frac{x}{2} \qquad y' = \frac{x^2}{4} + C_1
  \]
  \[
   y(x) = \frac{x^3}{12} + C_1 x + C_2 \text{ -- сімейство екстремалей.}
  \]
  \item Знайдемо допустимі екстремалі:
  \[
   y(0) = 0 \quad \Longrightarrow \quad C_2 = 0 \quad \Longrightarrow \quad y(1) = 0
  \]
  \[
   \frac{1}{12} + C_1 = 0 \quad \Longrightarrow \quad C_1 = - \frac{1}{12}
  \]
  \[
   \ty(x) = \frac{x^3}{12} - \frac{x}{12} \text{ --- єдина допустима екстремаль.}
  \]
  \item Запишемо умову Лежандра:
  \[
   \frac{\d^2 F}{\d (y')^2} = \frac{\d}{\d y'} (2y') = 2 > 0 \Longrightarrow \text{виконується.}
  \]
  Посилена умова Лежандра для мінімума.
  \item Запишемо умову Якобі. Рівняння Якобі:
  \[
   \frac{\mathrm{d}}{\mathrm{d} x} \left( \frac{\d^2 F}{\d (y')^2} h' + \frac{\d^2 F}{\d y \d y'} h \right) = h \frac{\d^2 F}{\d y^2} + h' \frac{\d^2 F}{\d y \d y'} \qquad \quad F = (y')^2 + xy
  \]
  \[
   \frac{\d F}{\d y'} = 2 y' \qquad
   \frac{\d F}{\d y} = x \qquad
   \frac{\d^2 F}{\d (y')^2} = 2\qquad
   \frac{\d^2 F}{\d y \d y'} = 0 \qquad
   \frac{\d^2 F}{\d y^2}= 0
  \]
  \[
   \frac{\mathrm{d}}{\mathrm{d} x } (2h') =0  \quad \Longrightarrow \quad h'' = 0 \text{ -- рівняння Якобі} \quad \Longrightarrow \quad h(x) = C_1 x + C_2
  \]
  \( h(0) = 0 \Longrightarrow h(x) = C_1 x \) --- не перетворюється в нуль в жодній точці або є тотожним нулем, тобто виконується посилена умова Якобі.
  \item Запишемо умову Вейєрштрасса:
  \[
   E(x,y,u,v) = F(x,y,v) - F(x,y,u)  - (v-u) \frac{\d F}{\d u} (x,y,u) =
  \]
  \[
   = \left| F(x,y,u)  = u^2 + xy \qquad \frac{\d F}{\d u} = 2 u \right| = v^2 + xy - u^2 - xy - (v-u) \cdot 2u =
  \]
  \[
   v^2 - u^2 - 2uv + 2 u^2 = v^2 -2 uv + u^2 = (v-u)^2  \geq 0
  \]
  Тобто, виконується умова Вейєрштрасса для мінімума. \par
  \( \displaystyle \ty (x) = \frac{x^3}{12} - \frac{x}{12} \) забезпечує водночас і сильний і слабкий лок. мінімум.
\end{enumerate}
\end{example}
\begin{example}
 \[
  \begin{dcases}
    \int\limits_{1}^{e}
    x(y')^2 + yy' \mathrm{d} x \to \mathrm{extr};\\
    y(1) = 0 \qquad y(e) = 1.
  \end{dcases}
 \]
\begin{enumerate}
  \item Складемо і розв'яжемо рівняння Ейлера:
  \[
   \frac{\d F}{\d y} - \frac{\mathrm{d}}{\mathrm{d} x} \frac{\d F}{\d y'} = 0
  \]
  \[
   y' - \frac{\mathrm{d}}{\mathrm{d} x} (2 xy' + y) = 0
  \]
  \[
   y' - 2 y' - 2 xy'' - y' = 0
  \]
  \[
    xy'' + y' = 0 \qquad y' = z , y'' = z' \qquad xz' + z = 0
  \]
  \[
   z' = - \frac{z}{x} \quad \Longrightarrow \quad \frac{\mathrm{d} z}{z} = -\frac{ \mathrm{d} x}{x}
  \]
  \[
   \ln |z| = \ln \left| \frac{1}{x} \right| + \ln |C_1|\qquad z = \frac{C_1}{x} \qquad y' = \frac{C_1}{x}
  \]
  \[
  y = C_1 \ln x + C_2 \text{ --- сімейство екстремалей.}
  \]
  \item Знайдемо допустимі екстремалі:
  \[
\begin{split}
y(1) = 0 &\quad \Longrightarrow \quad C_1 \cdot 0 + C_2 = 0 \quad \Longrightarrow \quad C_2 = 0 \\
y(e) = 1 &\quad \Longrightarrow \quad C_1 = 1   \\
& \ty = \ln x \text{ -- єдина допустима екстремаль.}
\end{split}
  \]
\item Запишемо умову Лежандра:
\[
 \frac{\d^2 F}{\d (y')^2} = \frac{\d}{\d y'} (2 xy' + y) = 2 x >0 \qquad \forall x \in [1, e]
\]
виконується посилена умова Лежандра для мінімума.
\item Запишемо умову Якобі:
\[
 \frac{\mathrm{d}}{\mathrm{d} x} \left( \frac{\d^2 F}{\d (y')^2} h' + \frac{\d^2 F}{\d y \d y'} h \right)  = h \frac{\d^2 F}{\d y^2} + h' \frac{\d^2 F}{\d y \d y'}
\]
\[
 F = x(y')^2 + yy'
\]
\[
 \frac{\d^2 F}{\d (y')^2} = 2x \qquad \frac{\d^2 F}{\d y \d y'} = \frac{\d}{\d y} (2xy' + y) = 1 \qquad \frac{\d^2 F}{\d y^2}  = 0
\]
\[
\begin{split}
\frac{\mathrm{d}}{\mathrm{d} x} (2xy' + h) = h' &\qquad   \frac{\mathrm{d}}{\mathrm{d} x} (2xy' + h) = \frac{\mathrm{d}}{\mathrm{d}x}(h)\\
\frac{\mathrm{d}}{\mathrm{d} x} (2 xh' + h - h) = 0 &\qquad \frac{\mathrm{d}}{\mathrm{d} x} (2xh') \quad \Longrightarrow \quad \frac{\mathrm{d}}{\mathrm{d}x} (xh') = 0
\end{split}
\]
\[
  xh' = C_1 \quad \Longrightarrow \quad h' = \frac{C_1}{x} \quad \Longrightarrow\quad h(x) = C_1 \ln x + C_2
\]
Якщо \( h(1) = 0 \  \Longrightarrow \  C_2 = 0 \  \Longrightarrow \  f(x) = C_1 \ln x \) --- не перетворюється в нуль в жодній точці \( x \in (1,e] \) або є тотожним нулем, тобто виконується посилена умова Якобі.
\item Запишемо умову Вейєрштрасса:
\[
 E = F(x,y,v) - F(x,y,u) - (v-u) \frac{\d F}{\d u} (x,y,u)  =
\]
\[
= \left| F(x,y,u) = xu^2 + yu ; \frac{\d F}{\d u} = 2xu + y \right| =
\]
\[
 = xv^2 + yv - xu^2 - yu - (v-u)(2xu + y) =
\]
\[
= xv^2 + yv - xu^2 - yu - 2xuv - yv + 2xu^2 + yu =
\]
\[
 = xv^2 - 2xuv  + xu^2 = x(u-v)^2 \geq 0 \qquad \forall x \in [1, e]
\]
Таким чином, умова Вейєрштрасса виконується для мінімума.\par
Отже, \( \ty (x) = \ln x \) забезпечує і сильний, і слабкий лок. мінімум.
\end{enumerate}
\end{example}
\begin{example}
 \[
\begin{dcases}
\inf\limits_{1}^{e} (2y - x^2 (y')^2) \mathrm{d} x \to \mathrm{extr}; \\
y(1) = e \qquad y(e) = 0 .
\end{dcases} \]
\begin{enumerate}
  \item Запишемо та розв'яжемо рівняння Ейлера:
  \[
   \frac{\d F}{\d y} - \frac{\mathrm{d}}{\mathrm{d} x} \frac{\d F}{\d y'} = 0
  \]
  \[
   2 + \frac{\mathrm{d}}{\mathrm{d} x} (x^2 \cdot 2 y') = 0
  \]
  \[
   2 + 2 x \cdot 2 y' + x^2 \cdot 2 y'' = 0
  \]
  \[
   1 + 2 x y' + x^2 y'' = 0
  \]
  \[
   x^2 y'' + 2 x y' = -1 \text{ --- рівняння Ейлера.} \qquad x = e^t
  \]
  \[
   \frac{\mathrm{d} y}{\mathrm{d} x} = \frac{\mathrm{d} y}{ \mathrm{d} t} \cdot \frac{\mathrm{d} t}{ \mathrm{d} x} = \frac{\mathrm{d} y}{ \mathrm{d} t}  \bigg/ \frac{\mathrm{d} x}{ \mathrm{d} t} = y'_t \cdot e^{-t}
  \]
  \[
   \frac{\d^2 y}{ \d x^2} = \frac{\mathrm{d}}{\mathrm{d} x} \left( \frac{\mathrm{d} y}{\mathrm{d} x} \right) = \frac{\mathrm{d}}{\mathrm{d} t} \left( \frac{\mathrm{d} y}{\mathrm{d} x} \right)
   \bigg/ \frac{\mathrm{d} x}{ \mathrm{d} t} =
  \]
  \[
   = (y''_t \cdot e^{-t} - y'_t \cdot e^{-t}) \cdot e^{-t} = (y''_t - y'_t) \cdot e^{-2t}
  \]
  \[
   (y''_t - y'_t) \cdot e^{-2t} \cdot e^{2t} + 2y'_t e^t e^{-t} = -1
  \]
  \[
   y''_t + y'_t  = 1 \qquad \Longrightarrow \qquad \begin{gathered}
    \lambda^2 + \lambda = 0\\
    \lambda(\lambda+ 1)= 0\\
    \lambda  = 0 \quad \lambda = -1
   \end{gathered}
  \]
  \[
   y_o = C_1 + C_2 e^{-t} \qquad y_{p} = A x , y_{p}' = A, y_{p}'' = 0, A = 1:
  \]
  \[
   y_p = -t \quad \Longrightarrow \quad y(t) = C_1 + C_2 e^{-t} - t
  \]
  \[
   y(x) = C_1 + \frac{C_2}{x}  - \ln x \textit{ --- сімейство екстремалей.}
  \]
  \item Знайдемо допустимі екстремалі:
  \[
\begin{split}
y(1) = e &\qquad  C_1 + C_2 = e \\
y(e) = 0 &\qquad C_1 + \frac{C_2}{e} -1 = 0
\end{split}
  \]
  \[
   C_2 - \frac{C_2}{e} + 1 = e \qquad C_2 \left( \frac{e-1}{e} \right) = e - 1 \quad \Longrightarrow \quad
   \begin{cases}
    C_2 = e \\ C_1 = 0
   \end{cases}
  \]
  \(
   \ty (x) = \frac{e}{x} - \ln x \text{ --- єдина допустима екстремаль.}
  \)
  \item Запишемо умови Лежандра:
  \[
   F = 2y - x^2 (y')^2 \qquad \frac{\d F}{\d y'} = -2y' x^2
  \]
  \[
   \frac{\d^2 F}{\d (y')^2} = -2 x^2 < 0 \qquad \forall x \in [1, e]
  \]
  Тобто, посилена умова Лежандра виконується.
  \item Запишемо умову Якобі:
  \[
   \frac{\mathrm{d}}{\mathrm{d} x} \left(  \frac{\d^2 F}{\d (y')^2} h' + \frac{\d^2 F}{\d y \d y'} h \right) = h \frac{\d^2 F}{\d y^2} + h' \frac{\d^2 F}{\d y \d y'}
  \]
  \[
   \frac{\d^2 F}{\d (y')^2} = -2 x^2 \qquad \frac{\d^2 F}{\d y \d y'} = 0 \qquad \frac{\d^2 F}{\d y^2} = 0
  \]
  \[
    \frac{\mathrm{d}}{\mathrm{d} x} (-2 x^2 h') = 0 \quad \Longrightarrow \quad \frac{\mathrm{d} }{\mathrm{d} x} (x^2 h') = 0 \quad \Longrightarrow \quad x^2 h' = C_1
  \]
  \[
    h' = \frac{C_1}{x^2} \quad \Longrightarrow \quad h = - \frac{C_1}{x} + C_2
  \]
  \[
   h(1) = 0 \quad \Longrightarrow \quad \frac{- C_1}{1} + C_2 = 0 \quad \Longrightarrow \quad C_2 = C_1 \quad \Longrightarrow \quad h = C_1 \cdot (1 - \frac{1}{x})
  \]
  Вираз не перетворюється на нуль \( \forall x \in [1, e] \) або є тотожним нулем, тому виконується посилена умова Якобі.
  \item Запишемо умову Вейєрштрасса:
  \[
   E = F(x,y,v) - F(x,y,u) - (v-u) \frac{\d F}{\d u} (x,y,u)  =
  \]
  \[
   = \left| F(x,y,u) = 2 y - x^2 u^2 \quad \frac{\d F}{\d u} = -2ux^2 \right| =
  \]
  \[
   = 2 y - x^2 v^2 - 2 y + x^2 u^2 - (v-u)\cdot(-2ux^2) = -x^2 v^2 + x^2 u^2 + 2uvx^2 - 2 u^2 x^2 =
  \]
  \[
   = -x^2 (v^2 - 2 uv + u^2) = -x^2 (v-u)^2 \leq 0
  \]
  Тобто, виконується умова Вейєрштрасса для максимума. \par
  Отже, \( \ty (x) = \frac{e}{x} - \ln x \) забезпечує водночас і сильний, і слабкий лок. max.
\end{enumerate}
\end{example}
