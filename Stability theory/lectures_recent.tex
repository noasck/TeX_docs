\begin{boxteo}
Якщо допустима функція:
$$
\tilde{y} \in C^2[a,b]
$$
є розв'язком задачі з вільним кінцем, то $\tilde{y} (x)$ задовільняє рівнянню Ейлера на $[a,b]$:
$$
\frac{\d F
}{\d y}  - \frac{\mathrm{d} }{\mathrm{d} x} \frac{\d F }{\d y'} = 0
$$
та граничній умові при $x = b$:
$$
\frac{\d }{
\d y'
} F(x, y(x), y'(x))\bigg|_{x=b} = 0
$$
\end{boxteo}
\begin{proof}
Для довільного $\alpha \in \mathbb{R}$, $\forall h\ in C^1 [a,b]:h(a)=0$ розглянемо функцію $\Phi (\alpha) = J(\tilde{y} = \alpha h)$.\par
Тоді для $\Phi (\alpha)$ справедливо, що:
$$
\Phi (\alpha) = J(\tilde{y} + \alpha h) \geq J(\tilde{y}) = \Phi(0)
$$
в деякому околі т. $\alpha = 0$ (якщо $\tilde{y}(x)$  -- забезпечує слабкий локальний мінімум).
Отже, т. $\alpha = 0$ - точка локального екстремуму функції $\Phi (\alpha) \Rightarrow \Phi' (0) = 0$.
$$
0 = \Phi'(0) = \frac{\mathrm{d} }{ \mathrm{d} \alpha} J(\ty - \alpha h) \bigg|_{\alpha = 0} =
$$
$$
= \frac{\mathrm{d}}{\mathrm{d} \alpha} \left(   \int\limits_{a}^{b}{
F(x, \ty(x) + \alpha h(x), \ty' (x) + \alpha h' (x) ) \mathrm{d}x
} \right) \bigg|_{\alpha = 0} =
$$
$$
=  \int\limits_{a}^{b}{ \frac{\d F(x, \ty, \ty')}{\d y} \cdot h + \frac{\d F(x, \ty, \ty')}{\d y'} \cdot h' } \mathrm{d} x \oeq
$$
Візьмемо другий додаток частинами:
$$
\oeq \frac{\d F(x,\ty, \ty')}{\d y'} \cdot h(x) \bigg|_{a}^{b} +  \int\limits_{a}^{b}{
\left(
\frac{\d F(x, \ty, \ty')}{\d y} - \frac{\mathrm{d}}{\mathrm{d} x} \left(
\frac{\d F(x, \ty, \ty')}{\d y'}
 \right)
 \right)\cdot h(x) \mathrm{d} x =
}
$$
$$
= \frac{\d F}{\d y'} (b, \ty(b), \ty'(b)) \cdot h(b) +  \int\limits_{a}^{b}{
\left( \frac{\d F}{ \d y} - \frac{\mathrm{d}}{\mathrm{d} x} \frac{\d F}{\d y'}    \right) h \mathrm{d} x = 0 \ \ (*)
}
$$
$$
\forall h \in C^1 [a,b] : h(a) = 0
$$
Дана рівність виконується в тому числі і при:
$$
\forall h \in C^1 [a,b] : h(a) = h(b) = 0.
$$
Тоді $\forall h \in C^1[a,b] : h(a) = h(b) = 0$:
$$
 \int\limits_{a }^{b}{ \left( \frac{\d F}{\d y} - \frac{\mathrm{d}}{\mathrm{d} x} \frac{\d F}{\d y'}    \right)
h \mathrm{d} x = 0
 }
$$
$\Longrightarrow $ внаслідок леми Лагранжа отримуємо рівняння Ейлера.\par
Доведемо граничну умову. Повернемося до рівності $(*)$:
$$
\forall h \in C^1 [a,b] : h(a) = 0:
$$
$$
(*) \ \Longrightarrow \ \frac{\d F}{\d y'} (b, \ty(b), \ty'(b)) \cdot h(b) =0 \ \Longrightarrow
$$
$\Longrightarrow \ $ (з довільного вибору $h(b)$) $\Longrightarrow$ гранична умова.
\end{proof}
\begin{remark}
  Розв'язки рівняння Ейлера називаються екстремалями. Розв'язки, які задовольняють умові $x(a) = A$ та граничній умові:
  $$
  \frac{\d F}{\d y'}\bigg|_{x=b} = 0
  $$
  називаються \textit{допустимими екстремалями}.
\end{remark}
\begin{remark}
  Аналогічно можна розглянути задачу з закріпленим верхнім кінцем та вільним нижнім кінцем $(x(b) = B)$. Тоді гранчна умова має виконуватись в т. $x=a$.\par
  Аналогічно можна розглянути задачу з двома вільними кінцями (тоді має виконуватись дві граничні умови).
\end{remark}
\begin{example}
 $$
 \begin{dcases}
  J(y) =  \int\limits_{0}^{1}{ (y'(x))^2 \mathrm{d} x};\\
  y(0) = 1.
 \end{dcases}
 $$
 \begin{enumerate}
   \item Складаємо і розв'язуємо рівняння Ейлера:
   $$
   \frac{\d F}{\d y} - \frac{\mathrm{d}}{ \mathrm{d} x} \frac{\d F}{\d y'}  = 0
   $$
   $$
   F=(y')^2 \qquad \frac{\d F}{\d y} = 0 \qquad \frac{\d F}{\d y'} = 2y'
   $$
   $$
 - \frac{\mathrm{d}}{\mathrm{d} x} (2y') = 0 \Longrightarrow   y' = C_1  \Longrightarrow \ \fbox{$y = C_1 x + C_2 $ -- сімейство екстремалей.}
   $$
   \item Знаходимо допустимі екстремалі:
   $$
   y(0) = 1 \quad \Longrightarrow \quad 1 = 0 + C_2 \quad \Longrightarrow \quad C_2 = 1
   $$
   $$
   \frac{\d F}{\d y'} \bigg|_{x=1}  = 0 \qquad 2y' (1) = 0 \qquad (y' = C_1)
   $$
   $$
   2C_1 = 0 \quad \Longrightarrow \quad C_1 = 0 \quad \Longrightarrow \quad \fbox{$\ty (x)= 1$ -- єдина допустима екстремаль.}
   $$
   \item Перевірка.
   $$
   \forall h\in C^1 [0,1] : h(0) = 0
   $$
   $$
   J(\ty + h) - J(\ty) =  \int\limits_{0}^{1}{ (\ty' + h')^2 \mathrm{d} x} -  \int\limits_{0}^{ 1}{ (\ty')^2 \mathrm{d} x} =  \int\limits_{0}^{1}{ (2\ty' h' + (h')^2) \mathrm{d} x} \oeq
   $$
   1-ий доданок беремо частинами: $2\ty' = y ; h' = v'$.
   $$
   \hspace*{-2em}\oeq 2\ty' h \bigg|_{0}^{1} +  \int\limits_{0}^{1}{-2 \ty'' h + (h')^2 \mathrm{d}x}
    = \left| \ \begin{gathered}
     h(0) = 0 \Rightarrow 2\ty' h(0) = 0;\\
     \ty (x) = C_1 \Rightarrow y' = y'' = 0
    \end{gathered} \ \right| =  \int\limits_{0}^{1}{ (h')^2 \mathrm{d} x} \geq  0
   $$
   $$
   \forall h \in C^1[0,1] : h(0) = 0 \quad \Longrightarrow \quad \text{Маємо глобальний мінімум.}
   $$
 \end{enumerate}
 \end{example}
 \subsection{Задача Больца.}
 Нехай $a,b \in \mathbb{R}, a < b$:
 \begin{itemize}
   \item $F(x,y,p) \in C^2([a,b] \times \mathbb{R}^2)$ -- задана функція.
   \item $f(u,v) \in C^1 (\mathbb{R}^2)$ -- задана функція.
 \end{itemize}
 В просторі $C^1[a,b]$ розглянемо інтеграл:
 $$
 J(y) =  \int\limits_{a}^{b}{F(x,y(x), y'(x))} \mathrm{d} x + f(y(a), y(b))
 $$
Означення слабкого локального (глобального) мінімуму (максимуму) аналогічні попереднім задачам.
\begin{defo}
 Задача пошуку слабкого локального (глобального) екстремуму функцаоналу $J(y)$ в просторі $C^1 [a,b]$ називається \textbf{задачею Больца}.
\end{defo}
\begin{boxteo}
Нехай $\ty (x) \in C^2  [a,b]$  -- розв'язок задачі Больца. Тоді $\ty(x)$ задовольняэ рівнянню Ейлера на $[a,b]$:
$$
\frac{\d F}{ \d y} - \frac{\mathrm{d}}{\mathrm{d} x} \frac{\d F}{\d y'} = 0
$$
та умови трансверсальності:
$$
\frac{\d }{\d y'} F (x, y(x), y'(x)) \bigg|_{x = a}  = \frac{\d f}{\d u} (y(a), y(b))
$$
$$
\frac{\d }{\d y'} F (x, y(x), y'(x)) \bigg|_{x = b}  = -\frac{\d f}{\d v} (y(a), y(b))
$$
\end{boxteo}
\begin{proof}
 $\forall \alpha \in \mathbb{R}, \forall h\in C^1 [a,b]$ розглянемо функцію:
 $$
 \Phi (\alpha) = J(\ty - \alpha h)
 $$
 Оскільки $\ty$ - слабкий локальний (глобальний) екстремум функціоналу $J(\ty)$, то точка $\alpha = 0$ -- т. локального екстремуму функції:
 $$\Phi (\alpha) \Longrightarrow 0 = \Phi'(0) = \frac{\mathrm{d}}{\mathrm{d} \alpha} J( \ty + \alpha h) \bigg|_{\alpha=0}= $$
 $$
 = \frac{\mathrm{d}}{\mathrm{d} \alpha} \left(
 \int\limits_{a}^{ b}{
 F(x,\ty+ \alpha h , \ty' + \alpha h')
 } \right) \mathrm{d} x  + f (\underbrace{\ty (a) + \alpha h(a)}_{u}, \underbrace{\ty (b) + \alpha h (b)}_{v})\bigg|_{\alpha = 0} =
 $$
 $$
  =  \int\limits_{a}^{b}{ \left( \frac{\d F}{\d y} (x,\ty, \ty') \cdot h + \frac{\d F}{\d y'} (x, \ty, \ty') \cdot h'   \right) \mathrm{d} x} +
 $$
 $$
 + \frac{\d f}{\d u}(\ty(a), \ty(b))\cdot h(a) + \frac{\d f}{\d v} (\ty (a), \ty(b)) \cdot h(b) \oeq
 $$
 Візьмемо 2-гий доданок під інтегралом частинами:
 $$
 \oeq \frac{\d F}{\d y'} (x,\ty, \ty') \cdot h\bigg|_a^b +  \int\limits_{ a}^{b}{ \left(
\frac{\d F}{\d y} (x ,\ty, \ty') - \frac{\mathrm{d}}{\mathrm{d} x} \frac{\d F}{\d y'} (x, \ty, \ty')
 \right) \cdot h \mathrm{d} x
 } +
 $$
 $$
 + \frac{ \d f}{\d u}  (\ty(a), \ty (b) )  \cdot h(a) + \frac{\d f}{\d v} (\ty(a), \ty(b)) \cdot h(b) = 0 \qquad (*)
  $$
  Дана рівність (*) виконується $\forall h \in C^1[a,b]$, в тому числі:
  $$
  \forall h \in C^1 [a,b] : h(a) = h(b) = 0
  $$
  Розглянемо саме цей випадок. Маємо:
  $$
   \int\limits_{a}^{b}{
   \left( \frac{\d F}{\d y} (x, \ty, \ty')  - \frac{\mathrm{d} }{\mathrm{d} x} \frac{\d F}{\d y'} (x,\ty, \ty')  \right)  \cdot h(x) \mathrm{d} x = 0
   }
  $$
  З леми Лагранжа отримуємо рівняння Ейлера:
  $$
  \frac{\d F}{ \d y} - \frac{ \mathrm{d}}{\mathrm{d} x} \frac{\d F}{ \d y'} = 0 \ \text{ для } \  \ty(x) \text{ на } [a,b]
  $$
  Повернемося до рівності (*) $ \forall h \in C^1[a,b]$. Маємо:
  $$
  \frac{\d F}{\d y'} (b, \ty(b) , \ty'(b))\cdot h(b) - \frac{\d F}{\d y'} (a, \ty(a), \ty'(a))\cdot h(a) +
  $$
  $$
  + \frac{\d f}{\d u} (\ty(a), \ty(b)) \cdot h(a)+ \frac{\d f}{\d n}   (\ty(a), \ty(b)) \cdot h(b) = 0 \Longrightarrow
  $$
  $$
  \Longrightarrow \left[
   \frac{\d F}{\d y'} (b, \ty(b) , \ty'(b)) + \frac{\d f}{\d n}   (\ty(a), \ty(b))
    \right] \cdot h(b) +
   $$
   $$
   + \left[
- \frac{\d F}{\d y'} (a, \ty(a), \ty'(a))+ \frac{\d f}{\d u} (\ty(a), \ty(b))
    \right]\cdot h(a) = 0
   $$
   З довільності вибору $h(x) \in C^1 [a,b]$ отримуємо умови трансверсальності.
\end{proof}
\begin{remark}
  Аналогічно всі розв'язки рівняння Ейлера називаютсья \textit{екстремалями}.
  Розв'язи, які задовольняють умовам трансверсальності --- \textit{допустимими екстремалями}.
\end{remark}

\begin{example}
  $$
   \int\limits_{0}^{1}{ (y')^2 - y} \mathrm{d} x + \frac{y^2(1)}{2}  \longrightarrow \mathrm{extr}
  $$
\begin{spacing}{1}
\begin{enumerate}
  \item Складаємо і розв'язуємо рівняння Ейлера:
  $$
  \frac{\d F}{\d y}  - \frac{\mathrm{d} }{\mathrm{d} x} \frac{\d F}{\d y'} = 0
  $$
  $$
  -1 - \frac{\mathrm{d}}{\mathrm{d} x} (2y') = 0
  $$
  $$
  -1 - 2y'' = 0 \quad \Longrightarrow \quad y'' = \frac{1}{2}
  $$
  $$
  y' = - \frac{1}{2} x + C_1
  $$
  $$
  y = - \frac{1}{4} x^2 + C_1 x + C_2 \text{ --- сімейство екстремалей.}
  $$
  \item Знаходимо допустимі екстремалі з умов трансверсальності:
  $$
  \frac{\d F}{\d y' } \bigg|_{x=0} = \frac{\d f}{ \d u} (y(0), y(1))
  $$
  $$
  \frac{\d F}{\d y' } \bigg|_{x=1} = - \frac{\d f}{ \d v} (y(0), y(1))
  $$
  $$
  \frac{\d F}{\d y'} =2 y' = 2 \left( - \frac{1}{2} x + C_1  \right) = - x + 2C_1
  $$
  $$
\frac{\d F}{\d y'} (0) = 2C_1 \qquad \frac{\d F}{\d y'} (1) = -1 + 2 C_1
  $$
  $$
  f(u,v) = \frac{v^2}{2} \qquad \frac{\d f}{ \d u} = 0 \qquad \frac{\d f}{ \d v} = v
  $$
  $$
  \frac{\d f}{\d v} (y(0), y(1)) = y(1) = - \frac{1}{4} + C_1 + C_2
  $$
  $$
  \frac{\d F}{\d y' } \bigg|_{x =0 } = \frac{\d f}{\d u} (y(0), y(1)) \ \Longrightarrow \  2 C_1 = 0 \ \Longrightarrow \  C_1 = 0
  $$
  $$
  \frac{\d F}{\d y' }\bigg|_{x=1} =   \frac{\d f}{\d u} (y(0), y(1))
  \ \Longrightarrow \   -1 + 2C_1 = \frac{1}{4} - C_1 - C_2
  $$
  $$
  -1 = \frac{1}{4} - C_2  \ \Longrightarrow \   C_2 = \frac{1}{4} + 1 = \frac{5}{4}
  $$
  Остаточно, $\ty(x) = - \frac{1}{4} x^2 + \frac{5}{4}  $ --- єдина допустима екстремаль.
  \item Перевірка $\forall h \in C^1 [0,1]$:
  $$
    \hspace*{-2em}
  J(\ty + h) - J(
  \ty
  ) \!=\!\!  \int\limits_{0}^{1} (\ty'+ h')^2 - (\ty - h)\  \mathrm{d} x +  \frac{(\ty(1) + h(1))^2}{ 2} - \! \int\limits_{0}^{1}{(\ty')^2 - \ty\  \mathrm{d}x} - \frac{(\ty (1))^2}{2}\! =\!
  $$
  $$
  =  \int\limits_{0}^{ 1}{\underbrace{2\ty'}_{u}\underbrace{h'}_{v'} + (h')^2 - h} \mathrm{d} x  + \ty(1)h(1) + \frac{h^2 (1)}{2}  =
  $$
  $$
=2\ty' h\bigg|_{0}^1 +  \int\limits_{0}^{1}{ -2 \ty'' h + (h')^2 - h \mathrm{d} x} + \ty(1) h(1) + \frac{h^2}{2} \oeq
  $$
  $$
  \left| \
  \begin{gathered}
    \ty(1) = \frac{5}{4} - \frac{1}{4} = 1 \quad  \ty'(x) = - \frac{1}{4} \cdot 2 x = - \frac{x}{2}  \\
     \ty'(0)=0 \quad \ty'(1) = - \frac{1}{2} \quad \ty''(x) = -\frac{1}{2}
  \end{gathered}
   \right|
  $$
  $$
  \oeq - h(1) +  \int\limits_{0}^{1}{ h + (h')^2 - h \ \mathrm{d} x} + h(1) +
  \frac{h^2(1)}{2}
  =  \int\limits_{0}^{1}{(h')^2 \mathrm{d} x} + \frac{h^2(1)}{2} \geq 0
  $$
  $
  \forall h \in C^1[0,1] \Longrightarrow
  $ маємо \textbf{глобальний мінімум.}
\end{enumerate}
\end{spacing}
\end{example}

\subsection{Векторні задачі.}
Розглянемо простір:
$$
C_n^1[a,b] = \left\lbrace \overrightarrow{y} (x) = (y_1(x), \dots , y_n(x)) \ \bigg| \  y_i(x) \in C^1[a,b] \ \forall i = \overline{1, n} \right\rbrace
$$
$
C_n^1 [a,b]
$ --- нормований простір з нормою:
$$
||\overrightarrow{y}||_{C_n^1 [a,b] } = \max\limits_{x\in[a,b]}||\overrightarrow{y}(x)||_{\mathbb{R}^n} +
 \max\limits_{x\in[a,b]}||\overrightarrow{y}'(x)||_{\mathbb{R}^n}
$$
\subsubsection{Найпростіша векторна задача.}
Нехай $a,b \in \mathbb{R} , a < b$ -- задані числа.
$$
F(x, \overrightarrow{y}, \overrightarrow{p}) = F(x, y_1, \dots, y_n, p_1, \dots, p_n)\in C^2 ([a,b] \times \mathbb{R}^{2n}) \text{ -- задана функція.}
$$
Розглянемо інтеграл:
$$
J(\overrightarrow{y}) = J(y_1, \dots, y_n) =  \int\limits_{a}^{b}{F(x,\overrightarrow{y}(x), \overrightarrow{y}'(x)) \mathrm{d} x} =
$$
$$
=  \int\limits_{a}^{ b}{ F(x, y_1 (x), \dots, y_n(x), y_1'(x), \dots, y_n'(x)) \ \mathrm{d} x}
$$
де $\overrightarrow{A} = (A_1, \dots , A_n) ; \overrightarrow{B} = (B_1, \dots , B_n)$ -- задані, функції $\overrightarrow{y} \in M$ --- допустимі. \par
Означення сабкого локального (глобального) мінімуму (максимуму) аналогічні попереднім задачам (тільки норма береться в просторі $C_n^1[a,b]$).
\begin{defo}
 Задача пошуку екстремуму функціоналу $J(y)$ на $M$ називається \textit{векторною найпростішою задачею.}
\end{defo}

\begin{boxteo}
 Нехай функція  $\tvy (x) \in C_n^2 [a,b]$ є розв'язком найпростішої векторної задачі. Тоді $\tvy (x)$ задовільняє
 \textit{систему рівнянь Ейлера} на $[a,b]$:
 $$
 \frac{\d F}{\d y_i } - \frac{\mathrm{d}}{\mathrm{d} x} \frac{\d F}{\d y_i'} = 0 \qquad i = \overline{1, n}
 $$
\end{boxteo}
\begin{proof}
 Нехай $\tvy(x) = \ty_1 (x),  \dots , \ty_n (x)$ -- розв'язок. Зафіксуємо в $J(\overrightarrow{y})$:
 $$
 y_2 (x) = \tilde{y}_2(x), y_3 (x) = \tilde{y}_3(x), \dots , y_n (x) = \tilde{y}_n(x)
 $$
 Таким чином, отимуємо найпростішу задачу відносно $y_1$:
 $$
 \begin{dcases}
  J(y_1) =  \int\limits_{a}^{ b}{F(x, y_1(x), \ty_2(x), \dots, \ty_n(x), y_1'(x), \ty_2'(x), \dots, \ty_n'(x))} \mathrm{d} x \to \mathrm{exrt};\\
  y_1(a) = A_1 \qquad y_1 (b) = B_1
 \end{dcases}
 $$
 Тоді $\ty_1(x)$ -- розв'язок даної задачі, тож $\ty_1(x)$  задовольняє р-ня Ейлера на $[a,b]$:
 $$
 \frac{\d F}{\d y_1} - \frac{\mathrm{d}}{ \mathrm{d} x} \frac{\d F}{\d y_1'} = 0
 $$
 Аналогічно зафіксувавши $ y_1 (x) = \tilde{y}_1(x), y_3 (x) = \tilde{y}_3(x), \dots , y_n (x) = \tilde{y}_n(x)$,  отримаємо найпростішу задачу для $y_2(x)$ та виконяння рівняння Ейлера для $\ty_2 (x)$. Продовживши аналогічні міркування, отримаємо систему рівнянь Ейлера.
\end{proof}
\newpage
\subsubsection{Векторна задача Больца.}
Нехай $a,b \in \mathbb{R}; a<b$ -- задані числа.
$$
F(x,\overrightarrow{y}, \overrightarrow{p}) \in C^2([a,b]\times \mathbb{R}^{2n}) \text{ -- задана функція.}
$$
$$
f(\overrightarrow{u}, \overrightarrow{v})\in C^1(\mathbb{R}^{2n}) \text{ -- задана функція.}
$$
В просторі $C_n^1[a,b]$ розглянемо функціонал:
$$
J(\overrightarrow{y}) = J(y_1, \dots , y_n) =  \int\limits_{a}^{b}{ F(x, \overrightarrow{y}(x), \overrightarrow{y}'(x))} \mathrm{d} x + f(\overrightarrow{y}(a), \overrightarrow{y}(b))
$$
\begin{defo}
 Задача пошуку слабкого лок. (глобального) екстремуму функціоналу $J(\overrightarrow{y})$ в просторі $C_n^1 [a,b]$ називається \textbf{ векторною задачею Больца}.
\end{defo}
\begin{boxteo}
 Нехай вектор-функція $\tvy (x) \in C_n^2 [a,b]$ є розв'язком задачі Больца. Тоді $\tvy (x)$ задовольняє систему рівнянь Ейлера на $[a,b]$:
 $$
 \frac{\d F}{\d y_i} - \frac{\mathrm{d}}{ \mathrm{d} x} \frac{\d F}{\d y_i'}  = 0 \qquad \forall i = \overline{1, n}
 $$
 та умови \textit{трансверсальності}:
 $$
 \frac{\d}{\d y_i'} F(x, \overrightarrow{y}(x),  \overrightarrow{y}'(x))\vline_{x =a} = \frac{\d f}{\d u_i} (\overrightarrow{a}, \overrightarrow{y}(b)) \qquad i = \overline{1,n}
 $$
 $$
  \frac{\d}{\d y_i'} F(x, \overrightarrow{y}(x),  \overrightarrow{y}'(x))\vline_{x =b} =- \frac{\d f}{\d v_i} (\overrightarrow{a}, \overrightarrow{y}(b)) \qquad i = \overline{1,n}
 $$
 \begin{proof}
  Аналогічно попередній теоремі.
 \end{proof}
\end{boxteo}
