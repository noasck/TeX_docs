\subsection{Метод функцій Ляпунова.}

Розглянемо систему:
\be
 \overrightarrow{x}' = f( \overrightarrow{x})
\ee
дe $f: D \to \mathbb{R}^n, f\in C^{1} \left( D \right), D \subset \mathbb{R}^n $ та $\overrightarrow{f} ( \overrightarrow{0}) = 0$.
\textbf{Задача. } Дослідити на стійкість розв'язок $ \overrightarrow{x} = \overrightarrow{0}$ системи (1).

 \begin{defo}
  Функція $V : D \to \mathbb{R} , V \in C(D) $ називається додатньо(від'ємно)-визначеною в $D$, якщо:
  \\1. $V(0) = 0$;\\
  2. $ V(x) > 0 \ (<0) \ \forall x \in D(\left\lbrace 0 \right\rbrace)$;
 \end{defo}

 \begin{defo}
Похідною функції $ V : D \to \mathbb{R}$ в суму системи (1) називають функцію:
$$
\dot{V}_f(\overrightarrow{x}) =  \sum\limits_{i = 1}^{n}{ \frac{\partial V}{ \partial x_i} f_{i}( \overrightarrow{x}) } = <\nabla V \overrightarrow{x}, f(\overrightarrow{x})>
$$
 \end{defo}

 \begin{defo}
  Нехай $B_{R} ( \overrightarrow{0} ) $ - деякий окіл точки $ \overrightarrow{0} $. Функція $V \in C^{1} (B_{R} (\overrightarrow{0}))$ називається функцією Ляпунова системи (1), якщо:\\
  1. $ V $ - доданьо-визначена. \\
  2. $ \dot{V}_f (\overrightarrow{x}) \leq 0 \quad x \in B_{R} ( \overrightarrow{0})$.
 \end{defo}

 \begin{remark}
     Якщо $ V $ - від'ємно-визначена і $ \dot{V}_f ( \overrightarrow{x}) \geq 0$, то:
     $$
     - V ( \overrightarrow{x} ) - \text{ функція Ляпунова. }
     $$
 \end{remark}

 \begin{boxteo}[Ляпунова про стійкість]
   Якщо в деякій кулі $B_{R}( \overrightarrow{0}) $ існує функція Ляпунова для системи (1), то розв'язок $ \overrightarrow{x} = \overrightarrow{0} $ системи (1) стійкий.
 \end{boxteo}
\begin{proof}
 Нехай $ V \in C^{1} ( B_{R} ( \overrightarrow{0} ))$ - функція Ляпунова.\\
 Доведемо, що розв'язок стійкий за означенням.
 Візьмемо $ \forall \varepsilon > 0, \varepsilon < \mathbb{R}$.\\ Покладемо $ c( \varepsilon) = \min\limits_{\overrightarrow{x} : ||\overrightarrow{x}|| = \varepsilon} V(\overrightarrow{x}) $.
 Тоді $ c(\varepsilon) > 0, $ бо $ V( \overrightarrow{x}) $ - додатньо-визначена.\\
 Виберемо $ \delta > 0 : 0 < \delta < \varepsilon $ таким чином, щоб: $ \forall \overrightarrow{x}: ||\overrightarrow{x}|| < \delta $ справдовується:\\
 $V (\overrightarrow{x}) < C(\varepsilon) $ ( таке $\delta$ існує в силу неперервності $V( \overrightarrow{x})$ та $V ( \overrightarrow{x}) =0 $).\\
 Візьмемо $\forall \overrightarrow{x}_0: ||\overrightarrow{x}_0||< \delta $ і розглянемо розв'язок $\overrightarrow{x} (t) $ з початковими умовами $ \overrightarrow{x}(t_0) = \overrightarrow{x}_0$.\\
 Потрібно показати за означенням, що:
 $$
 ||\overrightarrow{x}(t)|| < \varepsilon \quad \forall t \geq t_0
 $$
Припустимо протилежне. Нехай $ \exists t_1 > t_0 $ таке, що $|| \overrightarrow{x}(t) ||< \varepsilon \ \forall t \in [ t_0 , t_1 ]:  $ $$ || \overrightarrow{ x} (t_1) || = \varepsilon$$
Тоді за вибором $c(\varepsilon)$ справедливо, що:
$$
|| \overrightarrow{x} (t_1) || = \varepsilon
$$
Тоді за вибором $ c ( \varepsilon)$ справедливо, що:
$$
V(\overrightarrow{x} (t_1) ) \geq c(\varepsilon)
$$
З іншого боку:
$$
\frac{d}{dt} V(\overrightarrow{x} (t)) =  \sum\limits_{i = 1}^{ n}{ \frac{\partial V}{ \partial x_i } } \cdot \frac{dx_i}{dt} =  \sum\limits_{i = 1}^{n}{ \frac{\partial V}{ \partial x_i } f_i ( \overrightarrow{x} (t))} = \dot{V}_f (x) \leq 0 \  (\text{за умовою теореми})
$$
Отже, $V (\overrightarrow{x}(t))\!\! \downarrow \  \!\Rightarrow\!
c'(\varepsilon) \!\leq\!  V(\overrightarrow{x} (t_1)) \!\leq\! V(\overrightarrow{x} (t_0)) \!=\! V(\overrightarrow{x}_0) \!<\! c(\varepsilon ) \!\Rightarrow\! \text{ протирічча. }
$

\end{proof}


\begin{boxteo}[Ляпунова про асимптотичну стійкість]\ \\ Нехай в деякому околі $B_R(\overrightarrow{0})$ існує функція Ляпунова для системи (1), причому $\dot{V}_f (\overrightarrow{x})$ - від'ємно-визначена.
Тоді розв'язок $\overrightarrow{x} = \overrightarrow{0}$ системи (1) є \textbf{асимптотично стійким.}
\end{boxteo}

\begin{proof}
  По-перше, зауважимо, що існує функція Ляпунова.\\
  Отже, за попередньою теоремою розв'язок $\overrightarrow{x}  = \overrightarrow{0} $ принаймі стійкий, тобто:
  $$
  \forall \varepsilon  >0 \ \forall t_0 \ \exists \delta > 0 : \forall \overrightarrow{x}_0 : ||\overrightarrow{x}_0|| < \delta \ \text{справедливо, що:}
  $$
  $$
  \text{Для розв'язку з початковою умовою } \overrightarrow{x}(t_0) = \overrightarrow{x}_0 : ||\overrightarrow{x}(t)|| < \varepsilon \ \  \forall t \geq t_0
  $$
  По-друге, відзначимо, що:
  $$
  \frac{d}{dt} V (\overrightarrow{x}(t)) = \dot{V}_f ( \overrightarrow{x}(t)) \ \text{ - від'ємно-визначена.}
  $$
  Отже, $V( \overrightarrow{x} (t))$ спадає і $V(\overrightarrow{x}(t))$  - додатньо-визначена.\\
  Потрібно довести, що: $$||\overrightarrow{x} (t)|| \xrightarrow[t \to +\infty]{} 0$$
  Припустимо протилежне. Нехай $|| \overrightarrow{x}(t)|| \xrightarrow[t \to + \infty]{} a $, де $a > 0$. Тоді:
  $$
  \forall t \geq t_0 \ ||\overrightarrow{x})t||>0
  $$
  Якщо ж $\exists t_1: ||\overrightarrow{x}(t_1)|| = 0 \Rightarrow V(\overrightarrow{x}(t_1)) = 0 \Rightarrow \begin{cases}
   V(\overrightarrow{x}(t)) \! \downarrow \\
   V(\overrightarrow{x}(t)) \text{ - дод.-визнач.}
  \end{cases}$, то:
$$
V(\overrightarrow{x}(t))  = 0 \ \ \forall t \geq t_1\ \Rightarrow\ \overrightarrow{x}(t) = 0 \ \ \forall t \geq t_1\ \Rightarrow\ ||\overrightarrow{x}(t)|| \xrightarrow[ t
\to +\infty]{} 0
$$
Але, за припущенням $||\overrightarrow{x}(t)||$ не збігається до 0 при $t \to + \infty$.\\
Отже, $\forall t \geq t_0: ||\overrightarrow{x}(t)|| > 0$. Отримали:
$$
\exists \varepsilon _1 > 0 : \varepsilon_1 \leq ||\overrightarrow{x} (t)|| < \varepsilon  \quad \forall t \geq t_0
$$
Покладемо:
$$
m = \max\limits_{\overrightarrow{x}: \varepsilon_1 \leq  ||\overrightarrow{x}|| \leq \varepsilon } \dot{V}_f ( \overrightarrow{x} ) < 0 \ \  \Longleftarrow \ \ \dot{V}_f (\overrightarrow{x}) \text{ - від'ємно-визначена }
$$
Тоді:
$$
\frac{d}{dt} V(\overrightarrow{t}) = \dot{V}_f (\overrightarrow{x}(t))  \leq m
$$
Проінтегруємо від $t_0$ до $t$ нерівність:
$$
\frac{d}{dt} V(\overrightarrow{x} (t)) \leq m \ \Longrightarrow \  V(\overrightarrow{x} (t)) - V(\overrightarrow{x} (t_0)) \leq  m (t-t_0)
$$
$$
V(\overrightarrow{x}(t)) \leq \underbrace{mt}_{<0} - mt_0 + V(\overrightarrow{x}_0) \xrightarrow[t \to + \infty]{} -\infty \ \Longrightarrow \ V(\overrightarrow{x} (t)) \xrightarrow[t\to+\infty]{}  -\infty
$$
А оскільки $V(\overrightarrow{x})$ - додатньо-визначена, то отримаємо протирічча.
\end{proof}
\begin{boxteo}[Ляпунова про нестійкість] \ \\Нехай $\exists V \in C^{1} (B_R (\overrightarrow{0}))$ така, що:\\
1. $V(\overrightarrow{0}) = 0$\\
2. $\dot{V}_f $ - додатньо-визначена в $B_R (\overrightarrow{0})$\\
3. Для як завгодно малого $\delta > 0 $ знайдеться $\overrightarrow{x}_0 : ||\overrightarrow{x}_0|| < \delta $ i $V (\overrightarrow{x}_0) > 0 $. \\
(тобто $\dot{V}_f$ та $V$ одного знаку для $\overrightarrow{x}_0$)\\
Тоді, розв'язок $\overrightarrow{x} = \overrightarrow{0} $ системи (1) нестійкий.

\end{boxteo}
\begin{proof}
 За означенням потрібно показати, що $\exists \varepsilon : \forall > 0 $ знайдеться\\
 $\overrightarrow{x}_0 : ||\overrightarrow{x}_0|| < \delta$, але для розв'язку з початковою умовою $\overrightarrow{x}(t_0) = \overrightarrow{x}_0 \ \ \exists t_1 > t_0 $ таке, що $||\overrightarrow{x} (t_1)||>\varepsilon $.
 Візьмемо довільне $ \varepsilon > 0 \ \varepsilon  < R  $ та довільне $\delta > 0$. \\
 Тоді за умовою теореми $\exists \overrightarrow{x}_0 : ||\overrightarrow{x}_0|| < \delta $ і $V(\overrightarrow{x}_0) > 0$.\\
Розглянемо $\overrightarrow{x}(t)$ - розв'язок з початковими умовами $\overrightarrow{x}(t_0) = \overrightarrow{x}_0$.\\
Покажемо, що $\exists t_1 : ||\overrightarrow{x}(t_1)|| > \varepsilon $. \\
Припустимо, що це не так. Нехай $\forall t \geq t_0 : ||\overrightarrow{x}(t)|| \leq \varepsilon $. Нехай:\\
$$
\forall t \geq t_0 : ||\overrightarrow{x}(t)|| \leq \varepsilon
$$
Зауважимо, що:
$$
\frac{d}{dt} V(\overrightarrow{x}(t)) = \dot{V}_f (\overrightarrow{x}(t)) >0 \Longrightarrow V(\overrightarrow{x}(t)) \! \uparrow \Longrightarrow
$$
$$
\Longrightarrow \forall t \geq t_0 : V(\overrightarrow{x}(t)) \geq  V(\overrightarrow{x}_0) := \alpha > 0 \Longrightarrow
$$
$$
\Longrightarrow (\text{оскільки } V(\overrightarrow{0}) = 0) \Longrightarrow \exists \varepsilon_0 >0 : \forall t \geq t_0 : \varepsilon _0 \leq ||\overrightarrow{x}(t)|| \leq \varepsilon
$$
Покладемо:
$$
m = \min\limits_{\overrightarrow{x} : \varepsilon_1 \leq ||\overrightarrow{x}|| \leq \varepsilon } \dot{V}_f(\overrightarrow{x}) > 0
$$
Отже, $ \frac{d}{dt}V(\overrightarrow{x}(t))  = \dot{V}_f(\overrightarrow{x}(t)) \geq m $. Проінтегруємо від $t_0$ до $t$:
$$
V(\overrightarrow{x}(t)) - V(\overrightarrow{x}(t_0)) \geq m (t -t_0)
$$
$$
V(\overrightarrow{x}(t)) \geq  mt - mt_0 + V(\overrightarrow{x}_0) \xrightarrow[t\to+\infty]{}+ \infty
$$
$$
V(\overrightarrow{x}(t)) \xrightarrow[t\to+\infty]{} + \infty
$$
Отже, $V(\overrightarrow{x}(t))$ є необмеженою на обмеженій множині $\left\lbrace \overrightarrow{x} \Big| ||\overrightarrow{x}|| \leq \varepsilon  \right\rbrace$ і $V$ - неперервна в цій множині.
Отримали протирічча $\bigotimes$.
\end{proof}

\begin{boxteo}[Четаєва про нестійкість]
  Нехай $V \in C^{1} (D):$
  \begin{enumerate}
  \item $V(\overrightarrow{0}) = 0$
  \item $\exists D_+ \subset D$:
  \begin{enumerate}
      \item $V(\overrightarrow{x} ) > 0 \quad \forall \overrightarrow{x} \in D_+$
      \item $\overrightarrow{0} \in \partial D_+ $
      \item $V(\overrightarrow{x}) = 0 \quad \forall \overrightarrow{x} \in \partial D_+ \cap D$
      \item $\dot{V}_f (\overrightarrow{x})>0 \quad \forall \overrightarrow{x }\in D_{+}$
  \end{enumerate}
  \end{enumerate}
  Тоді розв'язок $\overrightarrow{x} = \overrightarrow{0} $ нестійкий.
\end{boxteo}
\begin{remark}
    На жаль, не існує жодних методів та схем пошуку функції Ляпунова в загальному випадку. Але іноді, в дуже частинних випадках може допомогти метод розділення змінних. Розглянемо його на прикладах.
\end{remark}
\begin{example}
  $$
  \begin{cases}
   \dot{x} = x^3 - y \\
   \dot{y} = x + y^3
  \end{cases}
  $$
  Дослідити на стійкість розв'язок $\begin{bmatrix}
   x \\
   y
  \end{bmatrix} = \begin{bmatrix}
   0\\
   0
  \end{bmatrix}.$ Спробуємо за 1-м наближенням:
  $$
  A = \left. \begin{bmatrix}
   \frac{\partial f_1}{\partial x} & \frac{\partial f_1}{\partial y}\\
   \frac{\partial f_2}{\partial x} & \frac{\partial f_2}{\partial y}
  \end{bmatrix}  \right|_{(0,0)} = \left. \begin{bmatrix}
   3x^2 & -1 \\
   1 & 3y^2
  \end{bmatrix} \right|_{(0,0)} = \begin{bmatrix}
   0 & -1 \\
   1 & 0
  \end{bmatrix}
  $$
  $$
  \det(A - \lambda I) = \begin{vmatrix}
   -\lambda & -1 \\
   1 & - \lambda
   \end{vmatrix} = \lambda^2 + 1 = 0 \Rightarrow \fbox{$\lambda = \pm i$} \Rightarrow \Re{\lambda} = 0 \Rightarrow \text{критичний випадок.}
  $$
Спробуємо метод функцій Ляпунова. Спробуємо знайти $V(x,y) = A(x) + B(y)$. Функції $A(x)$ та $B(y)$ підберемо так, щоб $V(x,y) ; \dot{V}_f(x,y)$ була знаковизначеною. Маємо:
$$
\dot{V}_f (x,y) = \frac{\partial V}{\partial x} f_1 (x,y ) + \frac{\partial V}{ \partial y} f_2 (x,y) \Longrightarrow
$$
$$
\dot{V}_f (x,y)=  A'(x)(x^3-y) + B'(y) (x+y^3) = A'(x)x^3 - A' (x) y + B'(y) x + B'(y)y^3
$$
Підберемо $A(x)$ та $B(x)$ так, щоб $A'(x)y = B'(y)x$:
$$
\begin{cases}
 B'(y)= y\\
 A'(x) = x
\end{cases}
$$
Візьмемо $B(y) = \dfrac{y^2}{2} \ \  A(x) = \dfrac{x^2}{2} $. Тоді:
$$
V(x,y) = \frac{x^2}{2} + \frac{y^2}{2} \ \text{ - додатньо-визначена.}
$$
$$
\dot{V}_f (x,y) = x^4 + y^4  \ \text{ - додатньо-визначена.}  \Longrightarrow
$$
$\Longrightarrow $ розв'язок нестійкий (бо $V$ та $\dot{V}_f$ одного знаку).
\end{example}

\begin{example}
 $$
 \begin{cases}
\dot{x} = y-x+xy\\
\dot{y} = x-y-y^3 -x^2
 \end{cases}
 $$
 Спробуємо за першим наближенням:
 $$
 A = \left. \begin{vmatrix}
  -1+y & 1 + x \\
  1 - 2x & -1-3y^2
 \end{vmatrix} \right|_{(0,0)} = \begin{bmatrix}
  -1 & 1 \\
  1 & -1
 \end{bmatrix}
 $$
 $$
 \det(A - \lambda I) = \begin{vmatrix}
   -1 - \lambda & 1 \\
   1 & -1-\lambda
 \end{vmatrix} = (\lambda + 1)^2  - 1 = 0 \Rightarrow \begin{gathered}
  \lambda_1 = 0 \\
  \lambda_2 = -2
 \end{gathered}
 $$
 $\exists \lambda : \Re \lambda = 0 \Longrightarrow $ маємо критичний випадок.\\
Спробуємо метод функцій Ляпунова. Знайдемо $V(x,y) = A(x) + B(x)$:
$$
\dot{V}_f (x,y) = A'(x)(y-x+xy) + B'(y) (x-y-y^3 -x^2) =
$$
$$
= A'(x)y - A'(x)x + A'(x)xy + B'(y)x - B'(y)y - B'(y)y^3 - B'(y)x^2
$$
Підберемо $A(x)$ та $B(y)$ так, щоб $
A'(x)y = B'(y)x
$:
$$
\begin{cases}
 A'(x)  = x\\
 B'(y) = y
\end{cases}
$$
Візьмемо $A(x) = \dfrac{x^2}{2}\ \  B(y) = \dfrac{y^2}{2} $ Тоді:
$$
V(x,y) = \frac{x^2}{2} + \frac{y^2}{2} \ \text{ - додатньо-визначена.}
$$
$$
\dot{V}_f (x,y) = xy - x^2 + x^2 y  + xy - y^2 - y^4 - x^2 y =
$$
$$
= -x^2 + 2xy - y^2 -y^4 = -(x-y)^2 - y^4 \ \text{ - від'ємно-визначена.}  \Longrightarrow
$$
\end{example}

\begin{example}
 $$
 \begin{cases}
  \dot{x } = -xy^2\\
  \dot{y} = 3yx^2
 \end{cases}
 $$
 Спробуємо метод функцій Ляпунова:
 $$
 V(x,y) = A(x) + B(y)
 $$
 $$
 \dot{V}_f (x,y) = A'(x) \cdot (-xy^2) + B'(y) 3yx^2
 $$
 Виберемо $A(x)$ та $B(y)$:
 $$
 A'(x) \cdot xy^2 = 3 B'(y) yx^2 \Longrightarrow \begin{cases}
  A'(x)  = 3x\\
  B'(y) = y
 \end{cases}
 $$
Візьмемо $A(x) = \dfrac{3x^2}{2} \ \  B(y) = \frac{y^2}{2} $. Тоді:
$$
V(x,y) = \frac{3x^2}{2} + \frac{y^2}{2}  \ \text{ - додатньо-визначена.}
$$
$$
\dot{V}_f (x,y) = 0 \leq 0 \Longrightarrow \text{ розв'язок стійкий.}
$$
\end{example}
